\chapter{Pozemkové úpravy}
\label{pu}

Tato kapitola se věnuje pozemkovým úpravám s~důrazem na~části, které
se týkají vytvořeného zásuvného modulu. Cílem není obsáhnout všechny
informace o~pozem\-kových úpravách, nýbrž nastínit základní principy
a~myšlenky souvisejících s~tématem práce.

V~této kapitole bylo čerpáno ze~zákona o~pozemkových úpravách
\citep{pu_zakon}, metodic\-kého návodu \citep{metodicky_navod}
a~dalších zdrojů \citep{pu_cr}~\citep{pu_skripta}.

Pojmem pozemkový úřad se v~rámci této kapitoly rozumí pobočka
krajského pozemkového úřadu.

\section{Pojem pozemkových úprav}
\label{pojem_pu}

Pozemkové úpravy zahrnují mnoho na~sebe navazujících činností, jejichž
společným cílem je zlepšení podmínek pro~zemědělské hospodaření,
zpřístupnění pozemků, zmírnění nepříznivých účinků vodní a~větrné
eroze, zlepšení životního prostředí, zvýšení ekologické stability
krajiny a~zachování či~obnova krajinného rázu. Děje se tak pomocí
prostorového a~funkčního uspořádávání pozemků, pozemky se dělí
a~scelují. K~pozemkům se vyhotovují vlastnická práva a~s~tím
související věcná břemena. Výsledky pozemkových úprav slouží jako
podklady pro~obnovu katastrálního operátu.

Pozemkové úpravy jsou multidisciplinární obor, který využívá znalostí
a~poznatků z~mnoha dalších sfér. Mezi ně patří zemědělství, krajinné
a~územní plánování, geodézie, fotogrammetrie, vodohospodářství,
ochrana životního prostředí, katastr nemovitostí a~další.

\section{Význam pozemkových úprav}
\label{vyznam_pu}

Pozemkové úpravy mají význam jak pro~účastníky pozemkových úprav~–
vlastníky, stavebníky, obce~– tak pro~obyvatele a~návštěvníky venkova,
orgány státní správy, podnikatelské subjekty, správce inženýrských
sítí a~zájmové organizace. Ve~výsledku mají tedy pozemkové úpravy
dopad na~životy jednotlivců, společnosti a~celého státu.

Význam \zk{PU} pro vlastníky a~nájemce půdy:
\vspace{-\topsep}
	\begin{itemize}[leftmargin=1.5cm, noitemsep]
		\item přehledné a~jasné vlastnické vztahy
		\item vytyčené hranice pozemků v~terénu
		\item zajištěný přístup na~pozemky
		\item lepší tvar pozemků vhodných pro~racionální zemědělské hospodaření
		\item možnost uzavřít nájemní smlouvy na~přesné výměry a~hranice pozemků
		\item lepší organizace půdní držby
		\item zvýšená tržní cena pozemků
	\end{itemize}

Význam \zk{PU} pro zemědělské subjekty:
\vspace{-\topsep}
	\begin{itemize}[leftmargin=1.5cm, noitemsep]
		\item lepší tvar pozemků vhodných pro~racionální zemědělské hospodaření
		\item zajištěný přístup na~pozemky
		\item možnost uzavření nájemních smluv na~přesné výměry a~hranice pozemků
		\item možnost žádat o~dotace
	\end{itemize}

Význam \zk{PU} pro obce:
\vspace{-\topsep}
	\begin{itemize}[leftmargin=1.5cm, noitemsep]
		\item vyjasněné právnické vztahy v~území
		\item zpřístupnění a~zprůchodnění krajiny
		\item nalezení a~zapsání historického majetku obce
		\item podrobná dokumentace o~území
		\item realizace společných zařízení za~státní peníze
		\item podklad pro zpracování územního plánu
		\item zvýšená ekologická stabilita území
		\item protipovodňová ochrana obce
		\item podpora pěší turistiky a~cykloturistiky
		\item zkvalitnění života na~venkově
	\end{itemize}

Význam \zk{PU} pro orgány státní správy:
\vspace{-\topsep}
	\begin{itemize}[leftmargin=1.5cm, noitemsep]
		\item obnova katastrálního operátu
		\item odstranění zjednodušené evidence
		\item nová digitální katastrální mapa
		\item nové podrobné polohové bodové pole
		\item zvýšená retence krajiny
		\item snížení eroze
		\item zvýšená ekologická stabilita
		\item ochrana povrchových a~podzemních vod
	\end{itemize}

\newpage

\section{Důvody a~cíle pozemkové úprav}
\label{duvody_cile_pu}

Důvodů k~zahájení pozemkových úprav bývá obvykle několik, přičemž
jeden či~více mají větší prioritu a~ostatní jsou spíše doplňující.

Nejčastější důvody pro~pozemkové úpravy:
\vspace{-\topsep}
	\begin{itemize}[leftmargin=1.5cm, noitemsep]
		\item území s~nedokončeným přídělovým nebo scelovacím řízením
		\item území s~množstvím jednoduchých pozemkových úprav
		\item investiční záměr velkého rozsahu
		\item žádost vlastníků nadpoloviční výměry
		\item vyjasnění a~uspořádání vlastnických vztahů
		\item nevhodné tvary pozemků
		\item zpřístupnění pozemků a~krajiny
		\item nízká ekologická stabilita
		\item protipovodňová ochrana
		\item obnova katastrálního operátu
		\item návaznost na~sousední katastrální území
	\end{itemize}

        Cíle pozemkových úprav úzce souvisí s~důvody jejich
        zahájení. Snahou je soustře\-dit se na~hlavní cíle a~zároveň
        neopomenout cíle vedlejší.

Hlavní cíle většiny pozemkových úprav:
\vspace{-\topsep}
	\begin{itemize}[leftmargin=1.5cm, noitemsep]
		\item vyjasnění a~uspořádání vlastnických práv
		\item zlepšení podmínek pro~racionální zemědělské hospodaření
		\item scelení roztříštěných pozemků jednoho vlastníka do~menšího počtu větších pozemků
		\item zlepšení tvaru pozemků pro~hospodaření
		\item zajištění přístupu na~pozemky
		\item zvýšení ekologické stability území
		\item zvýšení retence krajiny
		\item protipovodňová ochrana
		\item ochrana a~zúrodnění půdního fondu
	\end{itemize}

\section{Formy pozemkových úprav}
\label{formy_pu}

\subsection{Jednoduché pozemkové úpravy}
\label{jednoduche_pu}

Jak název napovídá, jednoduché pozemkové úpravy (\zk{JPU}) se týkají
menších oblastí, obyčejně části katastrálního území.

Varianta \zk{JPU} bez~přechodu vlastnických práv se používala
například po~ro\-ce~1990, kdy bylo potřeba narychlo umožnit
hospodaření jednotlivým zemědělským subjektům, ovšem od~roku 2002 se
již tyto \zk{JPU} neprovádějí.

V současné době se zahajují již jen \zk{JPU} se~zápisem vlastnických
práv do~kata\-stru nemovitostí. Tato varianta \zk{PU} se používá
například v~pohraničních oblastech, kde jsou v~důsledku nedokončených
přídělových řízení z~poválečného období nedořešené právnické vztahy,
v~místech, kde vlastníci ve~velké většině souhlasí s~obnovou pozemků
dle původní pozemkové evidence, nebo v~oblastech, kde je nutné vyřešit
specifický problém jako velké ohrožení pozemků větrnou erozí,
či~povodněmi.

\subsection{Komplexní pozemkové úpravy}
\label{komplexní_pu}

Komplexní pozemkové úpravy (\zk{KoPU}) zpravidla řeší nezastavěné
území celého kata\-strálního území. Cílem \zk{KoPU} není pouze jeden
konkrétní problém, jak~tomu může být u~\zk{JPU}, ale snaží se
řešit uspořádání pozemků v~širším kontextu.

\section{Obvod a předmět pozemkových úprav}
\label{obvod_a_predmet_pu}

\subsection{Obvod pozemkových úprav}
\label{obvod_pu}

Obvodem pozemkových úprav se rozumí území dotčené pozemkovými
úpra\-vami, které může být tvořeno jedním nebo více celky v~jednom
katastrálním území. V~pří\-padě potřeby lze do~\zk{ObPU} zahrnout
i~navazující části sousedních katastrálních území. Hranice obvodu
pozemkové úpravy bývá obvykle rozdělena na~vnitřní a~vnější. Vnitřní
hranice obvodu je nejčastěji určena hranicí mezi zastavěnou částí obce
(intravilánem) a~nezastavěným územím (extravilánem). Vnější hranice
zpravidla pro\-chází po~hranici katastrálního území, po~hranici lesa,
liniového objektu či~průmy\-slového areálu, může zasahovat
i~do~sousedních katastrálních území. Při~volbě obvodu pozemkové úpravy
by měly být zohledněny širší územní vztahy, neboť síť cest,
ani~oblasti ohrožené erozí či~povodněmi se neřídí podle hranic
katastrálních území. Z~důvodu komplikovaného oceňování lesní pozemky
zpravidla nebývají předmětem pozemkových úprav, obvod většinou končí
na~jejich okraji.

\subsection{Předmět pozemkových úpravy}
\label{predmet_pu}

Všechny pozemky v~obvodu pozemkových úprav bez~ohledu na~dosavadní
způsob využívání a~stávající vlastnické vztahy jsou předmětem
\zk{PU}. Převážně se jedná o~zemědělské pozemky, ale~i~další pozemky
v~extravilánu mohou být zahrnuty.

Pozemky v \zk{ObPU} se dělí na~tyto kategorie:
\vspace{-\topsep}
	\begin{itemize}[leftmargin=1.5cm, noitemsep]
        \item \underline{pozemky v~\zk{ObPU} řešené dle § zákona
          \citep{pu_zakon}}~– pozemky, u~kterých ve většině případů
          dochází ke~změnám v~jejich poloze. Mohou být děleny,
          scelovány a~musí být zajištěna jejich přístupnost.
        \item \underline{pozemky v~\zk{ObPU} neřešené dle § zákona
          \citep{pu_zakon}}~– pozemky v~obvodu pozem\-kových úprav,
          u~kterých se pouze obnovují geodetické informace.
          U~těchto pozemků se zjistí průběh jejich hranic, označí
          se lomové body a~vypočítá se nová výměra ze~souřadnic
          v~S-JTSK. Do~\zk{PU} jsou zahrnuty proto, aby nová
          katastrální mapa neobsahovala vynechané části. Tyto
          pozemky se neoceňují.
        \item \underline{pozemky mimo \zk{ObPU}}~– pozemky, které
          nejsou předmětem řízení o~pozem\-kových úpravách. Nesměňují
          se, nezpřístupňují, nezaměřují a~ani neoceňují. Nerozhoduje
          o~nich pozemkový úřad.
	\end{itemize}

\section{Fáze pozemkových úprav}
\label{etapy_pu}

Proces pozemkových úprav lze pro lepší přehlednost rozdělit
do několika fází. Hra\-nice těchto fází nejsou striktně určené, dochází
k~jejich částečnému překrývání.

\subsection{Programová fáze}
\label{programova_faze}

Programová fáze je plně v~kompetenci pozemkového úřadu. Pozemkový úřad
shromažďuje a~vyhodnocuje informace o~katastrálních územích, zjišťuje
zájem vlastníků, obcí a~nájemců o~provedení \zk{PU}. Na~základě
výsledného pořadníku katastrálních území a~finančních možností potom
pozemkový úřad zahajuje pozemkové úpravy.

\subsection{Přípravná fáze}
\label{pripravna_faze}

Zahájení řízení o~pozemkových úpravách je oznámeno veřejnou vyhláškou,
kterou pozemkový úřad po~dobu patnácti dní vyvěsí na~úřední desku svou
a~obcí, kterých se pozemkové úpravy budou týkat. Vlastníci jsou
upozorněni na~nutnost trvale stabilizovat hranice pozemků. Pozemkový
úřad s~jednoročním předstihem kontaktuje katastrální úřad, aby mohl
zkontrolovat \zk{SPI}, \zk{SGI} a~opravit případné nesrovnalosti. Oba
úřady, pozemkový a~katastrální, se dohodnou na~rozsahu pozemkové
úpravy a~předběžně určí obvod. V~případě potřeby se pozemkový úřad
spojí s~Výzkumným ústavem meliorací a~ochrany půdy (\zk{VUMOP})
a~zařídí aktualizaci \zk{BPEJ}. Dále také pozemkový úřad písemně
informuje všechny dotčené orgány státní správy.

Ve~veřejném výběrovém řízení je vybrán zpracovatel, který začne
shromažďovat podklady, zjišťovat stav území z~hlediska zemědělství,
ochrany půdy, vody, vlastnických a~nájemních vztahů.

Po~zahájení \zk{PU} je svoláno úvodní jednání, na~které jsou pozváni
všichni účastníci. Vlastníci jsou povinni prokázat vlastnická a~další
věcná práva k~pozemkům. Pozemkový úřad sdělí účastníkům důvody
k~zahájení pozemkových úprav a~seznámí je s~účelem a~předpokládaným
obvodem. Zpracovatel představí plánovaný harmonogram prací a~vysvětlí
potřebu spolupráce s~vlastníky. Nutným úkolem úvodního jednání je také
zvolit sbor zástupců. Ten musí být lichý, počet členů se pohybuje
v~rozmezí od~pěti do~patnácti členů. Automatickými členy se stávají
zástupce pozemkového úřadu a~zástupce obce. Sbor během \zk{PU}
zastupuje vlastníky, spolupracuje se zpracovatelem, vyjadřuje se
k~navrhovanému plánu společných zařízení a~ve~své činnosti pokračuje
i~během realizační etapy.

Při~zjišťování průběhu hranic se srovnává skutečnost se~stavem
zakresleným v~katastrální mapě a~s~výsledky předchozích zeměměřičských
prací. Lomové body vnitřní i~vnější hranice obvodu se v~terénu vyznačí
a~později i~zaměří. Zjišťování se účastní zástupce obce, zpracovatel,
zástupci pozemkového a~katastrálního úřadu a~hlavně samotní
vlastníci. Také se vytyčí a~označí vlastnické hranice pozemků, které
nejsou v~terénu trvale stabilizovány.

Velmi důležitým krokem přípravné fáze je sestavení nároků vlastníků,
na~jehož základě se posuzuje přiměřenost návrhu nového umístění
pozemků. V~potaz se berou zejména výměry pozemků, vzdálenost těžiště
pozemků od~zvoleného referenčního bodu a~ocenění podle
\zk{BPEJ}. Touto problematikou se podrobněji zabývá samostatná sekce
(viz \ref{naroky}).

\subsection{Projekční fáze}
\label{projekcni_faze}

Po~přípravné fázi přichází na~řadu fáze projekční. Spočívá nejprve
v~návrhu plánu společných zařízení, který byl dříve nazýván jako
generel nebo územní či~polyfunkční kostra.

Plán společných zařízení obsahuje čtyři základní části:
\vspace{-\topsep}
	\begin{itemize}[leftmargin=1.5cm, noitemsep]
		\item síť polních cest
		\item síť protierozních opatření
		\item síť vodohospodářských opatření
		\item síť prvků systémové ekologické stability
	\end{itemize}

Po~schválení plánu společných zařízení sborem zástupců
a~zastupitelstvem obce se přikračuje k~samotnému vytvoření návrhu
nového uspořádání vlastnických pozemků. Při~něm je nutné dodržet
kritéria přiměřenosti výměr, cen i~dopravních vzdáleností pozemků
jednotlivých vlastníků. V~průběhu pozemkových úprav, které mohou trvat
i~několik let, se vyhlašují tzv. kontrolní dny, kdy se schází sbor
zástupců se~zpraco\-vatelem a~projednává se např. návrh plánu
společných zařízení s dotčenými orgány státní správy. Na jednáních
se vyhotovují předběžně návrhy nového uspořádání pozemků a~projednávají se
s~účastníky.

Když je návrh zpracovaný, vystaví se na~úřední desce obce
a~pozemkového úřadu na~dobu třiceti dnů, během kterých mají vlastníci
příležitost vznést své připomínky. Pokud s návrhem souhlasí vlastníci
alespoň 60 \% výměry pozemků řešených podle § 2 zákona \citep{pu_zakon},
je návrh schválen. Po~uplynutí zmíněných třiceti dnů je svoláno
závěrečné jednání, na~kterém se hodnotí výsledky pozemkových úprav
a~pozemkový úřad seznámí účastníky s návrhem. Poté pozemkový úřad vydá
první rozhodnutí o~schválení návrhu pozemkové úpravy, informuje o~tom
veřejnou vyhláškou a~rozešle všem účastníkům část dokumentace, která
se jich týká. Do patnácti dnů od~prvního rozhodnutí se vlastníci mohou
odvolat, jakmile tato lhůta uběhne, nabývá první rozhodnutí pozemkového
úřadu právní moci a~přistupuje se k~vydání druhého rozhodnutí pozemkového
úřadu o~výměně nebo~přechodu vlastnických práv a~zřízení nebo~zrušení věcného
břemene. Pozemkový úřad druhé rozhodnutí oznámí veřejnou vyhláškou,
doručí jej katastrálnímu úřadu, vlastníkům a~dotčeným osobám. Proti
druhému rozhodnutí se již není možné odvolat. Kata\-strální úřad
obdrží dokumentaci o~novém geometrickém uspořádání pozemků a~jejich
vlastnických práv.

\subsection{Realizační fáze}
\label{realizacni_faze}

Během realizační fáze se uskutečňuje schválený návrh
\zk{PU}. Realizují se společná zařízení, vytyčuje se nové uspořádání
pozemků a~lomové body hranic se označují trvalým způsobem.

\subsection{Kontrolní fáze}
\label{kontrolni_faze}

Pozemkový úřad vyhodnocuje, zda bylo dosaženo vytyčených
cílů. Kontroluje správ\-nost návrhu společných zařízení a~jeho
funkčnost, přijímá zpětnou vazbu od~vlastníků, nájemníků, dotčených
osob a~orgánů státní správy. Využívá těchto poznatků a~zkušeností
při~dalších pozemkových úpravách.

\section{Sestavení vstupních soupisů nároků vlastníků}
\label{naroky}

Všichni vlastníci vstupují do~pozemkové úpravy se~svými pozemky, které
mají určitou výměru, vzdálenost a~cenu. V~průběhu pozemkové úpravy
budou jejich pozemky scelovány do~větších výměr, budou narovnávány
jejich hranice a~budou přesouvány na~nová místa. Na~konci pozemkové
úpravy potom vlastníci dostanou nové pozemky, jejichž výměra,
vzdálenost a~cena bude odpovídat pozemkům původním. Výsledkem je tedy
to, že každý vlastník bude mít menší počet pozemků s~větší průměrnou
výměrou, všechny budou mít vhodný tvar pro~zemědělskou činnost, budou
přístupné a~budou chráněné proti erozi.

Soupisy vstupních nároků se vyhotovují pro všechny vlastníky pozemků,
které alespoň částečně zasahují do~\zk{ObPU}, a~jsou závazným
podkladem pro~návrh nového uspořádání pozemků.

Pro~sestavení soupisu nároků se používají tyto podklady:
\vspace{-\topsep}
	\begin{itemize}[leftmargin=1.5cm, noitemsep]
		\item katastrální operát~– \zk{SPI} a~\zk{SGI}
		\item mapy dřívější pozemkové evidence
		\item výsledky podrobného zaměření hranice \zk{ObPU}
		\item údaje o~\zk{BPEJ}
		\item cenový předpis pro~oceňování pozemků
	\end{itemize}

Během zpracování \zk{PU} se odstraňují chyby v~katastrálním operátu,
aby se v~obvodu pozemkové úpravy nenacházel pozemek bez~vlastníka,
a~také se kontrolují nabývací tituly, na~jejichž základě bylo
vlastnictví zapsáno do~\zk{KN}.

Proces vyhotovení vstupních soupisů nároků je popsán v~následujících
částech.

\subsection{Digitalizace mapových podkladů}
\label{digitalizace}

Pokud nejsou mapové podklady ve~vektorové formě\footnote{Na 97 \% území
České republiky je katastrální mapa v digitální podobě, viz
\url{https://goo.gl/Qw3FZu}.}, musí se převést do~digitální podoby.
Pro digitalizaci je nejprve nutné naskenovanou katastrální mapu pomocí
identických bodů transformovat do~\zk{S-JTSK}. Kvůli identifikaci parcel
vedených ve~zjednodušené evidenci se do~\zk{S-JTSK} natransformují i~mapy
předchozích pozemko\-vých evidencí. Mapy se poté digitalizací převedou
do~vektorového formátu. Kód kvality podrobných bodů určených
digitalizací je stanoven na~základě měřítka katastrální mapy, viz
tab.~\ref{tab:kody_kvality_digit}.

\begin{table}[H]
    \begin{tabular}{|l|l|l|} \hline kód kvality & měřítko katastrální
mapy & \begin{tabular}{@{}l@{}} základní střední \\ souřadnicová chyba
[m] \end{tabular} \\ \hline \hline \texttt{6} & 1:1000, 1:1250 & 0.21
\\ \hline \texttt{7} & 1:2000, 1:2500 & 0.51 \\ \hline \texttt{8} &
1:2880 a jiné & 1.00 \\ \hline
    \end{tabular} \centering
    \caption[Kódy kvality podrobných bodů určených digitalizací]{Kódy
kvality podrobných bodů určených digitalizací
(zdroj:~\citep{vyhlaska_357})}
    \label{tab:kody_kvality_digit}
\end{table}

Základní střední souřadnicová chyba je dána vztahem:

\begin{equation} m_{xy} = \sqrt{\frac{(m_{x}^{2}+{m_{y}^{2}})}{2}}
\end{equation}

kde
\begin{tabbing} \hspace{2em} \= \hspace{5em} \= \kill \> $m_{x}$ \> je
střední chyba určení souřadnice $x$ \\ \> $m_{y}$ \> je střední chyba
určení souřadnice $y$
\end{tabbing}

\subsection{Kontrola souladu SPI a SGI}
\label{soulad_spi_sgi}

Pro všechny parcely zahrnuté do~obvodu pozemkové úpravy se provádí
kontrola souladu souboru popisných a~geodetických
informací. Kontrolují se druhy pozemků, parcelní čísla a~nesoulady
v~geometrickém a~polohovém určení pozemku.

Dále se porovnávají výměry parcel evidovaných v~\zk{SPI} s~výměrou
vypočtenou z~\zk{SGI}. Mezní odchylka výměr se vypočte pomocí vzorce
z~tab.~\ref{tab:odchylky_vymer}, kde $P$ je větší z~porovnávaných
výměr v~metrech čtverečních.

\begin{table}[H]
    \begin{tabular}{|l|l|} \hline
         \begin{tabular}{@{}l@{}} kód kvality nejméně přesně \\
určeného bodu na~hranici parcely \end{tabular} & mezní odchylka
[m\textsuperscript{2}] \\ \hline \hline \texttt{3} & $2$ \\ \hline
\texttt{4} & $0.4*\sqrt{P}+4$ \\ \hline \texttt{5} & $1.2*\sqrt{P}+12$
\\ \hline \texttt{6} & $0.3*\sqrt{P}+3$ \\ \hline \texttt{7} &
$0.8*\sqrt{P}+8$ \\ \hline \texttt{8} & $2.0*\sqrt{P}+20$ \\ \hline
    \end{tabular} \centering
    \caption[Mezní odchylky výměr]{Mezní odchylky výměr
(zdroj:~\citep{vyhlaska_357})}
    \label{tab:odchylky_vymer}
\end{table}

\subsection{Vlastnická mapa}
\label{vlastnicka_mapa}

Ze~zpracovaných vektorových dat se následně vytvoří tzv.~vlastnická
mapa, ve~které na~rozdíl od~platné katastrální mapy má každá parcela
svého vlastníka.

Vlastnická mapa se potom vytiskne v~barevném provedení, kde jsou
parcely pro~každý list vlastnictví znázorněny jinou barvou
nebo~šrafou. Součástí vytištěné vlastnické mapy je i~legenda.

Během procesu pozemkové úpravy se vlastnická mapa vyhotovuje
dvakrát. Nej\-prve na~začátku \zk{PU}, kdy slouží k~projednávání
soupisu nároků. Této variantě vlastnické mapy se také říká mapa
nároků. Podruhé je vytvořena tzv.~vlastnická mapa návrhu, která obsahuje nový
stav navržených pozemků a~používá se k~seznámení vlastníků s~umístěním
a~tvarem nových pozemků. Ukázka vlastnické mapy se nachází
na~obr.~\ref{fig:vlastnicka_mapa}.

	\begin{figure}[H] \centering
		\includegraphics[width=.8\textwidth]{./pictures/vlastnicka_mapa.pdf}
		\caption[Vlastnická mapa]{Vlastnická mapa
(zdroj:~\citep{skvorec})}
		\label{fig:vlastnicka_mapa}
 	\end{figure}

\subsection{Výpočet opravného koeficientu dle zaměření skutečného stavu}
\label{vypocet_ok}

Výpočtu opravného koeficientu dle zaměření skutečného stavu výměr předchází
zjišťování průběhu hranice obvodu pozemkové úpravy za~účasti vlastníků.
Lomové body jsou v~terénu označeny, případně vytyčeny a~stabilizovány.
Poté jsou tyto body s~požadovanou přesností (střední souřadnicová chyba
$m_{xy}=0.14~m$, kód charakteristiky kvality bodu $3$) zaměřeny
do~systému \zk{S-JTSK} a~provede se výpočet výměry obvodu pozemkových
úprav ze~souřadnic. Součet výměr všech parcel v~obvodu pozemkové
úpravy dává dohromady výměru obvodu pozemkových úprav podle katastru
nemovitostí.

Opravný koeficient dle zaměření skutečného stavu (\zk{OK}) se vypočte pomocí
následujícího vztahu:

\begin{equation} OK =
\frac{P\textsubscript{S-JTSK}}{P\textsubscript{KN}}
\end{equation}

kde
\begin{tabbing} \hspace{2em} \= \hspace{5em} \= \kill \> $P_{S-JTSK}$
\> je výměra pozemků řešených podle § 2 \\ \> \> vypočtená ze~souřadnic \\
\> $P_{KN}$ \> je výměra pozemků řešených podle § 2 \\ \> \> podle
katastru nemovitostí
\end{tabbing}

Výsledná hodnota opravného koeficientu je číslo, které by se mělo jen
málo lišit od~$1$. Pokud je \zk{OK} menší než~$1$, pak se nároky
zmenšují, v~opačném případě se nároky zvětšují. Opravný koeficient
slouží k~úpravě nároků podle skutečnosti.

\subsection{Ocenění pozemků}
\label{oceneni}

Všechny řešené pozemky se oceňují. K~oceňování se používají data
s~hranicemi \zk{BPEJ} od~Výzkumného ústavu meliorací a~půdy. Případnou
změnu hranic \zk{BPEJ} podle~skutečného průběhu v~terénu musí
odsouhlasit~\zk{VUMOP}. Ocenění se provede jako průnik vlastnické mapy
a~hranic \zk{BPEJ}.

	\begin{figure}[H] \centering
		\includegraphics[width=.5\textwidth]{./pictures/vumop.png}
		\caption[Výzkumný ústav meliorací a~půdy~–
logo]{Výzkumný ústav meliorací a~půdy~– logo (zdroj:~\citep{vumop})}
		\label{fig:vumop}
 	\end{figure}

Nárokový list obsahuje nejen celkové ceny pozemků, ale také cenu
za~metr čtvereční dle~kódu \zk{BPEJ} a~ceny částí pozemků
dle~jednotlivých bonit.

Pozemky chmelnic, vinic, sadů, zahrad a~pozemků s~lesním porostem, které
spadají do skupiny řešených dle § 2, je povinné ocenit. V~nárokovém listu
se uvede cena pozemku odděleně od~ceny porostu.

\subsection{Výpočet vzdálenosti pozemků}
\label{vypocet_vzdalnosti_pozemku}

Pro~všechny řešené pozemky je nutné vypočítat vzdálenost
od~referenčního bodu. Vzdálenost pozemku se určí jako délka přímé
spojnice těžiště pozemku a~referenčního bodu.

\subsection{Vlastní sestavení vstupních soupisů nároků vlastníků}
\label{vlastni_naroky}

Pro každého vlastníka (číslo listu vlastnictví), jehož pozemky jsou celé
nebo z~části zahrnuty v \zk{ObPU}, je sestaven soupis nároků neboli nárokový
list, jehož hlavním výsledkem jsou tři hodnoty:
\vspace{-\topsep}
	\begin{itemize}[leftmargin=1.5cm, noitemsep]
		\item celková výměra pozemků~– $P_{U}$
		\item celková cena pozemků~– $C_{U}$
		\item průměrná vzdálenost pozemků~– $D_{U}$
	\end{itemize}

\subsubsection{Prosté nároky}
\label{proste_naroky}

Nejprve je zapotřebí pro~každé \zk{LV} určit prosté nároky ve~výměře
a~ceně. Vypočítají se jednoduchým součtem:

\begin{equation} P_{LV} = \sum\nolimits P_{p}
\end{equation}

kde
\begin{tabbing} \hspace{2em} \= \hspace{5em} \= \kill \> $P_{LV}$ \>
je prostý nárok ve výměře \\ \> $P_{p}$ \> jsou výměry jednotlivých
pozemků dle \zk{SPI}
\end{tabbing}

\begin{equation} C_{LV} = \sum\nolimits C_{p}
\end{equation}

kde
\begin{tabbing} \hspace{2em} \= \hspace{5em} \= \kill \> $C_{LV}$ \>
je prostý nárok v ceně \\ \> $C_{p}$ \> jsou ceny jednotlivých pozemků
\end{tabbing}

\subsubsection{Průměrná vzdálenost pozemků}
\label{prumerna_vzdalenost_pozemku}

Průměrná vzdálenost pozemků pro~každé \zk{LV} se vypočte jako vážený
průměr jednotlivých vzdáleností:

\begin{equation} D_{LV} = \frac{\sum\nolimits
d_{p}*P_{p}}{\sum\nolimits P_{p}}
\end{equation}

kde
\begin{tabbing} \hspace{2em} \= \hspace{5em} \= \kill \> $D_{LV}$ \>
je průměrná vzdálenost pozemků \\ \> $d_{p}$ \> jsou vzdálenosti
jednotlivých pozemků \\ \> $P_{p}$ \> jsou výměry jednotlivých pozemků
dle \zk{SPI}
\end{tabbing}

\subsubsection{Upravené nároky}
\label{upravene_naroky}

Prosté nároky ve~výměře a~ceně se upravují opravným koeficientem, aby
byly v~soula\-du se~skutečností. Průměrná vzdálenost se pomocí
opravného koeficientu neupravuje.

\begin{equation} P_{U} = P_{LV}*OK
\end{equation}

kde
\begin{tabbing} \hspace{2em} \= \hspace{5em} \= \kill \> $P_{U}$ \> je
celková výměra pozemků \\ \> \>(výsledný upravený nárok vlastníka ve výměře)
\\ \> $P_{LV}$ \> je prostý nárok ve výměře \\ \> $OK$ \> je opravný koeficient
dle zaměření skutečného stavu
\end{tabbing}

\begin{equation} C_{U} = C_{LV}*OK
\end{equation}

kde
\begin{tabbing} \hspace{2em} \= \hspace{5em} \= \kill \> $C_{U}$ \> je
celková cena pozemků \\ \> \> (výsledný upravený nárok vlastníka v ceně) \\
\> $C_{LV}$ \> je prostý nárok v~ceně \\ \> $OK$ \> je opravný koeficient
dle zaměření skutečného stavu
\end{tabbing}

\newpage

\begin{equation} D_{U} = D_{LV}
\end{equation}

kde
\begin{tabbing} \hspace{2em} \= \hspace{5em} \= \kill \> $D_{U}$ \> je
průměrná vzdálenost pozemků \\ \> \> (výsledný nárok vlastníka ve vzdálenosti)
\\ \> $D_{LV}$ \> je průměrná vzdálenost pozemků
\end{tabbing}

Správnost sestavení soupisu nároků lze ověřit výpočtem sumy upravených
nároků ve~výměře, která by se měla rovnat výměře obvodu \zk{PU}
vypočtené ze~souřadnic:

\begin{equation} \sum\nolimits P_{U} = OK*\sum\nolimits P_{LV} = OK *
\sum\nolimits \sum\nolimits P_{p} = OK*P_{KN} =
\frac{P_{S-JTSK}}{P_{KN}}*P_{KN} = P_{S-JTSK}
\end{equation}

kde
\begin{tabbing} \hspace{2em} \= \hspace{5em} \= \kill \> $P_{U}$ \> je
celková výměra pozemků \\ \> \>(výsledný upravený nárok vlastníka ve výměře)
\\ \> $OK$ \> je opravný koeficient dle zaměření skutečného stavu \\
\> $P_{LV}$ \> je prostý nárok ve~výměře \\ \> $P_{p}$ \> jsou výměry
jednotlivých pozemků dle \zk{SPI} \\ \> $P_{S-JTSK}$ \> je výměra pozemků
řešených podle § 2 \\ \> \> vypočtená ze~souřadnic \\ \> $P_{KN}$ \>
je výměra pozemků řešených podle § 2 \\ \> \> podle katastru nemovitostí
\end{tabbing}

\subsubsection{Soupis nároků vlastníků}
\label{soupis_naroku_vlastniku}

V~soupisu nároků vlastníků se pro~přehlednost a~kontrolu uvádí
i~pozemky mimo \zk{ObPU}. Neřešené pozemky se neoceňují, u~takových
pozemků je v~nárokových listech zapsána pouze výměra podle \zk{KN}
a~ze~zaměření.

V~soupisu nároků se uvádí: \vspace{-\topsep}
	\begin{itemize}[leftmargin=1.5cm, noitemsep]
		\item číslo listu vlastnictví
		\item jméno a~adresa vlastníka
		\item pozemky podle parcelních čísel v~obvodu i~mimo
obvod \zk{PU}
		\item výměry pozemků
		\item druhy pozemků
		\item výměry částí pozemků dle \zk{BPEJ}
		\item celková výměra pozemků
		\item ceny pozemků
		\item ceny částí pozemků dle \zk{BPEJ}
		\item celková cena pozemků
		\item vzdálenosti pozemků
		\item průměrná vzdálenost pozemků
		\item opravný koeficient dle zaměření skutečného stavu
		\item údaje o~omezení vlastnického práva
	\end{itemize}

Nárokové listy jsou po~dobu patnácti dnů k~dispozici k~nahlédnutí
na~příslušném obecním úřadě a~jsou také rozeslány vlastníkům. Ti jsou
vyzváni k~tomu, aby si zkontrolovali své nárokové listy a~vyjádřili
svůj souhlas podpisem.
 	
	\begin{figure}[H] \centering
		\includegraphics[width=.9\textwidth]{./pictures/soupis_naroku.png}
		\caption[Část soupisu nároků - vzor]{Část soupisu nároků - vzor
(zdroj:~\citep{vyhlaska_13})}
		\label{fig:soupis_naroku}
 	\end{figure}

Vstupní soupisy nároků vlastníků patří mezi podklady pro~zpracování
návrhu pozemkové úpravy. Celková výměra, cena a~průměrná vzdálenost
nově navržených pozemků musí odpovídat pozemkům původním. Maximální
rozdíly jsou dané záko\-nem o~pozemkových úpravách \citep{pu_zakon}.

\section{Programy pro zpracování pozemkových úprav}
\label{programy_pu}

Nezbytným nástrojem pro~zpracování pozemkových úprav je vhodný
software, který podporuje práci s~navzájem propojenými geografickými
daty a~databází. Tuto podmínku splňují všechny programy typu \zk{GIS}
a~některé programy typu \zk{CAD}. Na~trhu je několik programů, které
se specializují čistě na~pozemkové úpravy, ale častěji se jedná
o~extenze programů s~širším využitím. Tyto programy umožňují načítání
vstupních dat ze~souboru \zk{VFK} a~podporují práci s~vektorovými
i~rastrovými daty.

Všechny programy uvedené v~této sekci jsou distribuovány pouze
pro~platformu Windows a~patří mezi~proprietární software.

\subsection{POZEM}
\label{pozem}

Systém POZEM je nadstavba programu Microstation nebo~jeho derivací
určená pro~projektování komplexních pozemkových úprav. Nabízí
zpracování všech etap \zk{KoPU} \citep{pozem}~\citep{pu_skripta}.

Funkčnost programu je možné rozdělit do~pěti skupin: \vspace{-\topsep}
	\begin{enumerate}[leftmargin=1.5cm, noitemsep]
		\item \underline{Import dat}~– import \zk{VFK}
a~dalších podkladů.
		\item \underline{Příprava dat}~– výkresy je možné
pomocí sady funkcí topologicky vyčistit a~připojit k~nim i~negrafické
informace.
		\item \underline{Zpracování nároků}~– na~základě
mapových podkladů lze vypočítat výměru, cenu a~vzdálenost parcel.
		\item \underline{Zpracování návrhu}~– umožňuje návrh
parcel s~okamžitým výpočtem výměry, ceny, vzdálenosti a~jeho porovnání
s~nárokovými hodnotami.
		\item \underline{Výstupy}~– export dat do~výměnného
formátu pozemkových úprav (\zk{VFP}). Z~výsledného návrhu lze také
zpracovat digitální katastrální mapu a~vyexportovat ji ve~formátu
\zk{VFK}.
	\end{enumerate}

Výhody programu POZEM: \vspace{-\topsep}
	\begin{itemize}[leftmargin=1.5cm, noitemsep]
		\item podpora zpracování všech etap \zk{KoPU}
		\item automatizace většiny procesů \zk{KoPU}
		\item vytváření sestav a~dokumentů podle platné
legislativy
		\item export do~\zk{VFK} i~\zk{VFP}
		\item automatické aktualizace
	\end{itemize}

	\begin{figure}[H] \centering
		\includegraphics[width=.8\textwidth]{./pictures/pozem.png}
		\caption[POZEM~– zpracování nároku]{POZEM~– zpracování
nároku (zdroj:~\cite{pozem})}
		\label{fig:pozem_obrazek}
 	\end{figure}

\subsection{PROLAND}
\label{proland}

Dalším softwarových produktem pro~zpracování pozemkových úprav
a~navazujících geodetických prací je program PROLAND. Jedná se
o~rozšíření grafického systému KOKEŠ, které obsahuje sadu funkcí
pro~automatické zpracování pozemkových úprav a~pro evidenci účastníků
řízení~\citep{proland}~\citep{pu_skripta}.

Program PROLAND plně podporuje import a~export dat ve~výměnném formátu
katastru nemovitostí.

Postup práce v~programu PROLAND je podobný jako v~případě programu
POZEM: \vspace{-\topsep}
	\begin{enumerate}[leftmargin=1.5cm, noitemsep]
		\item \underline{Import dat}~– načtení dat \zk{VFK}
a~dalších podkladů.
		\item \underline{Příprava dat}~– možnost vyhotovení
výkresů vektorizací rastrových souborů a~následná kontrola topologie.
		\item \underline{Zpracování nároků}~– tvorba vstupních
nároků včetně automatického přiřaze\-ní kódu BPEJ, přiřazení druhu
a~způsobu využití pozemků odpovídající skutečnému stavu, ocenění
parcel.
		\item \underline{Zpracování návrhu}~– při~tvorbě
nových pozemků se využívá především postupné dělení bloků půdy, které
jsou vymezeny naprojektovanou kostrou území. V~konkrétních případech
lze využít obecných funkcí systému KOKEŠ. Program generuje soupisy
nově navržených pozemků, přehled navržených parcel, souhrnnou bilanci
nároku a~návrhu.
		\item \underline{Výstupy} - export výstupů do~\zk{VFK}
nebo~\zk{VFP}.
	\end{enumerate}

Výhody programu PROLAND: \vspace{-\topsep}
	\begin{itemize}[leftmargin=1.5cm, noitemsep]
		\item možnost zpracování všech etap \zk{KoPU}
		\item automatické zpracování mnoha procesů \zk{KoPU}
		\item export dat do~\zk{VFK} i~\zk{VFP}
		\item pravidelné aktualizace
	\end{itemize}

	\begin{figure}[H] \centering
		\includegraphics[width=.8\textwidth]{./pictures/proland.png}
		\caption[PROLAND~– komunikační výstupy]{PROLAND~–
komunikační výstupy (zdroj:~\citep{proland_obrazek})}
		\label{fig:proland_obrazek}
 	\end{figure}

\subsection{TOPOL xT}
\label{topol_xt}

Na~rozdíl od~obou předchozích programů typu \zk{CAD}, TOPOL~xT patří
mezi geografické informační systémy. Nejširší oblastí využití programu
TOPOL~xT je jednoznačně lesnictví, ale~své uplatnění najde
i~při~zpracování pozemkových úprav. Poskytuje funkce pro~zpracování
nároků, návrh nových parcel a~tvorbu všech nutných výstupů. Mezi
výhody patří možnost tvorby vlastních uživatelských aplikací
\citep{topol}~\citep{pu_skripta}.

	\begin{figure}[H] \centering
		\includegraphics[width=.8\textwidth]{./pictures/topol.png}
		\caption[TOPOL xT~– náhled tisku]{TOPOL xT~– náhled
tisku (zdroj:~\citep{topol})}
		\label{fig:topol_obrazek}
 	\end{figure}
