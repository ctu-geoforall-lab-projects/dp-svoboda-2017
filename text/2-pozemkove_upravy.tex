\chapter{Pozemkové úpravy}
\label{2-pu}

Tato kapitola se věnuje pozemkovým úpravám. Popisuje význam, důvody, cíle, formy a~celý proces pozemkových úprav s~důrazem na~části, kterých se týka zásuvný modul vytvořený v~rámci této práce.

V~této části bylo čerpáno z~\citep{pu_zakon} \citep{pu_cr} \citep{metodicky_navrh} a~\citep{pu_skripta}.

\section{Pojem pozemkových úprav}
\label{pojem_pu}

Pozemkové úpravy zahrnují mnoho na~sebe navazujícíh činností, jejichž společným cílem je zlepšení podmínek pro~zemědelské hospodaření, zpřístupnění pozemků, zmírnění nepříznivých účinků vodní a~větrné eroze, zlepšení životního prostředí, zvýšení ekologické stability krajiny a~zachování či~obnova krajinného rázu. Děje se tak pomocí prostorového a~funkčního uspořádávání pozemků, pozemky se dělí a~scelují. K pozemkům se vyhotovují vlastnická práva a~s~tím související věcná břemena. Výsledky pozemkových úprav slouží jako podklady pro obnovu katastrálního operátu.

Pozemkové úpravy jsou multidiscilinární obor, který využívá znalostí a poznatků z mnoha dalších oborů. Mezi ně patří zemědělství, krajinné a~územní plánování, geodézie, fotogrammetrie, vodohospodářství, ochrana životního prostředí, katastr nemovitostí a další. Důležitá je spolupráce všech odborníků, aby byla zajištěna plynulá návaznost prací.

\section{Význam pozemkových úprav}
\label{vyznam_pu}

Pozemkové úpravy mají význam jak pro~účastníky pozemkových úprav, tedy vlastníky, stavebníky a~obce, tak pro obyvatele a~návštěvníky venkova, orgány státní správy, podnikatelsé subjekty, správce inženýrských sítí a~zájmové organizace. Ve výsledku mají tedy pozemkové úpravy dopad na~životy jednotlivců, společnosti a celého státu.

Význam \zk{PÚ} pro vlastníky a~nájemce půdy:
	\begin{itemize}[noitemsep, leftmargin=1.5cm]
		\item přehledné a~jasné vlastnické vztahy
		\item vytyčené hranice pozemků v terénu
		\item zajištěný přístup na pozemky
		\item lepší tvar pozemků vhodných pro~racionální zemědělské hospodaření
		\item možnost uzavřít nájemní smlouvy na~přesné výměry a hranice pozemků
		\item lepší organizace půdní držby
		\item zvýšená tržní cena pozemků
	\end{itemize}

Význam \zk{PÚ} pro zemědělské subjekty:
	\begin{itemize}[noitemsep, leftmargin=1.5cm]
		\item lepší tvar pozemků vhodných pro~racionální zemědělské hospodaření
		\item zajištěný přístup na~pozemky
		\item možnost uzavření nájemních smluv na~přesné výměry a~hranice pozemků
		\item možnost žádat o dotace
	\end{itemize}

Význam \zk{PÚ} pro obce:
	\begin{itemize}[noitemsep, leftmargin=1.5cm]
		\item vyjasněné právnické vztahy v~území
		\item zpřístupnění a~zprůchodnění krajiny
		\item nalezení a~zapsání historického majetku obce
		\item podrobná dokumentace o území
		\item realizace společných zařízení za~státní peníze
		\item podklad pro zpracování územního plánu
		\item zvýšená ekologická stabilita území
		\item protipovodňová ochrana obce
		\item podpora pěší turistiky a~cykloturistiky
		\item zkvalitnění života na~venkově
	\end{itemize}

Význam \zk{PÚ} pro orgány státní správy:
	\begin{itemize}[noitemsep, leftmargin=1.5cm]
		\item obnova katastrálního operátu
		\item odstranění zjednodušené evidence
		\item nová digitální katastrální mapa
		\item nové podrobné polohové bodové pole
		\item zvýšená retence krajiny
		\item snížení eroze
		\item zvýšená ekologická stabilita
		\item ochrana povrchových a~podzemních vod
	\end{itemize}

\section{Důvody pro pozemkové úprav}
\label{duvody_pu}

Důvodů k zahájení pozemkových úprav býva obvykle několik, přičemž jeden či více mají větší prioritu a ostatní jsou spíše doplňující.

Zde jsou vyjmenovány nejčastější důvody pro pozemkové úpravy:
	\begin{itemize}[noitemsep, leftmargin=1.5cm]
		\item území s nedokončeným přídělovým nebo scelovacím řízením
		\item území s množstvím jednoduchýh pozemkových úprav
		\item investiční záměr velkého rozsahu
		\item žádost vlastníků nadpoloviční výměry
		\item vyjasnění a uspořádání vlastnických vztahů
		\item nevhodné tvary pozemků
		\item zpřístupnění pozemků a krajiny
		\item nízká ekologická stabilita
		\item protipovodňová ochrana
		\item obnova katastrálního operátu
		\item návaznost na sousední katastrální území
	\end{itemize}

\section{Cíle pozemkových úprav}
\label{cile_pu}

Cíle pozemkových úpravy úzce souvisí s~důvody jejích zahájení. Snahou je soustředit se na~hlavní cíle a~zároveň neopomenout cíle vedlejší.

Toto jsou hlavní cíle většiny pozemkových úprav:
	\begin{itemize}[noitemsep, leftmargin=1.5cm]
		\item vyjasnění a uspořádání vlastnických práv
		\item zlepšení podmínek pro~racionální zemědělské hospodaření
		\item scelení roztříštěných pozemků jednoho vlastníka do~menšího počtu větších pozemků
		\item zlepšení tvaru pozemků pro~hospodaření
		\item zajištění přístupu na~pozemky
		\item zvýšení ekologické stability území
		\item zvýšení retence krajiny
		\item protipovodňová ochrana
		\item ochrana a~zúrodnění půdního fondu
	\end{itemize}

\section{Formy pozemkových úprav}
\label{formy_pu}

\subsection{Jednoduché pozemkové úpravy}
\label{jednoduche_pu}

Jak název napovídá, jednoduché pozemkové úpravy se týkají menší oblasti, obyčejně části katastrálního území.

Varianta \zk{JPÚ} bez přechodu vlastnických práv se používala například po roce 1990, kdy bylo potřeba narychlo umožnit hospodaření jednotlivým zemědělským subjektům, ovšem od roku 2002 se již tyto \zk{JPÚ} neprovádějí.

V současné době se zahajují již jen \zk{JPÚ} se zápisem vlastnických práv do katastru nemovitostí. Tato varianta \zk{PÚ} se používá například v pohraničních oblastech, kde jsou v důsledku nedokončených přídělových řízení z~poválečného období nedořešené právnické vztahy, v~místech, kde vlastníci ve~velké většině souhlasí s~obnovou pozemků dle původní pozemkové evidence, nebo~v~oblastech, kde je nutné vyřešit specifický problém jako velké ohrožení pozemků půdní erozí, či~povodněmi.

\subsection{Komplexní pozemkové úpravy}
\label{komplexní_pu}

Komplexní pozemkové úpravy zpravidla řeší nezastavěné území - extravilán - celého katastrálního území. Cílem \zk{KPÚ} není pouze jeden konkrétní problém, jak tomu může být u~\zk{JPÚ}, ale snaží se uspořádat pozemky v~širším kontextu. 

\section{Obvod a předmět pozemkových úprav}
\label{obvod_a_predmet_pu}

\subsection{Obvod pozemkových úprav}
\label{obvod_pu}

Obvod pozemkových úprav je území dotčené pozemkovými úpravami, které je tvořeno jedním nebo více celky v jednom katastrálním území. V případě potřeby lze do \zk{ObPÚ} zahrnout i~navazující části sousedních katastrálních území. Hranice obvodu pozemkové úpravy býva obvykle rozdělena na vnitřní a vnější. Vnitřní hranice obvodu je nejčastěji určena hranicí mezi zastavěnou částí obce - intravilánem - a nezastavěným územím - extravilánem. Vnější hranice nejčastěji prochází po hranici katastrálního území, po hranici lesa, liniového objektu či průmyslového areálu, může zasahovat i do sousedních katastrálních území. Při volbě obvodu pozemkové úpravy by měly být zohledněny širší územní vztahy, neboť síť cest, ani oblasti ohrožené erozí či povodněmi se neřídí podle hranic katastrálních území. Z důvodu komplikovaného oceňování lesní pozemky zpravidla nebývají předmětem pozemkových úprav, obvod většinou končí na jejich okraji.

\subsection{Předmět pozemkových úpravy}
\label{predmet_pu}

Všechny pozemky v obvodu pozemkových úprav bez ohledu na dosavadní způsob využívání a stávající vlastnické vztahy jsou předmětem \zk{PÚ}. Převážně se jedná o zemědělské pozemky, ale i další pozemky v extravilánu mohou být zahrnuty. Pozemky v \zk{ObPÚ} se dělí na tyto skupiny:
	\begin{itemize}[leftmargin=1.5cm]
		\item \underline{pozemky v~\zk{ObPÚ} řešené} - pozemky, u~kterých ve většině případů dochází ke~změnám v jejich poloze. Mohou být děleny, scelovány a~musí být zajištěna jejich přístupnost.
		\item \underline{pozemky v~\zk{ObPÚ} neřešené} - pozemky v~obvodu pozemkových úprav, u~kterých se pouze obnovují geodetické informace. U~těchto pozemků se zjistí průběh jejich hranic, označí se lomové body a~vypočítá se nová výměra ze~souřadnic v~S-JTSK. Do~\zk{PÚ} jsou zahrnuty proto, aby nová katastrální mapa neobsahovala vynechané části. Tyto pozemky se neoceňují.
		\item \underline{pozemky mimo \zk{ObPÚ}} - pozemky, které nejsou předmětem řízení o~pozemkových úpravách. Nesměňují se, nezpřístupňují, nezaměřují a~ani neoceňují. Nerozhoduje o~nich pozemkový úřad.
	\end{itemize}

\section{Fáze pozemkových úprav}
\label{etapy_pu}

\subsection{Programová fáze}
\label{programova_faze}

Programová fáze je plně v kompetenci pozemkového úřadu. Cílem je vytvořit strategii v rámci mikroregionu, okresu, kraje a~státu. Tato činnost podléhá momentálním politickým představám a období vývoje společnosti. Vychází z aktuálních priorit, ale měla by také sledovat dlouhodobou kontinuitu v životě občanské společnosti. Důležitou roli zde hraje agrární politika státu.

Pozemkový úřad shromažduje a~vyhodnocuje informace o~katastrálních územích, zjištuje zájem vlastníků, obcí a~nájemců o~provedení \zk{PÚ}. Na základě výsledného pořadníku katastrálních území a finančních možností potom pozemkový úřad zahajuje pozemkové úpravy a~informuje o~tom další orgány státní správy, kterých se budou \zk{PÚ} týkat. Ve~veřejném výběrovém řízení je vybrán zpracovatel, se kterým pozemkový úřad podepíše obchodní smlouvu.

\subsection{Přípravná fáze}
\label{pripravna_faze}

Na rozdíl od~programové fáze, která může probíhat na~úrovni státu, krajů a~okresů, příprávná fáze se týká konkrétního vybraného katastrálního území. Zadavatel, pozemkový úřad, stanovuje cíle, rozsah a~zásady zpracování. Hledání cílů, úprava obvodu a jiné korekce během následující návrhové etapy jsou velmi nežádoucí, neefektivní a protahují již tak dlouhou dobu pozemkových úprav, proto je dobré dbát zvýšenou pozornost právě této etapě. Ovšem i přesto se v~tak složitém procesu, jakým pozemkové úpravy bezesporu jsou, mohou vyskytnout nové skutečnosti, které pozmění dílčí cíle a obchodní smlouvu mezi zadavatelem a zpracovatelem. Z praxe je zřejmé, že právě přípravnou fázi pozemkové úřady, zpracovatelé a katastrální úřady mnohdy podceňují.

Zásadní věcí, kterou je v přípravné fázi nutné vyřešit, je předběžné určení obvodu. Obecným doporučením je, že obvod pozemkové úpravy by měl být stanoven tak, aby řešil identifikované problémy území. Z~toho vyplývá, že se nelze striktně držet hranice katastrálního území. Do~obvodu je možné zahrnout i~navazující části sousedních katastrálních území, nebo naopak lze některé části, ve kterých není nutná změna, například lesní komplexy, vynechat. Důležité je také zohledit, aby nová digitální mapa byla co nejvíce souvislá a obsahovala co nejméně prázdných míst.