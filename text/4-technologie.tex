\chapter{Použité technologie}
\label{technologie}

\section{QGIS}
\label{qgis}

	\begin{figure}[H]
		\centering
		\includegraphics[width=.3\textwidth]{./pictures/qgis_logo.png}
      	\caption[logo QGIS]{logo QGIS (zdroj: \href{https://commons.wikimedia.org/wiki/File:QGis_Logo.png}{Wikimedia Commons})}
		\label{fig:qgis_logo}
 	\end{figure}

QGIS je open-source geografický informační systém (\zk{GIS}) distribuovaný pod ~licencí \textit{GNU General Public License}. Mezi~jeho velké výhody patří přenositelnost zdrojového kódu, je dostupný pro~platformy Windows, Linux, Unix, MacOS a~vyvíjí se i~mobilní verze pro~Android.

Vývoj programu, tehdy pod~názvem Quantum GIS, započal roku 2002, později projekt zaštítila organizace Open Source Geospatial Foundation (\zk{OSGeo}) a~verze~1.0 vyšla v~roce 2009. V~současné době jsou verze QGISu pojmenovávány podle~měst.

Systém QGIS nabízí možnost prohlížení, vytváření, editaci a~analýzu prostorových dat, tvorbu mapových výstupů, i~zpracování dat GPS. Podporuje velké množství vektorových, rastrových a~databázových formátů.

Samotný program je napsán v~jazyce C++ a~používá knihovnu Qt. Funkcionalitu programu je možné rozšířit pomocí zásuvných modulů, které mohou být vytvořeny v~jazyce C++ a~nebo~Python~\citep{qgis}~\citep{wiki_qgis}.

\section{Python}
\label{python}

	\begin{figure}[H]
		\centering
		\includegraphics[width=.5\textwidth]{./pictures/python_logo.png}
      	\caption[logo Python]{logo Python (zdroj:~\citep{python})}
		\label{fig:python_logo}
 	\end{figure}

Python je vysokoúrovňový objektově orientovaný programovací jazyk s~dynamickou kontrolou datových typů. Mezi~hlavní myšlenky jazyka Python patří důraz na~čitelnost, která je zajištěna povinným odsazováním datových bloků, a~jednoduchou syntaxi, díky~které jsou programátoři schopni zapsat své nápady na~méně řádcích než~vě~většině běžně používaných programovacích jazyků. Python je vyvíjen jako~open-source software a~nabízí instalační balíky pro~většinu platforem. Jedná se o~interpretovaný jazyk a~je vhodným nástrojem pro psaní skriptů i~rozsáhlých programů. Disponuje širokou nabídkou modulů pro řešení úloh téměr z jakékoli oblasti. Aktuálně se Python vyvíjí ve verzích 2.7.x a 3.x, ovšem v~roce 2020 bude podpora verze 2.7.x ukončena~\citep{python}~\citep{wiki_python}.

\section{SQLite}
\label{sqlite}

	\begin{figure}[H]
		\centering
		\includegraphics[width=.2\textwidth]{./pictures/sqlite_logo.png}
      	\caption[logo SQLite]{logo SQLite (zdroj: \href{https://commons.wikimedia.org/wiki/File:SQLite_Logo_4.png}{Wikimedia Commons})}
		\label{fig:sqlite_logo}
 	\end{figure}

SQLite je relační databázový systém šířený pod~licencí \textit{public domain}. Každá SQLite databáze je uložena v~samostatném souboru, který je nezávislý na~platformě. Na~rozdíl od~většiny databází, SQLite není implementována jako samostatný serverový proces, ale~čte a~zapisuje data přímo z~databázového souboru na~disku. Databáze SQLite nevyžaduje žádnou konfiguraci, částečně je ji možné nastavit příkazy \textit{PRAGMA}~\citep{sqlite}~\citep{wiki_sqlite}.

\section{PyQt}
\label{pyqt}

	\begin{figure}[H]
		\centering
		\includegraphics[width=.2\textwidth]{./pictures/pyqt_logo.png}
      	\caption[logo PyQt]{logo PyQt (zdroj: \href{https://commons.wikimedia.org/wiki/File:Python_and_Qt.svg}{Wikimedia Commons})}
		\label{fig:pyqt_logo}
 	\end{figure}

PyQt je modul, který umožňuje používat knihovnu Qt v~programovacím jazyce Python. Existují dvě verze modulu - PyQt4, která podporuje knihovnu Qt~4, a~PyQt5 pracující s~knihovnou Qt~5. Obě verze jsou vyvíjeny firmou Riverbank Computing. Modul PyQt je dostupný pro všechny platformy, které podporuje knihovna Qt, a~je distribuován pod licencí \textit{GNU GPL v3} nebo~\textit{Riverbank Commercial License}. Nejširší uplatnění nachází~při tvorbě grafického uživatelského prostředí, ale~obsahuje například i~třídy pro~řízení vláken, práci s~databází, parsování \textit{XML} souborů a~další~\citep{pyqt}~\citep{wiki_pyqt}.

\section{GDAL}
\label{gdal}

	\begin{figure}[H]
		\centering
		\includegraphics[width=.2\textwidth]{./pictures/gdal_logo.png}
      	\caption[logo GDAL]{logo GDAL (zdroj: \href{https://commons.wikimedia.org/wiki/File:GDALLogoColor.svg}{Wikimedia Commons})}
		\label{fig:gdal_logo}
 	\end{figure}

GDAL je knihovna pro čtení a~zápis rastrových i~vektorových \zk{GIS} formátů. Je vyvíjena pod záštitou organizace \zk{OSGeo} a vydávána pod licencí \textit{X/MIT}. Pro všechny podporované formáty používá jeden datový model. Samotná knihovna je napsána v programovacím jazyce C++ a obsahuje rozhraní i pro další jazyky~\citep{gdal}~\citep{wiki_gdal}.

