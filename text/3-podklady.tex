\chapter{Podklady}
\label{podklady}

Dvěma nejdůležitějšími podklady pro~zásuvný modul vytvořený v rámci této diplomové práce jsou výměnný formát katastru nemovitostí a~hranice bonitovaných půdně ekologických jednotek.

V~první části této kapitoly je stručně představen výměnný formát katastru nemo\-vitostí. Uvedené informace byly čerpány z~oficiální dokumentace \citep{struktura_vfk}, ukázky formátu pochází právě odtud nebo z~veřejně dostupných dat \citep{zdroj_vfk}.

Bonitovaným půdně ekologickým jednotkám se věnuje druhá část kapitoly, kde za zdroj informací posloužil eKatalog \zk{BPEJ}~\citep{vumop_bpej} a~skripta~\citep{pu_skripta}.

\section{VFK}
\label{vfk}

Na~rozdíl od~starého výměnného formátu obsahuje \zk{VFK} jak soubor popisných informací (\zk{SPI})~– tedy informace o~vlastnících, parcelách, stavbách a~dalších skutečnostech~– tak i~soubor geodetických informací (\zk{SGI})~– informace o~polohovém určení.

Soubory \zk{VFK} jsou poskytovány zpracovatelům pozemkových úprav, pozemko\-vým úřadům, obecním úřadům a~zhotovitelům geometrických plánů. Slouží k~vzájemnému předávání dat mezi~informačním systémem katastru nemovitostí (\zk{ISKN}) a~jiný\-mi systémy.

Výměnný formát katastru nemovitostí je tvořen textovým souborem s~koncovkou \textit{*.vfk}, který má tuto strukturu:
	\begin{itemize}[leftmargin=1.5cm, noitemsep]
		\item \underline{hlavička}~– řádky uvozené \texttt{\&H}
		\item \underline{datové bloky}~– řádky uvozené \texttt{\&B} a \texttt{\&D}
		\item \underline{koncový znak}~– znak \texttt{\&K}
	\end{itemize}

Datový soubor je kódován v~češtině dle~ČSN ISO 8859-2 (ISO Latin2), ve~výjimeč\-ných případech kódování dle
WIN1250. Oddělovačem desetinných čísel je tečka, datum a~čas je zapsán ve~tvaru "\texttt{03.06.1999 09:58:42}", jednotlivé údaje na~řádku jsou odděleny středníkem, textové a~datumové položky se uvádí v~uvozovkách.

Věty hlavičky (\texttt{\&H}), definice bloku (\texttt{\&B}) a~věty dat (\texttt{\&D}) jsou zakončeny znaky \texttt{<CR><LF>}.

\subsection{Hlavička}
\label{hlavicka}

Každý řádek hlavičky začíná skupinu znaků \texttt{\&H}, za~kterou následuje označení položky a~poté samotné údaje oddělené středníkem. Povinné položky hlavičky s~krátkým popisem jsou uvedené v~tab.~\ref{tab:polozky_hlavicky}.

\begin{table}[H]
    \begin{tabular}{|l|l|}
        \hline
         položka & popis \\
        \hline
        \hline
         \texttt{VERZE} & označení verze \zk{VFK} \\ \hline
         \texttt{VYTVORENO} & datum a~čas vytvoření souboru \\ \hline
         \texttt{PUVOD} & původ dat \\ \hline
         \texttt{CODEPAGE} & označení kódování \\ \hline
         \texttt{SKUPINA} & seznam datových bloků \\ \hline
         \texttt{JMENO} & jméno autora souboru \\ \hline
         \texttt{PLATNOST} & časová podmínka použitá pro vytvoření souboru \\ \hline
         \texttt{ZMENY} & typ souboru \\ \hline
         \texttt{KATUZE} & omezující podmínka~– katastrální území \\ \hline
         \texttt{OPSUB} & omezující podmínka~– oprávněné subjekty \\ \hline
         \texttt{PAR} & omezující podmínka~– parcely \\ \hline
         \texttt{POLYG} & omezující podmínka~– polygon \\
         \hline
    \end{tabular}
    \centering
    \caption[\zk{VFK}~– položky hlavičky]{\zk{VFK}~– položky hlavičky (zdroj \citep{struktura_vfk})}
    \label{tab:polozky_hlavicky}
\end{table}

Tab.~\ref{tab:hlavicka_priklady} obsahuje příklady položek hlavičky. Kvůli délce zápisu v~ní nejsou uvedeny příklady pro~omezující podmínky.

\begin{table}[H]
    \begin{tabular}{|l|l|}
        \hline
         položka & příklad \\
        \hline
        \hline
         \texttt{VERZE} & \texttt{\&HVERZE;"5.1"} \\ \hline
         \texttt{VYTVORENO} & \texttt{\&HVYTVORENO;"03.12.2013 09:58:42"} \\ \hline
         \texttt{PUVOD} & \texttt{\&HPUVOD;"ISKN"} \\ \hline
         \texttt{CODEPAGE} & \texttt{\&HCODEPAGE;"WE8ISO8859P2"} \\ \hline
         \texttt{SKUPINA} & \texttt{\&HSKUPINA;"NEMO";"JEDN";"BDPA";"VLST"} \\ \hline
         \texttt{JMENO} & \texttt{\&HJMENO;"Kokeš Petr Ing."} \\ \hline
         \texttt{PLATNOST} & \texttt{\&HPLATNOST;"03.12.2013 09:56:42";"03.12.2013 09:56:42"} \\ \hline
         \texttt{ZMENY} & \texttt{\&HZMENY;0} \\
         \hline
    \end{tabular}
    \centering
    \caption[\zk{VFK}~– příklady položek hlavičky]{\zk{VFK}~– příklady položek hlavičky (zdroj \citep{struktura_vfk})}
    \label{tab:hlavicka_priklady}
\end{table}

\begin{description}
	\item[\texttt{VERZE}:] Právě jeden řádek obsahující informaci o~verzi \zk{VFK} souboru. Tato informace je důležitá pro programy, které s~\zk{VFK} pracují.
	\item[\texttt{VYTVORENO}:] Právě jeden řádek s~časem a~datem vytvoření souboru.
	\item[\texttt{PUVOD}:] Právě jeden řádek specifikující původ dat. Může obsahovat libovolný text.
	\item[\texttt{CODEPAGE}:] Právě jeden řádek označující kódování souboru. Možné hodnoty a~odpo\-vídající kódování popisuje tab.~\ref{tab:kodovani}.

    \begin{table}[H]
        \begin{tabular}{|l|l|}
            \hline
             hodnota & popis \\
            \hline
            \hline
             \texttt{WE8ISO8859P2} & kódování češtiny dle ČSN ISO 8859-2 \\ \hline
             \texttt{EE8MSWIN1250} & kódování češtiny dle MS WIN1250 \\
             \hline
        \end{tabular}
        \centering
        \caption[\zk{VFK} - hodnoty kódování]{\zk{VFK} - hodnoty kódování (zdroj \citep{struktura_vfk})}
        \label{tab:kodovani}
    \end{table}

	\item[\texttt{SKUPINA}:] Právě jeden řádek obsahující seznam datových bloků souboru.
	\item[\texttt{JMENO}:] Právě jeden řádek se~jménem autora souboru.
	\item[\texttt{PLATNOST}:] Právě jeden řádek s~časovou podmínkou použitou pro~vytvoření souboru. Tab.~\ref{tab:platnost} uvádí dvě možnosti zápisu.

    \begin{table}[H]
        \begin{tabular}{|l|l|}
            \hline
             příklad & popis \\
            \hline
            \hline
             \begin{tabular}{@{}l@{}l@{}} \texttt{\&HPLATNOST;} \\ \texttt{"03.12.2013 09:56:42";} \\ \texttt{"03.12.2013         09:56:42"} \end{tabular} & stav dat k~určitému okamžiku \\ \hline
             \begin{tabular}{@{}l@{}l@{}} \texttt{\&HPLATNOST;} \\ \texttt{"03.12.2012 09:56:42";} \\ \texttt{"03.12.2013 09:56:42"} \end{tabular} & stav dat pro určité období \\
             \hline
        \end{tabular}
        \centering
        \caption[\zk{VFK}~– možnosti zápisu časové podmínky]{\zk{VFK}~– možnosti zápisu časové podmínky (zdroj \citep{struktura_vfk})}
        \label{tab:platnost}
    \end{table}

	\item[\texttt{ZMENY}:] Právě jeden řádek informující o~typu souboru. Možné hodnoty a~jejich popis se nachází v~tab.~ \ref{tab:zmeny}.

    \begin{table}[H]
        \begin{tabular}{|l|l|}
            \hline
             hodnota & popis \\
            \hline
            \hline
             \texttt{0} & stavový soubor \\ \hline
             \texttt{1} & změnový soubor \\
             \hline
        \end{tabular}
        \centering
        \caption[\zk{VFK}~– hodnoty typu souborů]{\zk{VFK}~– hodnoty typu souborů (zdroj \citep{struktura_vfk})}
        \label{tab:zmeny}
    \end{table}

Stavový soubor obsahuje všechny informace ke~konkrétnímu času a~datu, ve~změno\-vém souboru se nachází pouze změny za~určitý časový úsek.

	\item[\texttt{KATUZE}, \texttt{OPSUB}, \texttt{PAR}, \texttt{POLYG}:] Soubor \zk{VFK} může být vytvořen pro konkrétní kata\-strální území, oprávněné subjekty, parcely, nebo~pro~oblast zadanou polygonem. Jedná se o~jeden řádek, který obsahuje hlavičku omezující podmínky a~za~ním následují řádky definující samotnou omezující podmínku. V~případě, že je omezující podmínka prázdná, není za~hlavičkou ani jeden řádek s~daty. Příklad pro katastrální území:

\begin{lstlisting}[basicstyle=\footnotesize\ttfamily, backgroundcolor = \color{light-gray},  numbers=left]
&HKATUZE;KOD N6;OBCE_KOD N6;NAZEV T48;PLATNOST_OD D;
PLATNOST_DO D&DKATUZE;693936;550426;"Jama";"19.06.1991 00:00:00";""
 \end{lstlisting}

\end{description}

\subsection{Datové bloky}
\label{datove_bloky}

Každý datový blok obsahuje tyto řádky:
	\begin{itemize}[leftmargin=1.5cm, noitemsep]
		\item \underline{uvozující řádek bloku} - řádek uvozený \texttt{\&B}
		\item \underline{řádky s vlastními daty} - řádky uvozené \texttt{\&D}
	\end{itemize}

\begin{description}	
	\item[Uvozující řádek bloku:] Právě jeden řádek obsahující seznam atributů a~jejich datové typy. V~tabulce \ref{tab:datove_typy} jsou uvedené dostupné datové typy.

\begin{table}[H]
    \begin{tabular}{|l|l|}
        \hline
         zkratka & datový typ \\
        \hline
        \hline
         \texttt{N} & číselný \\ \hline
         \texttt{T} & textový \\ \hline
         \texttt{D} & datumový \\
         \hline
    \end{tabular}
    \centering
    \caption[\zk{VFK}~– datové typy]{\zk{VFK}~– datové typy (zdroj \citep{struktura_vfk})}
    \label{tab:datove_typy}
\end{table}

Pro~číselné položky označuje číslo za~\texttt{N} maximální délku položky. Pro~desetinná čísla udává číslice před~desetinnou tečkou maximální počet číslic, číslice za~desetinnou tečkou definuje počet desetinných míst.

U~textového datového typu číslo za~\texttt{T} značí maximální délku.

Ukázka uvozujícího řádku pro~blok parcela:

	\begin{lstlisting}[basicstyle=\footnotesize\ttfamily, backgroundcolor = \color{light-gray},  numbers=left]
&BPAR;ID N30;STAV_DAT N2;DATUM_VZNIKU D;DATUM_ZANIKU D;
PRIZNAK_KONTEXTU N1;RIZENI_ID_VZNIKU N30;RIZENI_ID_ZANIKU N30;
PKN_ID N30;PAR_TYPE T10;KATUZE_KOD N6;KATUZE_KOD_PUV N6;
DRUH_CISLOVANI_PAR N1;KMENOVE_CISLO_PAR N5;ZDPAZE_KOD N1;
PODDELENI_CISLA_PAR N3;DIL_PARCELY N1;MAPLIS_KOD N30;
ZPURVY_KOD N1;DRUPOZ_KOD N2;ZPVYPA_KOD N4;TYP_PARCELY N1;
VYMERA_PARCELY N9;CENA_NEMOVITOSTI N14.2;DEFINICNI_BOD_PAR T100;
TEL_ID N30;PAR_ID N30;BUD_ID N30;IDENT_BUD T1;SOUCASTI T1;
PS_ID N30;IDENT_PS T1
	\end{lstlisting}

	\item[Řádky s vlastními daty:] Pro~každý objekt jeden řádek.

Ukázka řádku s~vlastními daty pro~objekt parcely:
	
	\begin{lstlisting}[basicstyle=\footnotesize\ttfamily, backgroundcolor = \color{light-gray},  numbers=left]
&DPAR;3067989306;0;"26.06.2003 07:43:05";"";3;3003873306
;;;"PKN";693936;;1;37;;1;;6780;2;13;;;332;;"";674674306;;
323700306;"a";"n";;"n"
	\end{lstlisting}
\end{description}

\subsubsection{Datové bloky důležité pro zásuvný modul}
\label{datove_bloky_zasuvny_modul}

Soubor výměnného formátu katastru nemovitostí obsahuje mnoho datový bloků. Tato sekce se věnuje pouze blokům, které jsou relevantní pro~zásuvný modul.

V~současné době zásuvný modul pracuje s~těmito datovými bloky\footnote{Zásuvný modul nevyužívá dat \zk{BPEJ} přímo ze~souboru \zk{VFK}, protože hranice \zk{BPEJ} není polohopisným prvkem katastrální mapy. Více o~\zk{BPEJ} viz část \ref{bpej}.}:

	\begin{itemize}[leftmargin=1.5cm, noitemsep]
		\item \texttt{\zk{PAR}} - parcely
		\item \texttt{\zk{SOBR}} - souřadnice obrazu bodů polohopisu v~mapě
		\item \texttt{\zk{SPOL}} - souřadnice polohy bodů polohopisu (měřené)
	\end{itemize}

\begin{description}	
	\item[\texttt{PAR}:] Tabulka \texttt{\zk{PAR}} obsahuje parcely evidované v~\zk{ISKN}. Z~pohledu zásuvného modulu vytvořeného v~rámci této práce se jedná o~nejdůležitější část souboru \zk{VFK}.\linebreak Je součástí největší skupiny datových bloků nemovitosti. V~tab.~\ref{tab:par_sloupce} jsou uvedeny sloupce, kterých využívá zásuvný modul.
	
    \begin{table}[H]
        \begin{tabular}{|l|l|l|l|l|}
            \hline
             název & povinný & typ & velikost & popis\\
            \hline
            \hline
            \texttt{ID} & ano & \texttt{N} & \texttt{30.0} & \begin{tabular}{@{}l@{}} unikátní generované \\ číslo parcely \end{tabular} \\ \hline
            \texttt{KMENOVE\_CISLO\_PAR} & ano & \texttt{N} & \texttt{5} & kmenové parcelní číslo \\ \hline
            \texttt{PODDELENI\_CISLA\_PAR} & ne & \texttt{N} & \texttt{3} & poddělení čísla parcely \\ \hline
            \texttt{DRUPOZ\_KOD} & ne & \texttt{N} & \texttt{2.0} & kód druhu pozemku. \\ \hline
            \texttt{VYMERA\_PARCELY} & ano & \texttt{N} & \texttt{9.0} & \begin{tabular}{@{}l@{}} výměra parcely \\ v metrech čtverečních \end{tabular} \\
             \hline
        \end{tabular}
        \centering
        \caption[\zk{VFK}~– sloupce datového bloku \texttt{\zk{PAR}}]{\zk{VFK}~– sloupce datového bloku \texttt{\zk{PAR}} (zdroj \citep{struktura_vfk})}
        \label{tab:par_sloupce}
    \end{table}   
		
	\item[\texttt{SOBR}, \texttt{SPOL}:] Tabulka \texttt{\zk{SOBR}} obsahuje body polohopisu (čísla bodů a~souřadnice\linebreak obrazu v~mapě). V~tabulce \texttt{\zk{SPOL}} jsou uvedeny body polohopisu (čísla bodů a~souřadnice polohy). Obě tabulky jsou součástí skupiny datových bloků prvky katastrální mapy. Zásuvný modul používá pouze jeden sloupec z~těchto datových bloků (viz~tab.~\ref{tab:sobr_spol_sloupce}).
	
    \begin{table}[H]
        \begin{tabular}{|l|l|l|l|l|}
            \hline
             název & povinný & typ & velikost & popis\\
            \hline
            \hline
            \texttt{KODCHB\_KOD} & ne & \texttt{N} & \texttt{2.0} & \begin{tabular}{@{}l@{}} kód charakteristiky \\ kvality bodu \end{tabular} \\
             \hline
        \end{tabular}
        \centering
        \caption[\zk{VFK}~– sloupce datových bloků \texttt{\zk{SOBR}} a~\texttt{\zk{SPOL}}]{\zk{VFK}~– sloupce datových bloků \texttt{\zk{SOBR}} a~\texttt{\zk{SPOL}} (zdroj \citep{struktura_vfk})}
        \label{tab:sobr_spol_sloupce}
    \end{table}
	
\end{description}

\subsection{Koncový znak}
\label{koncovy_znak}

Znak \texttt{\&K} signalizuje konec souboru \zk{VFK}. Pro~software, který načítá \zk{VFK}, to znamená pokyn pro~ukončení importu.

\section{BPEJ}
\label{bpej}

\subsection{Systém BPEJ}
\label{system_bpej}

Bonitovaná půdně ekologická jednotka vyjadřuje produkční potenciál zemědělské půdy s~ohledem na~místo, kde se půda nachází. Systém \zk{BPEJ} vznikl mezi lety~1973 a~1980 na~základě Komplexního průzkumu zemědělských půd. Původně byl systém \zk{BPEJ} zamýšlen jako~podklad pro~plánování zemědělské produkce, ale~po~roce 1989 se začal používat i~pro~jiné účely. Z~toho vyplývají některá jeho omezení a~nedostatky.

Správcem a~garantem údajů \zk{BPEJ} je Výzkumný ústav meliorací a~půdy sídlící v~Praze Zbraslavi.

Od roku 1998 jsou údaje \zk{BPEJ} vedeny v~katastru nemovitostí a~používají je další orgány státní správy. Číselné vyjádření ceny \zk{BPEJ} za~metr čtvereční slouží například pro~výpočet daně z~nemovitostí, pro~stanovení úředních cen zemědělské půdy nebo~pro~určení nároků v~ceně při~pozemkových úpravách.

Celostátní databáze \zk{BPEJ} je od~dubna 2017 veřejně dostupná\footnote{\url{http://www.spucr.cz/bpej/celostatni-databaze-bpej}}. V~době psaní tohoto dokumentu byly hranice \zk{BPEJ} k~nahlížení v mapové aplikaci, nebo bylo data možné stáhnout ve~formátu Esri Shapefile.

\subsection{Kód BPEJ}
\label{kod_bpej}

Kód \zk{BPEJ} zahrnuje tyto vlivy:
	\begin{itemize}[leftmargin=1.5cm, noitemsep]
		\item vlastnosti klimatu
		\item druh půdy
		\item vlastnosti půdy
			\begin{itemize}[leftmargin=1cm, noitemsep]
				\item zrnitost
				\item obsah skeletu
				\item obsah organických částí
				\item hloubka půdy
			\end{itemize}
		\item sklonitost pozemku
		\item orientace pozemku
	\end{itemize}

Vlastnosti a~charakteristiky oblasti \zk{BPEJ} jsou vyjádřeny pětimístným kódem, například:

\begin{align*}
	1.23.45
\end{align*}

kde
\begin{tabbing}
\hspace{2em} \= \hspace{5em} \= \kill
	\> $1$	\> první číslice udává příslušnost do klimatického regionu \\
	\> $23$	\> druhá a~třetí číslice vyjadřují hlavní půdní jednotku \\
	\> $4$	\> čtvrtá číslice zahrnuje sklonitost a~expozici\\
	\> $5$	\> pátá číslice kombinuje obsah skeletu a~hloubku půdy
\end{tabbing}

V~mapách se může vyskytnout zápis s~pomlčkami místo teček, v~počítačovém zpracování se používá zápis bez~dělících znaků.

Pro~všechny nezemědělské nebo nebonitované plochy se od~roku 2008 používá jednotný kód \texttt{99}~\citep{metodika_bpej}.

\subsubsection{Klimatický region}
\label{klimaticky_region}

Klimatický region je území s~přibližně stejnými klimatickými podmínkami pro~růst a~vývoj zemědělských plodin. V~kódu~\zk{BPEJ} se uvádí jako první číslice.

Vymezení klimatických regionů pro~účely systému \zk{BPEJ} bylo provedeno na~zá\-kladě údajů Českého hydrometeorologického ústavu z~let 1901 až~1950. V~úvahu se brala tato kritéria:
	\begin{itemize}[leftmargin=1.5cm, noitemsep]
		\item suma průměrných denních teplot nad $10^\circ$C
		\item průměrná roční teplota
		\item průměrný roční úhrn srážek
		\item pravděpodobnost suchých vegetačních období
		\item vláhová jistota
		\item doplňující hlediska
			\begin{itemize}[leftmargin=1cm, noitemsep]
				\item nadmořská výška
				\item expoziční ráz krajiny
				\item fénové jevy
				\item údaje místních literárních pramenů
				\item vztahy k~dlouhodobým výnosovým řadám
			\end{itemize}
	\end{itemize}

V~České republice je vymezeno 10 klimatických regionů označených kódy \texttt{0}~až~\texttt{9}, od~nejteplejší po~nejchladnější~\citep{vyhlaska_327}. Rozmístění klimatických regionů je na~obr.~\ref{fig:klimaticke_regiony}.

	\begin{figure}[H]
		\centering
		\includegraphics[width=.9\textwidth]{./pictures/klimaticky_region.png}
		\caption[Klimatické regiony]{Klimatické regiony (zdroj:~\citep{vumop_bpej})}
		\label{fig:klimaticke_regiony}
 	\end{figure}

\subsubsection{Hlavní půdní jednotka}
\label{hpj}

Hlavní půdní jednotka je definována jako účelové seskupení půdních forem s~příbuz\-nými ekologickými a~agronomickými vlastnostmi. Je charakterizována genetickým půdním typem, subtypem, půdotvorným substrátem, hloubkou půdního profilu, zrnitostí a~stupněm hydromorfismu. V~systému \zk{BPEJ} se uvádí na~druhém a~třetím místě číselného kódu.

V~současné době systém \zk{BPEJ} vymezuje 78 hlavních půdních jednotek a~ty jsou dále seskupeny do~13 půdních typů. Obr.~\ref{fig:klimaticke_regiony} znázorňuje rozložení půdních typů.

	\begin{figure}[H]
		\centering
		\includegraphics[width=.9\textwidth]{./pictures/pudni_typy.png}
		\caption[Půdní typy]{Půdní typy (zdroj:~\citep{vumop_bpej})}
		\label{fig:pudni_typy}
 	\end{figure}

\subsubsection{Sklonitost a expozice}
\label{sklonitost_expozice}

Sklonitost se rozděluje do~sedmi skupin. V~terénu se sklonitost určuje sklonoměrem, jako pomocný podklad lze využít mapy s~podrobným výškopisem.

Expozice vyjadřuje polohu území \zk{BPEJ} vůči světovým stranám. V~klimatických regionech \texttt{0}, \texttt{1}, \texttt{2}, \texttt{3}, \texttt{4} a~\texttt{5} se jižní expozice samostatně hodnotí jako negativní, zbývající expozice se slučují bez~rozlišení. Samostatně se severní expozice v~klimatických regionech \texttt{6}, \texttt{7}, \texttt{8}, \texttt{9} uvažuje jako negativní, expozice východní, západní a~jižní se hodnotí jako sobě rovné. Expozice se dělí na~čtyři kategorie.

Výsledná třetí číslice kódu \zk{BPEJ} vznikne kombinací sklonitosti a~expozice~\citep{vyhlaska_327}.

\subsubsection{Obsah skeletu a hloubka půdy}
\label{hloubka_pudy_obsah_skeletu}

Obsah skeletu závisí na~obsahu kamene (pevné částice nad 30 mm) a~štěrku (pevné částice hornin od~4~do~30~mm), je rozdělen do~čtyř kategorií.

Hloubka půdy je dána částí půdního profilu omezeného silnou skeletovitostí, nebo pevnou horninou. Ve~vyhlášce~\citep{vyhlaska_327} jsou definovány tři kategorie hloubky půdy.

Na pátém místě číselného kódu \zk{BPEJ} se uvádí kód kombinace obsahu skeletu a~hloubky půdy~\citep{vyhlaska_327}.
