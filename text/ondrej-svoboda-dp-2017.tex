% ========================================================================
% 
% DIPLOMOVÁ PRÁCE - Zásuvný modul QGIS pro zpracování přípravné fáze komplexních pozemkových úprav
% 
% Ondřej Svoboda
% 
% ========================================================================

\documentclass[
  12pt,         			% velikost základního písma je 12 bodů
  a4paper,      			% formát papíru je A4
  oneside,       			% Oboustranný tisk
  pdftex,				    % překlad bude proveden programem 'pdftex' do PDF
]{report}       			% dokument třídy 'zpráva'


\newcommand{\Fbox}[1]{\fbox{\strut#1}}

\usepackage[czech, english]{babel}	% použití češtiny, angličtiny
\usepackage[utf8]{inputenc}			% kódování zdrojových souborů je UTF8

\usepackage[square,sort,comma,numbers]{natbib}

\usepackage{caption}
\usepackage{subcaption}
\usepackage{listings}
\usepackage{color}

\usepackage[dvipsnames]{xcolor}
\definecolor{light-gray}{gray}{0.95}

\captionsetup{font=small}
\usepackage{enumitem} 
\setlist{leftmargin=*} % bez odsazení

\makeatletter
\setlength{\@fptop}{0pt}
\setlength{\@fpbot}{0pt plus 1fil}
\makeatletter

\usepackage[dvips]{graphicx}   
\usepackage{color}
\usepackage{transparent}
\usepackage{wrapfig}
\usepackage{float} 

\usepackage{cmap}           
\usepackage[T1]{fontenc}    

\usepackage{textcomp}
\usepackage[compact]{titlesec}
\usepackage{amsmath}
\addtolength{\jot}{1em} 

\usepackage{chngcntr}
\counterwithout{footnote}{chapter}

\usepackage{acronym}

\usepackage[
    unicode,                
    breaklinks=true,        
    hypertexnames=false,
    colorlinks=true, % true for print version
    citecolor=black,
    filecolor=black,
    linkcolor=black,
    urlcolor=black
]{hyperref}         

\usepackage{url}
\usepackage{fancyhdr}
%\usepackage{algorithmic}
\usepackage{algorithm}
\usepackage{algcompatible}
\renewcommand{\ALG@name}{Pseudokód}% update algorithm name
\def\ALG@name{Pseudokód}

\usepackage[
  cvutstyle,          
  diploma           
]{thesiscvut}


\newif\ifweb
\ifx\ifHtml\undefined % mimo HTML.
    \webfalse
\else % v HTML.
    \webtrue
\fi 

\renewcommand{\figurename}{Obrázek}
\def\figurename{Obrázek}

\lstdefinestyle{python}{
   language=python,
   basicstyle={\footnotesize\ttfamily},
   keywordstyle=\color{blue}\ttfamily,
   stringstyle=\color{green}\ttfamily,
   commentstyle=\color{brown}\ttfamily,
   showstringspaces=false,
   morekeywords={True, False}
}

\renewcommand\lstlistingname{Kód}
\renewcommand*{\lstlistlistingname}{Seznam ukázek kódu}

% ========================================================================
% Definice informací o dokumentu
% ========================================================================

% název práce
\nazev{Zásuvný modul QGIS pro zpracování přípravné fáze komplexních pozemkových úprav}
{Complex Land Consolidation Preliminary Stage QGIS Plugin}

% jméno a příjmení autora
\autor{Ondřej}{Svoboda}

% jméno a příjmení vedoucího práce včetně titulů
\garant{Ing.~Martin~Landa,~Ph.D.}

% označení oboru studia
\oborstudia{Geomatika}{}

% označení ústavu
\ustav{Katedra geomatiky}{}

% rok obhajoby
\rok{2017}

% měsíc obhajoby
\mesic{červen}

% místo obhajoby
\misto{Praha}

% abstrakt
\abstrakt 
{}
{}

% klíčová slova
\klicovaslova
{GIS, QGIS, zásuvný~modul, python, pozemkové úpravy}
{GIS, QGIS, plugin, python, land consolidation}

% ========================================================================
% Nastavení polí ve vlastnostech dokumentu PDF
% ========================================================================
\nastavenipdf

% začátek dokumentu
\begin{document}

\catcode`\-=12  % pro vypnutí aktivního znaku '-' používaného např. v \cline 

% aktivace záhlaví
\zahlavi

% předefinování vzhledu záhlaví
\renewcommand{\chaptermark}[1]{%
	\markboth{\MakeUppercase
	{%
	\thechapter.%
	\ #1}}{}}

% vysázení přebalu práce
%\vytvorobalku

% vysázení titulní stránky práce
\vytvortitulku

% Vysázení listu zadani
\stranka{}%
	{\includegraphics[scale=0.7]{./pictures/zadanidp.pdf}}%\sffamily\Huge\centering\ }%ZDE VLOŽIT LIST ZADÁNÍ}%
	%{\sffamily\centering Z~důvodu správného číslování stránek}

% vysázení stránky s abstraktem
\vytvorabstrakt

% vysázení prohlaseni o samostatnosti
\vytvorprohlaseni

% vysázení poděkování
\stranka{%nahore
       }{%uprostred
       }{%dole
       \sffamily
	\begin{flushleft}
		\large
		\MakeUppercase{Poděkování}
	\end{flushleft}
	\vspace{1em}
		%\noindent
	\par\hspace{2ex}
	{Chtěl bych poděkovat všem.}
}

% vysázení obsahu
\setcounter{tocdepth}{1}
\obsah

% vysázení seznamu obrázků
\seznamobrazku

% vysázení seznamu tabulek
\seznamtabulek

% vysázení seznamu ukázek kódu
\cleardoublepage
\thispagestyle{empty}
\lstlistoflistings
\newpage

% jednotlivé kapitoly
\chapter{Úvod}
\label{1-uvod}

Vlivem působení člověka se krajina v~České republice změnila. Zanikly polní cesty, přirozené liniové prvky a~další krajinotvorné elementy. Důsledkem toho došlo ke~snížení ekologické stability krajiny, poškození zemědělského půdního fondu vodní a~větrnou erozí, narušení krajinného rázu a~zhoršení životního prostředí. Postupným převáděním, dělením a~slučováním vznikly pozemky, které mají nevhodné tvary, jsou přerušené komunikacemi, polními cestami, vodními toky, či~se~nachází uvnitř jiného bloku pozemků a~tím pádem na~ně není přístup.

Na~velké části území se stále používají katastrální mapy, jejichž původ se datuje do~1.~poloviny 19.~století, a~katastr nemovitostí bohužel i~při veškeré snaze obsahuje řadu chyb. Tato situace nejasného vlastnictví omezuje možnosti hospodaření a~komplikuje podnikání.

Všechny uvedené problémy a~mnohé další je možné zmírnit, či~úplně odstranit pomocí pozemkových úprav (\zk{PU}). Výsledkem pozemkových úprav je nová digitální katastrální mapa, obnovený operát katastru nemovitostí a~nové uspořádání pozemků. V~terénu jsou vyznačeny hranice nových pozemků, je vybudována síť polních cest, protierozních opatření a~jsou vymezeny prostory pro~prvky zvyšující ekologickou stabilitu.

Složitého procesu pozemkových úprav se účastní mnoho odborníků a~jejich práce se neobejde bez~kvalitního softwaru. V~současné době se všechny běžně používané programy pro~pozemkové úpravy řadí mezi komerční software a~jsou distribuovány pouze pro~platformu Windows.

S~rozmachem open-source projektů se nabízí možnost poskytnout zpracovatelům pozemkových úprav svobodný a~otevřený software, který by při~své práci mohli využít.

Mezi stále populárnější open-source programy pro~práci s~prostorovými daty patří geografický informační systém QGIS. Hlavními výhodami systému QGIS jsou intuitivní grafické uživatelské rozhraní a~široká nabídka zásuvných modulů (pluginů), které rozšiřují jeho funkcionalitu. Pluginy do~programu QGIS lze vyvíjet v~programovacím jazyce C++ nebo Python, v~prospěch jazyka Python oproti C++ hovoří lepší přenositelnost mezi platformami. Systém QGIS navíc používá knihovnu GDAL, jejíž součástí je VFK Driver. Ten umožňuje čtení důležitého formátu dat pro~pozemkové úpravy~– výměnného formátu katastru nemovitostí (\zk{VFK}).

Tato práce se zabývá vývojem nástroje pro~zpracování pozemkových úprav, který implementuje jako zásuvný modul do~programu QGIS psaný v~programovacím jazyce Python. Zaměřuje se na~jeden z~úvodních kroků pozemkových úprav~– přípravnou fázi.

První kapitola teoretické části se zabývá samotnými pozemkovými úpravami. Snaží se poskytnout informace, díky kterým bude patrné, jak v~celém procesu figuruje vytvořený zásuvný modul. Zvýšená pozoronost je věnována sestavení soupisů nároků vlastníků, pojednává i~o~nejvíce rozšířených programech pro~pozemkové úpravy.

Druhá kapitola teoretického úvodu popisuje dva nejdůležitější podklady zásuvného modulu~– \zk{VFK} a~hranice bonitovaných půdně ekologických jednotek (\zk{BPEJ}).

Praktická část se zaobírá samotným zásuvným modulem a~jeho technickým řešením. Neobsahuje návod, jak se~zásuvným modulem pracovat, to je předmětem uživatelského manuálu v~příloze tohoto dokumentu.

\chapter{Pozemkové úpravy}
\label{2-pu}

Tato kapitola se věnuje pozemkovým úpravám. Popisuje význam, důvody, cíle, formy a~celý proces pozemkových úprav s~důrazem na~části, kterých se týka zásuvný modul vytvořený v~rámci této práce.

V~této části bylo čerpáno z~\citep{pu_zakon} \citep{pu_cr} \citep{metodicky_navrh} a~\citep{pu_skripta}.

\section{Pojem pozemkových úprav}
\label{pojem_pu}

Pozemkové úpravy zahrnují mnoho na~sebe navazujícíh činností, jejichž společným cílem je zlepšení podmínek pro~zemědelské hospodaření, zpřístupnění pozemků, zmírnění nepříznivých účinků vodní a~větrné eroze, zlepšení životního prostředí, zvýšení ekologické stability krajiny a~zachování či~obnova krajinného rázu. Děje se tak pomocí prostorového a~funkčního uspořádávání pozemků, pozemky se dělí a~scelují. K pozemkům se vyhotovují vlastnická práva a~s~tím související věcná břemena. Výsledky pozemkových úprav slouží jako podklady pro obnovu katastrálního operátu.

Pozemkové úpravy jsou multidiscilinární obor, který využívá znalostí a poznatků z mnoha dalších oborů. Mezi ně patří zemědělství, krajinné a~územní plánování, geodézie, fotogrammetrie, vodohospodářství, ochrana životního prostředí, katastr nemovitostí a další. Důležitá je spolupráce všech odborníků, aby byla zajištěna plynulá návaznost prací.

\section{Význam pozemkových úprav}
\label{vyznam_pu}

Pozemkové úpravy mají význam jak pro~účastníky pozemkových úprav, tedy vlastníky, stavebníky a~obce, tak pro obyvatele a~návštěvníky venkova, orgány státní správy, podnikatelsé subjekty, správce inženýrských sítí a~zájmové organizace. Ve výsledku mají tedy pozemkové úpravy dopad na~životy jednotlivců, společnosti a celého státu.

Význam \zk{PÚ} pro vlastníky a~nájemce půdy:
	\begin{itemize}[noitemsep, leftmargin=1.5cm]
		\item přehledné a~jasné vlastnické vztahy
		\item vytyčené hranice pozemků v terénu
		\item zajištěný přístup na pozemky
		\item lepší tvar pozemků vhodných pro~racionální zemědělské hospodaření
		\item možnost uzavřít nájemní smlouvy na~přesné výměry a hranice pozemků
		\item lepší organizace půdní držby
		\item zvýšená tržní cena pozemků
	\end{itemize}

Význam \zk{PÚ} pro zemědělské subjekty:
	\begin{itemize}[noitemsep, leftmargin=1.5cm]
		\item lepší tvar pozemků vhodných pro~racionální zemědělské hospodaření
		\item zajištěný přístup na~pozemky
		\item možnost uzavření nájemních smluv na~přesné výměry a~hranice pozemků
		\item možnost žádat o dotace
	\end{itemize}

Význam \zk{PÚ} pro obce:
	\begin{itemize}[noitemsep, leftmargin=1.5cm]
		\item vyjasněné právnické vztahy v~území
		\item zpřístupnění a~zprůchodnění krajiny
		\item nalezení a~zapsání historického majetku obce
		\item podrobná dokumentace o území
		\item realizace společných zařízení za~státní peníze
		\item podklad pro zpracování územního plánu
		\item zvýšená ekologická stabilita území
		\item protipovodňová ochrana obce
		\item podpora pěší turistiky a~cykloturistiky
		\item zkvalitnění života na~venkově
	\end{itemize}

Význam \zk{PÚ} pro orgány státní správy:
	\begin{itemize}[noitemsep, leftmargin=1.5cm]
		\item obnova katastrálního operátu
		\item odstranění zjednodušené evidence
		\item nová digitální katastrální mapa
		\item nové podrobné polohové bodové pole
		\item zvýšená retence krajiny
		\item snížení eroze
		\item zvýšená ekologická stabilita
		\item ochrana povrchových a~podzemních vod
	\end{itemize}

\section{Důvody pro pozemkové úprav}
\label{duvody_pu}

Důvodů k zahájení pozemkových úprav býva obvykle několik, přičemž jeden či více mají větší prioritu a ostatní jsou spíše doplňující.

Zde jsou vyjmenovány nejčastější důvody pro pozemkové úpravy:
	\begin{itemize}[noitemsep, leftmargin=1.5cm]
		\item území s nedokončeným přídělovým nebo scelovacím řízením
		\item území s množstvím jednoduchýh pozemkových úprav
		\item investiční záměr velkého rozsahu
		\item žádost vlastníků nadpoloviční výměry
		\item vyjasnění a uspořádání vlastnických vztahů
		\item nevhodné tvary pozemků
		\item zpřístupnění pozemků a krajiny
		\item nízká ekologická stabilita
		\item protipovodňová ochrana
		\item obnova katastrálního operátu
		\item návaznost na sousední katastrální území
	\end{itemize}

\section{Cíle pozemkových úprav}
\label{cile_pu}

Cíle pozemkových úpravy úzce souvisí s~důvody jejích zahájení. Snahou je soustředit se na~hlavní cíle a~zároveň neopomenout cíle vedlejší.

Toto jsou hlavní cíle většiny pozemkových úprav:
	\begin{itemize}[noitemsep, leftmargin=1.5cm]
		\item vyjasnění a uspořádání vlastnických práv
		\item zlepšení podmínek pro~racionální zemědělské hospodaření
		\item scelení roztříštěných pozemků jednoho vlastníka do~menšího počtu větších pozemků
		\item zlepšení tvaru pozemků pro~hospodaření
		\item zajištění přístupu na~pozemky
		\item zvýšení ekologické stability území
		\item zvýšení retence krajiny
		\item protipovodňová ochrana
		\item ochrana a~zúrodnění půdního fondu
	\end{itemize}

\section{Formy pozemkových úprav}
\label{formy_pu}

\subsection{Jednoduché pozemkové úpravy}
\label{jednoduche_pu}

Jak název napovídá, jednoduché pozemkové úpravy se týkají menší oblasti, obyčejně části katastrálního území.

Varianta \zk{JPÚ} bez přechodu vlastnických práv se používala například po roce 1990, kdy bylo potřeba narychlo umožnit hospodaření jednotlivým zemědělským subjektům, ovšem od roku 2002 se již tyto \zk{JPÚ} neprovádějí.

V současné době se zahajují již jen \zk{JPÚ} se zápisem vlastnických práv do katastru nemovitostí. Tato varianta \zk{PÚ} se používá například v pohraničních oblastech, kde jsou v důsledku nedokončených přídělových řízení z~poválečného období nedořešené právnické vztahy, v~místech, kde vlastníci ve~velké většině souhlasí s~obnovou pozemků dle původní pozemkové evidence, nebo~v~oblastech, kde je nutné vyřešit specifický problém jako velké ohrožení pozemků půdní erozí, či~povodněmi.

\subsection{Komplexní pozemkové úpravy}
\label{komplexní_pu}

Komplexní pozemkové úpravy zpravidla řeší nezastavěné území - extravilán - celého katastrálního území. Cílem \zk{KPÚ} není pouze jeden konkrétní problém, jak tomu může být u~\zk{JPÚ}, ale snaží se uspořádat pozemky v~širším kontextu. 

\section{Obvod a předmět pozemkových úprav}
\label{obvod_a_predmet_pu}

\subsection{Obvod pozemkových úprav}
\label{obvod_pu}

Obvod pozemkových úprav je území dotčené pozemkovými úpravami, které je tvořeno jedním nebo více celky v jednom katastrálním území. V případě potřeby lze do \zk{ObPÚ} zahrnout i~navazující části sousedních katastrálních území. Hranice obvodu pozemkové úpravy býva obvykle rozdělena na vnitřní a vnější. Vnitřní hranice obvodu je nejčastěji určena hranicí mezi zastavěnou částí obce - intravilánem - a nezastavěným územím - extravilánem. Vnější hranice nejčastěji prochází po hranici katastrálního území, po hranici lesa, liniového objektu či průmyslového areálu, může zasahovat i do sousedních katastrálních území. Při volbě obvodu pozemkové úpravy by měly být zohledněny širší územní vztahy, neboť síť cest, ani oblasti ohrožené erozí či povodněmi se neřídí podle hranic katastrálních území. Z důvodu komplikovaného oceňování lesní pozemky zpravidla nebývají předmětem pozemkových úprav, obvod většinou končí na jejich okraji.

\subsection{Předmět pozemkových úpravy}
\label{predmet_pu}

Všechny pozemky v obvodu pozemkových úprav bez ohledu na dosavadní způsob využívání a stávající vlastnické vztahy jsou předmětem \zk{PÚ}. Převážně se jedná o zemědělské pozemky, ale i další pozemky v extravilánu mohou být zahrnuty. Pozemky v \zk{ObPÚ} se dělí na tyto skupiny:
	\begin{itemize}[leftmargin=1.5cm]
		\item \underline{pozemky v~\zk{ObPÚ} řešené} - pozemky, u~kterých ve většině případů dochází ke~změnám v jejich poloze. Mohou být děleny, scelovány a~musí být zajištěna jejich přístupnost.
		\item \underline{pozemky v~\zk{ObPÚ} neřešené} - pozemky v~obvodu pozemkových úprav, u~kterých se pouze obnovují geodetické informace. U~těchto pozemků se zjistí průběh jejich hranic, označí se lomové body a~vypočítá se nová výměra ze~souřadnic v~S-JTSK. Do~\zk{PÚ} jsou zahrnuty proto, aby nová katastrální mapa neobsahovala vynechané části. Tyto pozemky se neoceňují.
		\item \underline{pozemky mimo \zk{ObPÚ}} - pozemky, které nejsou předmětem řízení o~pozemkových úpravách. Nesměňují se, nezpřístupňují, nezaměřují a~ani neoceňují. Nerozhoduje o~nich pozemkový úřad.
	\end{itemize}

\section{Fáze pozemkových úprav}
\label{etapy_pu}

\subsection{Programová fáze}
\label{programova_faze}

Programová fáze je plně v kompetenci pozemkového úřadu. Cílem je vytvořit strategii v rámci mikroregionu, okresu, kraje a~státu. Tato činnost podléhá momentálním politickým představám a období vývoje společnosti. Vychází z aktuálních priorit, ale měla by také sledovat dlouhodobou kontinuitu v životě občanské společnosti. Důležitou roli zde hraje agrární politika státu.

Pozemkový úřad shromažduje a~vyhodnocuje informace o~katastrálních územích, zjištuje zájem vlastníků, obcí a~nájemců o~provedení \zk{PÚ}. Na základě výsledného pořadníku katastrálních území a finančních možností potom pozemkový úřad zahajuje pozemkové úpravy a~informuje o~tom další orgány státní správy, kterých se budou \zk{PÚ} týkat. Ve~veřejném výběrovém řízení je vybrán zpracovatel, se kterým pozemkový úřad podepíše obchodní smlouvu.

\subsection{Přípravná fáze}
\label{pripravna_faze}

Na rozdíl od~programové fáze, která může probíhat na~úrovni státu, krajů a~okresů, příprávná fáze se týká konkrétního vybraného katastrálního území. Zadavatel, pozemkový úřad, stanovuje cíle, rozsah a~zásady zpracování. Hledání cílů, úprava obvodu a jiné korekce během následující návrhové etapy jsou velmi nežádoucí, neefektivní a protahují již tak dlouhou dobu pozemkových úprav, proto je dobré dbát zvýšenou pozornost právě této etapě. Ovšem i přesto se v~tak složitém procesu, jakým pozemkové úpravy bezesporu jsou, mohou vyskytnout nové skutečnosti, které pozmění dílčí cíle a obchodní smlouvu mezi zadavatelem a zpracovatelem. Z praxe je zřejmé, že právě přípravnou fázi pozemkové úřady, zpracovatelé a katastrální úřady mnohdy podceňují.

Zásadní věcí, kterou je v přípravné fázi nutné vyřešit, je předběžné určení obvodu. Obecným doporučením je, že obvod pozemkové úpravy by měl být stanoven tak, aby řešil identifikované problémy území. Z~toho vyplývá, že se nelze striktně držet hranice katastrálního území. Do~obvodu je možné zahrnout i~navazující části sousedních katastrálních území, nebo naopak lze některé části, ve kterých není nutná změna, například lesní komplexy, vynechat. Důležité je také zohledit, aby nová digitální mapa byla co nejvíce souvislá a obsahovala co nejméně prázdných míst.
\chapter{Podklady}
\label{podklady}

Dvěma nejdůležitějšími podklady pro~zásuvný modul vytvořený v rámci mé diplomové práce jsou výměnný formát katastru nemovitostí (\zk{VFK}) a hranice bonitovaných půdně ekologických jednotek (\zk{BPEJ}).

V první části této kapitoly je stručně představen výměnný formát katastru nemovitostí. Uvedené informace byly čerpány z~oficiální dokumentace \citep{struktura_vfk}, ukázky formátu pochází právě odtud nebo z~veřejně dostupných dat \citep{zdroj_vfk}.

Bonitovaným půdně ekologickým jednotkám se věnuje druhá část kapitoly, kde za zdroj informací posloužila skripta~\citep{pu_skripta} a eKatalog \zk{BPEJ}~\citep{vumop_bpej}.

\section{VFK}
\label{vfk}

Na~rozdíl od~starého výměnného formátu (\zk{SVF}) obsahuje \zk{VFK} jak soubor popisných informací (\zk{SPI}) - tedy informace o~vlastnících, parcelách, stavbách a~dalších skutečnostech - tak i~soubor geodetických informací (\zk{SGI}) - informace o~polohovém určení.

Soubory \zk{VFK} jsou poskytovány zpracovatelům pozemkových úprav, pozemkovým úřadům, obecním úřadům a~zhotovitelům geometrických plánů. Slouží k~vzájemnému předávání dat mezi~informačním systémem katastru nemovitostí (\zk{ISKN}) a~jinými systémy.

Výměnný formát katastru nemovitostí je tvořen textovým souborem s~koncovkou *.vfk, který má tuto strukturu:
	\begin{itemize}[leftmargin=1.5cm, noitemsep]
		\item \underline{hlavička} - řádky uvozené \texttt{\&H}
		\item \underline{datové bloky} - řádky uvozené \texttt{\&B} a \texttt{\&D}
		\item \underline{koncový znak} - znak \texttt{\&K}
	\end{itemize}

Datový soubor je kódován v češtině dle~ČSN ISO 8859-2 (ISO Latin2), ve~výjimečných případech kódování dle
WIN1250. Oddělovačem desetinných čísel je tečka, datum a~čas je zapsán ve~tvaru "03.06.1999 09:58:42", jednotlivé údaje na~řádku jsou odděleny středníkem, textové a~datumové položky se uvádí v~uvozovkách.

Věty hlavičky (\texttt{\&H}), definice bloku (\texttt{\&B}) a~věty dat (\texttt{\&D}) jsou zakončeny znaky \texttt{<CR><LF>}.

\subsection{Hlavička}
\label{hlavicka}

Každý řádek hlavičky začíná skupinu znaků \texttt{\&H}, za~kterou následuje označení položky a~poté samotné údaje oddělené středníkem. Povinné položky hlavičky s~krátkým popisem jsou uvedené v~tabulce \ref{tab:polozky_hlavicky}.

\begin{table}[H]
    \begin{tabular}{|l|l|}
        \hline
         položka & popis \\
        \hline
        \hline
         VERZE & označení verze \zk{VFK} \\ \hline
         VYTVORENO & datum a čas vytvoření souboru \\ \hline
         PUVOD & původ dat \\ \hline
         CODEPAGE & označení kódování \\ \hline
         SKUPINA & seznam datových bloků \\ \hline
         JMENO & jméno autora souboru \\ \hline
         PLATNOST & časová podmínka použitá pro vytvoření souboru \\ \hline
         ZMENY & typ souboru \\ \hline
         KATUZE & omezující podmínka - katastrální území \\ \hline
         OPSUB & omezující podmínka - oprávněné subjekty \\ \hline
         PAR & omezující podmínka - parcely \\ \hline
         POLYG & omezující podmínka - polygon \\
         \hline
    \end{tabular}
    \centering
    \caption[Položky hlavičky]{Položky hlavičky (zdroj \citep{struktura_vfk})}
    \label{tab:polozky_hlavicky}
\end{table}

Tabulka \ref{tab:hlavicka_priklady} obsahuje příklady položek hlavičky. Kvůli délce zápisu v~ní nejsou uvedeny příklady pro~omezující podmínky.

\begin{table}[H]
    \begin{tabular}{|l|l|}
        \hline
         položka & příklad \\
        \hline
        \hline
         VERZE & \texttt{\&HVERZE;"5.1"} \\ \hline
         VYTVORENO & \texttt{\&HVYTVORENO;"03.12.2013 09:58:42"} \\ \hline
         PUVOD & \texttt{\&HPUVOD;"ISKN"} \\ \hline
         CODEPAGE & \texttt{\&HCODEPAGE;"WE8ISO8859P2"} \\ \hline
         SKUPINA & \texttt{\&HSKUPINA;"NEMO";"JEDN";"BDPA";"VLST"} \\ \hline
         JMENO & \texttt{\&HJMENO;"Kokeš Petr Ing."} \\ \hline
         PLATNOST & \texttt{\&HPLATNOST;"03.12.2013 09:56:42";"03.12.2013 09:56:42"} \\ \hline
         ZMENY & \texttt{\&HZMENY;0} \\
         \hline
    \end{tabular}
    \centering
    \caption[Příklady položek hlavičky]{Příklady položek hlavičky (zdroj \citep{struktura_vfk})}
    \label{tab:hlavicka_priklady}
\end{table}

\begin{description}
	\item[VERZE:] Právě jeden řádek obsahující informaci o~verzi \zk{VFK} souboru. Tato informace je důležitá pro programy, které s \zk{VFK} pracují.
	\item[VYTVOŘENO:] Právě jeden řádek s~časem a~datem vytvoření souboru.
	\item[PŮVOD:] Právě jeden řádek specfikující původ dat. Může obsahovat libovolný text.
	\item[CODEPAGE:] Právě jeden řádek označující kódóvání souboru. Možné hodnoty a~odpovídající kódování popisuje tabulka 	\ref{tab:kodovani}.

    \begin{table}[H]
        \begin{tabular}{|l|l|}
            \hline
             hodnota & popis \\
            \hline
            \hline
             \texttt{WE8ISO8859P2} & kódování češtiny dle ČSN ISO 8859-2 \\ \hline
             \texttt{EE8MSWIN1250} & kódování češtiny dle MS WIN1250 \\
             \hline
        \end{tabular}
        \centering
        \caption[Hodnoty kódóvání a jejich popis]{Hodnoty kódóvání a jejich popis (zdroj \citep{struktura_vfk})}
        \label{tab:kodovani}
    \end{table}

	\item[SKUPINA:] Právě jeden řádek obsahující seznam datových bloků souboru.
	\item[JMÉNO:] Právě jeden řádek se~jménem autora souboru.
	\item[PLATNOST:] Právě jeden řádek s~časovou podmínkou použitou pro~vytvoření souboru. Tabulka \ref{tab:platnost} uvádí dvě možnosti zápisu.

    \begin{table}[H]
        \begin{tabular}{|l|l|}
            \hline
             příklad & popis \\
            \hline
            \hline
             \begin{tabular}{@{}l@{}l@{}} \texttt{\&HPLATNOST;} \\ \texttt{"03.12.2013 09:56:42";} \\ \texttt{"03.12.2013         09:56:42"} \end{tabular} & stav dat k určitému okamžiku \\ \hline
             \begin{tabular}{@{}l@{}l@{}} \texttt{\&HPLATNOST;} \\ \texttt{"03.12.2012 09:56:42";} \\ \texttt{"03.12.2013 09:56:42"} \end{tabular} & stav dat pro určité období \\
             \hline
        \end{tabular}
        \centering
        \caption[Možnosti zápisu časové podmínky]{Možnosti zápisu časové podmínky (zdroj \citep{struktura_vfk})}
        \label{tab:platnost}
    \end{table}

	\item[ZMĚNY:] Právě jeden řádek informující o~typu souboru. Možné hodnoty a~jejich popis se nachází v~tabulce \ref{tab:zmeny}.

    \begin{table}[H]
        \begin{tabular}{|l|l|}
            \hline
             hodnota & popis \\
            \hline
            \hline
             \texttt{0} & stavový soubor \\ \hline
             \texttt{1} & změnový soubor \\
             \hline
        \end{tabular}
        \centering
        \caption[Hodnoty typu souborů a~jejich popis]{Hodnoty typu souborů a jejich popis (zdroj \citep{struktura_vfk})}
        \label{tab:zmeny}
    \end{table}

Stavový soubor obsahuje všechny informace ke~konkrétnímu času a~datu, ve~změnovém souboru se nachází pouze změny za~určitý časový úsek.

	\item[KATUZE, OPSUB, PAR, POLYG:] Soubor \zk{VFK} může být vytvořen pro konkrétní katastrální území, oprávněné subjekty, parcely, nebo~pro~oblast zadanou polygonem. Jedná se o~jeden řádek, který obsahuje hlavičku omezující podmínky a~za~ním následují řádky definující samotnou omezující podmínku. V~případě, že je omezující podmínka prázdná, není za~hlavičkou ani jeden řádek s daty. Příklad pro katastrální území:

\begin{lstlisting}[basicstyle=\footnotesize\ttfamily, backgroundcolor = \color{light-gray},  numbers=left]
&HKATUZE;KOD N6;OBCE_KOD N6;NAZEV T48;PLATNOST_OD D;
PLATNOST_DO D&DKATUZE;693936;550426;"Jama";"19.06.1991 00:00:00";""
 \end{lstlisting}

\end{description}

\subsection{Datové bloky}
\label{datove_bloky}

Každý datový blok obsahuje tyto řádky:
	\begin{itemize}[leftmargin=1.5cm, noitemsep]
		\item \underline{uvozující řádek bloku} - řádek uvozený \texttt{\&B}
		\item \underline{řádky s vlastními daty} - řádky uvozené \texttt{\&D}
	\end{itemize}

\begin{description}	
	\item[Uvozující řádek bloku:] Právě jeden řádek obsahující seznam atributů a~jejich datové typy. V~tabulce \ref{tab:datove_typy} jsou uvedené dostupné datové typy.

\begin{table}[H]
    \begin{tabular}{|l|l|}
        \hline
         zkratka & datový typ \\
        \hline
        \hline
         \texttt{N} & číselný \\ \hline
         \texttt{T} & textový \\ \hline
         \texttt{D} & datumový \\
         \hline
    \end{tabular}
    \centering
    \caption[Datové typy]{Datové typy (zdroj \citep{struktura_vfk})}
    \label{tab:datove_typy}
\end{table}

Pro~číselné položky označuje číslo za~\texttt{N} maximální délku položky. Pro desetinná čísla udává číslice před~desetinnou tečkou maximální počet číslic, číslice za~desetinnou tečkou definuje počet desetinných míst.

U~textového datového typu číslo za~\texttt{T} značí maximální délku.

Ukázka uvozujícího řádku pro~blok parcela:

	\begin{lstlisting}[basicstyle=\footnotesize\ttfamily, backgroundcolor = \color{light-gray},  numbers=left]
&BPAR;ID N30;STAV_DAT N2;DATUM_VZNIKU D;DATUM_ZANIKU D;
PRIZNAK_KONTEXTU N1;RIZENI_ID_VZNIKU N30;RIZENI_ID_ZANIKU N30;
PKN_ID N30;PAR_TYPE T10;KATUZE_KOD N6;KATUZE_KOD_PUV N6;
DRUH_CISLOVANI_PAR N1;KMENOVE_CISLO_PAR N5;ZDPAZE_KOD N1;
PODDELENI_CISLA_PAR N3;DIL_PARCELY N1;MAPLIS_KOD N30;
ZPURVY_KOD N1;DRUPOZ_KOD N2;ZPVYPA_KOD N4;TYP_PARCELY N1;
VYMERA_PARCELY N9;CENA_NEMOVITOSTI N14.2;DEFINICNI_BOD_PAR T100;
TEL_ID N30;PAR_ID N30;BUD_ID N30;IDENT_BUD T1;SOUCASTI T1;
PS_ID N30;IDENT_PS T1
	\end{lstlisting}

	\item[Řádky s vlastními daty:] Pro~každý objekt jeden řádek.

Ukázka řádku s~vlastními daty pro~objekt parcely.
	
	\begin{lstlisting}[basicstyle=\footnotesize\ttfamily, backgroundcolor = \color{light-gray},  numbers=left]
&DPAR;3067989306;0;"26.06.2003 07:43:05";"";3;3003873306
;;;"PKN";693936;;1;37;;1;;6780;2;13;;;332;;"";674674306;;
323700306;"a";"n";;"n"
	\end{lstlisting}
\end{description}

\subsubsection{Datové bloky důležité pro zásuvný modul}
\label{datove_bloky_zasuvny_modul}

Soubor výměnného formátu katastru nemovitostí obsahuje mnoho datový bloků. Tato sekce se věnuje pouze blokům, které jsou reletvatní pro~zásuvný modul.

V~současné době zásuvný modul pracuje s~těmito datovými bloky\footnote{Zásuvný modul nevyužívá dat \zk{BPEJ} přímo ze~souboru \zk{VFK}, protože hranice \zk{BPEJ} není polohopisným prvkem katastrální mapy. Více o~\zk{BPEJ} viz část \ref{bpej}.}:

	\begin{itemize}[leftmargin=1.5cm, noitemsep]
		\item \zk{PAR} - parcely
		\item \zk{SOBR} - souřadnice obrazu bodů polohopisu v~mapě
		\item \zk{SPOL} - souřadnice polohy bodů polohopisu (měřené)
	\end{itemize}

\begin{description}	
	\item[PAR:] Tabulka \zk{PAR} obsahuje parcely evidované v~\zk{ISKN}. Z~pohledu zásuvného modulu vytvořeného v~rámci této práce se jedná o~nejdůležitější část souboru \zk{VFK}. Je součástí největší skupiny datových bloků nemovitosti. V~tabulce \ref{tab:par_sloupce} jsou uvedeny sloupce, kterých využívá zásuvný modul.
	
    \begin{table}[H]
        \begin{tabular}{|l|l|l|l|l|}
            \hline
             název & povinný & typ & velikost & popis\\
            \hline
            \hline
            ID & ano & N & 30.0 & \begin{tabular}{@{}l@{}} unikátní generované \\ číslo parcely \end{tabular} \\ \hline
            KMENOVE\_CISLO\_PAR & ano & N & 5 & kmenové parcelní číslo \\ \hline
            PODDELENI\_CISLA\_PAR & ne & N & 3 & poddělení čísla parcely \\ \hline
            DRUPOZ\_KOD & ne & N & 2.0 & kód druhu pozemku. \\ \hline
            VYMERA\_PARCELY & ano & N & 9.0 & \begin{tabular}{@{}l@{}} výměra parcely \\ v metrech čtverečních \end{tabular} \\
             \hline
        \end{tabular}
        \centering
        \caption[Sloupce datového bloku \zk{PAR}]{Sloupce datového bloku \zk{PAR} (zdroj \citep{struktura_vfk})}
        \label{tab:par_sloupce}
    \end{table}   
		
	\item[SOBR, SPOL:] Tabulka \zk{SOBR} obsahuje body polohopisu (čísla bodů a~souřadnice obrazu v~mapě). V~tabulce \zk{SPOL} jsou uvedeny body polohopisu (čísla bodů a~souřadnice polohy). Obě tabulky jsou součástí skupiny datových bloků prvky katastrální mapy. Zásuvný modul používá pouze jeden sloupec z~těchto datových bloků (viz~\ref{tab:sobr_spol_sloupce}).
	
    \begin{table}[H]
        \begin{tabular}{|l|l|l|l|l|}
            \hline
             název & povinný & typ & velikost & popis\\
            \hline
            \hline
            KODCHB\_KOD & ne & N & 2.0 & \begin{tabular}{@{}l@{}} kód charakteristiky \\ kvality bodu \end{tabular} \\
             \hline
        \end{tabular}
        \centering
        \caption[Sloupce datových bloků \zk{SOBR} a \zk{SPOL}]{Sloupce datových bloků \zk{SOBR} a \zk{SPOL} (zdroj \citep{struktura_vfk})}
        \label{tab:sobr_spol_sloupce}
    \end{table}
	
\end{description}

\subsection{Koncový znak}
\label{koncovy_znak}

Znak \texttt{\&K} signalizuje konec souboru \zk{VFK}. Pro~software, který načítá \zk{VFK}, to znamená pokyn pro~ukončení importu.

\section{BPEJ}
\label{bpej}

\subsection{Systém BPEJ}
\label{system_bpej}

Bonitovaná půdně ekologická jednotka vyjadřuje produkční potenciál zemědělské půdy s ohledem na místo, kde se půda nachází. Systém \zk{BPEJ} vznikl mezi lety 1973 a~1980 na~základě Komplexního průzkumu zemědělských půd. Původně byl systém \zk{BPEJ} zamýšlen jako~podklad pro~plánování zemědělské produkce, ale~po~roce 1989 se začal používat i~pro~jiné učely. Z toho vyplývají některá jeho omezení a nedostatky.

Správcem a garantem údajů \zk{BPEJ} je Výzkumný ústav meliorací a půdy sídlící v Praze Zbraslavi.

Od roku 1998 jsou údaje \zk{BPEJ} vedeny v katastru nemovitostí a používají je další orgány státní správy. Číselné vyjádření ceny \zk{BPEJ} za~metr čtvereční slouží naříklad pro~výpočet daně z~nemovitostí, pro~stanovení úředních cen zemědělské půdy, nebo~pro~určení nároků v~ceně při~pozemkových úpravách.

Celostátní databáze \zk{BPEJ} je od~dubna 2017 veřejně dostupná \citep{databaze_bpej}. V době psaní tohoto dokumentu byly hranice \zk{BPEJ} k nahlížení v mapové aplikaci, nebo bylo data možné stáhnout ve~formátu shapefile.

\subsection{Kód BPEJ}
\label{kod_bpej}

Kód \zk{BPEJ} zahrnuje tyto vlivy:
	\begin{itemize}[leftmargin=1.5cm, noitemsep]
		\item vlastnosti klimatu
		\item druh půdy
		\item vlastnosti půdy
			\begin{itemize}[leftmargin=1cm, noitemsep]
				\item zrnitost
				\item obsah skeletu
				\item obsah organických částí
				\item hloubka půdy
			\end{itemize}
		\item sklonitost pozemku
		\item orientace pozemku
	\end{itemize}

Vlastnosti a~charakteristiky oblasti \zk{BPEJ} jsou vyjádřeny pětimístným kódem, například:

\begin{align*}
	1.23.45
\end{align*}

kde
\begin{tabbing}
\hspace{2em} \= \hspace{5em} \= \kill
	\> $1$	\> první číslice udává příslušnost do klimatického regionu \\
	\> $23$	\> druhá a~třetí číslice vyjadřují hlavní půdní jednotku \\
	\> $4$	\> čtvrtá číslice zahrnuje sklonitost a~expozici\\
	\> $5$	\> pátá číslice kombinuje obsah skeletu a~hloubku půdy
\end{tabbing}

V mapách se může vyskytnout zápis s pomlčkami místo teček, v~počítačovém zpracování se používá zápis bez dělících znaků.

Pro všechny nezemědělské nebo nebonitované plochy se od roku 2008 používá jednotný kód $99$~\citep{metodika_bpej}. Dříve se nezemědělské nebo nebonitované plochy označovaly označovaly pětimístným nebo zkráceným dvoumístným kódem, viz tabulka \ref{tab:pomocne_kody_bpej}.

\begin{table}[H]
    \begin{tabular}{|l|l|l|}
        \hline
         kategorie & pětimístný kód & dvoumístný kód \\
        \hline
        \hline
         haldy, navážka              	& 00026	& 26 \\ \hline
         ostatní neplodná půda			& 00029	& 29 \\ \hline
         intravilán                   	& 00030	& 30 \\ \hline
         lomy, těžební prostory       	& 00034	& 34 \\ \hline
         vodní plochy, toky           	& 00035	& 35 \\ \hline
         vojenské prostory				& 00070	& 70 \\ \hline
         nebonitovaná zemědělská půda	& 00099	& 99 \\
         \hline
    \end{tabular}
    \centering
    \caption[Pomocné kódy pro nebonitované plochy]{Pomocné kódy pro nebonitované plochy (zdroj:~\citep{metodika_bpej})}
    \label{tab:pomocne_kody_bpej}
\end{table}

\subsubsection{Klimatický region}
\label{klimaticky_region}

Klimatický region je území s přibližně stejnými klimatickými podmínkami pro růst a vývoj zemědělských plodin. V kódu~\zk{BPEJ} se uvádí jako první číslice.

Vymezení klimatických regionů pro účely systému \zk{BPEJ} bylo provedeno na základě údajů Českého hydrometeorologického ústavu z let 1901 až 1950. V úvahu se brala tato kritéria:
	\begin{itemize}[leftmargin=1.5cm, noitemsep]
		\item suma průměrných denních teplot nad $10^\circ$C
		\item průměrná roční teplota
		\item průměrný roční úhrn srážek
		\item pravděpodobnost suchých vegetačních období
		\item vláhová jistota
		\item doplňující hlediska
			\begin{itemize}[leftmargin=1cm, noitemsep]
				\item nadmořská výška
				\item expoziční ráz krajiny
				\item fénové jevy
				\item údaje místních literárních pramenů
				\item vztahy k dlouhodobým výnosovým řadám
			\end{itemize}
	\end{itemize}

V České republice je vymezeno 10 klimatických regionů označených kódy 0~až~9, od~nejteplejší po~nejchladnější~\citep{vyhlaska_327}. Rozmístění klimatických regionů je na obrázku \ref{fig:klimaticke_regiony}.

	\begin{figure}[H]
		\centering
		\includegraphics[width=.9\textwidth]{./pictures/klimaticky_region.png}
		\caption[Klimatické regiony]{Klimatické regiony (zdroj:~\citep{vumop_bpej})}
		\label{fig:klimaticke_regiony}
 	\end{figure}

\subsubsection{Hlavní půdní jednotka}
\label{hpj}

Hlavní půdní jednotka je definována jako účelové seskupení půdních forem s~příbuznými ekologickými a~agronomickými vlastnostmi. Je charakterizována genetickým půdním typem, subtypem, půdotvorným substrátem, hloubkou půdního profilu, zrnitostí a~stupněm hydromorfismu. V~systému \zk{BPEJ} se uvádí na druhém a~třetím místě číselného kódu.

V současné době systém \zk{BPEJ} vymezuje 78 hlavních půdních jednotek a~ty jsou dále seskupeny do 13 půdních typů. Obrázek~\ref{fig:klimaticke_regiony} znázorňuje rozložení půdních typů.

	\begin{figure}[H]
		\centering
		\includegraphics[width=.9\textwidth]{./pictures/pudni_typy.png}
		\caption[Půdní typy]{Půdní typy (zdroj:~\citep{vumop_bpej})}
		\label{fig:pudni_typy}
 	\end{figure}

\subsubsection{Sklonitost a expozice}
\label{sklonitost_expozice}

Sklonitost se rozděluje do~sedmi skupin. V~terénu se sklonitost určujě sklonoměrem, jako pomocný podklad lze využít mapy s~podrobným výškopisem.

Expozice vyjadřuje polohu území \zk{BPEJ} vůči světovým stranám. V klimatických regionech 0, 1, 2, 3, 4 a 5 se jižní expozice samostatně hodnotí jako negativní, zbývající expozice se slučují bez~rozlišení. Samostatně se severní expozice v~klimatických regionech 6, 7, 8, 9 uvažuje jako negativní, expozice východní, západní a~jižní se hodnotí jako sobě rovné. Expozice se dělí na čtyři kategorie.

Výsledná třetí číslice kódu \zk{BPEJ} vznikne kombinací sklonitosti a expozice~\citep{vyhlaska_327}.

\subsubsection{Obsah skeletu a hloubka půdy}
\label{hloubka_pudy_obsah_skeletu}

Obsah skeletu závisí na obsahu kamene (pevné částice nad 30 mm) a~štěrku (pevné částice hornin od 4 do 30 mm), je rozdělen do~čtyř kategorií.

Hloubka půdy je dána částí půdního profilu omezeného silnou skeletovostí, nebo pevnou horninou. Ve~vyhlášce~\citep{vyhlaska_327} jsou definovány tři kategorie hloubky půdy.

Na pátém místě číselného kódu \zk{BPEJ} se uvádí kód kombinace obsahu skeletu a hloubky půdy~\citep{vyhlaska_327}.

\chapter{Použité technologie}
\label{technologie}

\section{QGIS}
\label{qgis}

	\begin{figure}[H]
		\centering
		\includegraphics[width=.3\textwidth]{./pictures/qgis_logo.png}
      	\caption[logo QGIS]{logo QGIS (zdroj: \href{https://commons.wikimedia.org/wiki/File:QGis_Logo.png}{Wikimedia Commons})}
		\label{fig:qgis_logo}
 	\end{figure}

QGIS je open-source geografický informační systém (\zk{GIS}) distribuovaný pod ~licencí \textit{GNU General Public License}. Mezi~jeho velké výhody patří přenositelnost zdrojového kódu, je dostupný pro~platformy Windows, Linux, Unix, MacOS a~vyvíjí se i~mobilní verze pro~Android.

Vývoj programu, tehdy pod~názvem Quantum GIS, započal roku 2002, později projekt zaštítila organizace Open Source Geospatial Foundation (\zk{OSGeo}) a~verze~1.0 vyšla v~roce 2009. V~současné době jsou verze QGISu pojmenovávány podle~měst.

Systém QGIS nabízí možnost prohlížení, vytváření, editaci a~analýzu prostorových dat, tvorbu mapových výstupů, i~zpracování dat GPS. Podporuje velké množství vektorových, rastrových a~databázových formátů.

Samotný program je napsán v~jazyce C++ a~používá knihovnu Qt. Funkcionalitu programu je možné rozšířit pomocí zásuvných modulů, které mohou být vytvořeny v~jazyce C++ a~nebo~Python~\citep{qgis}~\citep{wiki_qgis}.

\section{Python}
\label{python}

	\begin{figure}[H]
		\centering
		\includegraphics[width=.5\textwidth]{./pictures/python_logo.png}
      	\caption[logo Python]{logo Python (zdroj:~\citep{python})}
		\label{fig:python_logo}
 	\end{figure}

Python je vysokoúrovňový objektově orientovaný programovací jazyk s~dynamickou kontrolou datových typů. Mezi~hlavní myšlenky jazyka Python patří důraz na~čitelnost, která je zajištěna povinným odsazováním datových bloků, a~jednoduchou syntaxi, díky~které jsou programátoři schopni zapsat své nápady na~méně řádcích než~vě~většině běžně používaných programovacích jazyků. Python je vyvíjen jako~open-source software a~nabízí instalační balíky pro~většinu platforem. Jedná se o~interpretovaný jazyk a~je vhodným nástrojem pro psaní skriptů i~rozsáhlých programů. Disponuje širokou nabídkou modulů pro řešení úloh téměr z jakékoli oblasti. Aktuálně se Python vyvíjí ve verzích 2.7.x a 3.x, ovšem v~roce 2020 bude podpora verze 2.7.x ukončena~\citep{python}~\citep{wiki_python}.

\section{SQLite}
\label{sqlite}

	\begin{figure}[H]
		\centering
		\includegraphics[width=.2\textwidth]{./pictures/sqlite_logo.png}
      	\caption[logo SQLite]{logo SQLite (zdroj: \href{https://commons.wikimedia.org/wiki/File:SQLite_Logo_4.png}{Wikimedia Commons})}
		\label{fig:sqlite_logo}
 	\end{figure}

SQLite je relační databázový systém šířený pod~licencí \textit{public domain}. Každá SQLite databáze je uložena v~samostatném souboru, který je nezávislý na~platformě. Na~rozdíl od~většiny databází, SQLite není implementována jako samostatný serverový proces, ale~čte a~zapisuje data přímo z~databázového souboru na~disku. Databáze SQLite nevyžaduje žádnou konfiguraci, částečně je ji možné nastavit příkazy \textit{PRAGMA}~\citep{sqlite}~\citep{wiki_sqlite}.

\section{PyQt}
\label{pyqt}

	\begin{figure}[H]
		\centering
		\includegraphics[width=.2\textwidth]{./pictures/pyqt_logo.png}
      	\caption[logo PyQt]{logo PyQt (zdroj: \href{https://commons.wikimedia.org/wiki/File:Python_and_Qt.svg}{Wikimedia Commons})}
		\label{fig:pyqt_logo}
 	\end{figure}

PyQt je modul, který umožňuje používat knihovnu Qt v~programovacím jazyce Python. Existují dvě verze modulu - PyQt4, která podporuje knihovnu Qt~4, a~PyQt5 pracující s~knihovnou Qt~5. Obě verze jsou vyvíjeny firmou Riverbank Computing. Modul PyQt je dostupný pro všechny platformy, které podporuje knihovna Qt, a~je distribuován pod licencí \textit{GNU GPL v3} nebo~\textit{Riverbank Commercial License}. Nejširší uplatnění nachází~při tvorbě grafického uživatelského prostředí, ale~obsahuje například i~třídy pro~řízení vláken, práci s~databází, parsování \textit{XML} souborů a~další~\citep{pyqt}~\citep{wiki_pyqt}.

\section{GDAL}
\label{gdal}

	\begin{figure}[H]
		\centering
		\includegraphics[width=.2\textwidth]{./pictures/gdal_logo.png}
      	\caption[logo GDAL]{logo GDAL (zdroj: \href{https://commons.wikimedia.org/wiki/File:GDALLogoColor.svg}{Wikimedia Commons})}
		\label{fig:gdal_logo}
 	\end{figure}

GDAL je knihovna pro čtení a~zápis rastrových i~vektorových \zk{GIS} formátů. Je vyvíjena pod záštitou organizace \zk{OSGeo} a vydávána pod licencí \textit{X/MIT}. Pro všechny podporované formáty používá jeden datový model. Samotná knihovna je napsána v programovacím jazyce C++ a obsahuje rozhraní i pro další jazyky~\citep{gdal}~\citep{wiki_gdal}.


\chapter{Zásuvný modul}
\label{plugin}

V~této kapitole je popsán samotný zásuvný modul. Pro~názornost
a~srozumitelnost jsou zde uvedeny důležité části kódu a~diagramy
znázorňující složitější algoritmy.

Kapitola se věnuje zejména technickému řešení a~jeho důvodům. Popis
toho, k~čemu jednotlivé prvky grafického uživatelského rozhraní
slouží, je obsahem uživatelského manuálu, viz
příloha~\ref{uzivatelsky_manual}.

	\begin{figure}[H] \centering
		\includegraphics[width=.1\textwidth]{./pictures/puplugin.png}
		\caption[Zásuvný modul~– ikona]{Zásuvný modul~– ikona (zdroj: autor)}
		\label{fig:ikona_pluginu}
 	\end{figure}

\section{Vývoj}
\label{vyvoj}

Vývoj projektu probíhal pomocí verzovacího systému
Git, zdrojový kód je dostupný
v~GitHub repositáři\footnote{\url{http://github.com/ctu-geoforall-lab-projects/dp-svoboda-2017}}.

Pro~vytvoření základní kostry pluginu byl použit zásuvný modul
\textit{Plugin Builder}, který je součástí oficiálního repositáře
programu QGIS\footnote{\label{oficialni_repositar_qgis}\url{http://plugins.qgis.org/}}. Postupem času
byly ovšem názvy tříd, modulů a~celá struktura zásuvného modulu
změněny.

K~testování a~ladění byly použity další zásuvné moduly \textit{Remote
Debug}, \textit{Plugin Reloader} a~\textit{ScriptRunner}, jež jsou rovněž
dostupné z oficiálního repositáře\footnoteref{oficialni_repositar_qgis}.

Během~vývoje zásuvného modulu bylo čerpáno z~literatury zabývající se
programem QGIS \citep{pyqgis_book}, programovacím jazykem Python
\citep{python3_oop_book}~\citep{dive_into_python} a~modulem PyQt
\citep{pyqt_book}.

\section{Grafické uživatelské rozhraní}
\label{gui}

Grafické uživatelské rozhraní zásuvného modulu je reprezentováno
jedním oknem třídy
\texttt{QDockWidget}\footnote{\url{http://pyqt.sourceforge.net/Docs/PyQt4/qdockwidget.html}},
jehož hlavní výhodou je možnost ukotvení do~samotného programu
QGIS. Díky tomu není nutné přepínat mezi~okny a~práce s~pluginem se
stává uživatelsky přívětivou.

Jelikož se zásuvný modul řídí podle~legislativy České republiky
a~používá vý\-měnný formát katastru nemovitostí, je grafické
uživatelské rozhraní v~českém jazyce.

\section{Načtení VFK souboru}
\label{nacteni_vfk}

Soubor \zk{VFK} obsahuje mnoho datových bloků, pro~pozemkové úpravy je
tím nej\-důležitějším vrstva parcel (\texttt{\zk{PAR}}).

\subsection{Algoritmus}
\label{nacteni_vfk_algoritmus}

Algoritmus pro načtení vrstvy parcel ze~souboru \zk{VFK} patří mezi
komplikovanější části zásuvného modulu a~je klíčový pro~správný chod
navazujících procesů.

První verze tohoto algoritmu byla inspirována \textit{VFK
Pluginem}\footnote{\url{\detokenize{http://github.com/ctu-geoforall-lab/qgis-vfk-plugin}}},
ale během vývoje bylo nezbytné algoritmus mnohokrát upravovat
a~ve~výsledku se dosti liší.

Pro čtení \zk{VFK} souborů používá QGIS knihovnu GDAL (viz kapitola
č.~\ref{podklady}), konkrétně se jedná o \zk{VFK}
Driver\footnote{\url{\detokenize{http://www.gdal.org/drv_vfk.html}}}. Ten
funguje tak, že při prvním čtení souboru vytvoří ve stejném adresáři,
ve kterém se nachází čtený \zk{VFK} soubor, SQLite databázi a~do~ní
naimportuje všechna data. Při~dalším čtení se již databáze nevytváří,
proto je čtení mnohonásobně rychlejší, viz
tab.~\ref{tab:nacteni_vfk_driver}. Pro porovnání rychlosti načtení byl
použit veřejně dostupný \zk{VFK} soubor~\citep{zdroj_vfk}.

%%% ML: uvest testovaci soubor (asi ten ze strane CUZK, je to tak?)
% OS: Ano, vsechno jsem testoval na testovacim souboru z CUZK.
% Zadal jsem i pana Bartu o data, ale bohuzel.
% Pridal jsem vetu, ve ktere to zminuju.
\begin{table}[H]
    \begin{tabular}{|l|l|} \hline načtení & čas [s] \\ \hline \hline
první & 6.516 \\ \hline opakované & 0.160 \\ \hline
    \end{tabular} \centering
    \caption[VFK Driver~– porovnání rychlosti načtení]{VFK
Driver~– porovnání rychlosti načtení (zdroj: autor)}
    \label{tab:nacteni_vfk_driver}
\end{table}

\zk{VFK} Driver ovšem umožňuje otevřít \zk{VFK} soubor pouze v~režimu
čtení a~to je pro~potřeby zpracování pozemkových úprav nedostatečné.

Pro~takové případy je knihovna GDAL vybavena SQLite
Driverem\footnote{\url{\detokenize{http://gdal.org/drv_sqlite.html}}},
který nabízí možnost zápisu do SQLite databáze. Aby byl SQLite driver
schopen rozpoznat a~přečíst geometrii, musí databáze obsahovat tabulky
\texttt{\detokenize{geometry_columns}}
a~\texttt{\detokenize{spatial-}}\\\texttt{\detokenize{_ref_sys}}. V~tabulce
\texttt{\detokenize{geometry_columns}} je uveden seznam tabulek, které
mají geo\-metrii, společně s~údaji jako název sloupce s~geometrií, typ
geometrie, souřadnicový systém a~další, viz
tab.~\ref{tab:geometry_columns}. Údaje o~souřadnicovém systému
odkazují na~tabulku \texttt{\detokenize{spatial_ref_sys}}, ve~které
jsou souřadnicové systémy definovány. Seznam sloupců tabulky
\texttt{\detokenize{spatial_ref_sys}} a~jejich datové typy popisuje
tab.~\ref{tab:spatial_ref_sys}.

\begin{table}[H]
    \begin{tabular}{|l|l|} \hline název sloupce & datový typ \\ \hline
\hline \texttt{\detokenize{F_TABLE_NAME}} & \texttt{varchar unique} \\
\hline \texttt{\detokenize{F_GEOMETRY_COLUMN}} & \texttt{varchar} \\
\hline \texttt{\detokenize{GEOMETRY_TYPE}} & \texttt{integer} \\
\hline \texttt{\detokenize{COORD_DIMENSION}} & \texttt{integer} \\
\hline \texttt{\detokenize{SRID}} & \texttt{integer} \\ \hline
\texttt{\detokenize{GEOMETRY_FORMAT}} & \texttt{varchar} \\ \hline
    \end{tabular} \centering
    \caption[Tabulka \texttt{geometry\textunderscore columns}~–
sloupce]{Tabulka \texttt{geometry\textunderscore columns}~– sloupce (zdroj: autor)}
    \label{tab:geometry_columns}
\end{table}

\begin{table}[H]
    \begin{tabular}{|l|l|} \hline název sloupce & datový typ \\ \hline
\hline \texttt{\detokenize{SRID}} & \texttt{integer unique} \\ \hline
\texttt{\detokenize{AUTH_NAME}} & \texttt{text} \\ \hline
\texttt{\detokenize{AUTH_SRID}} & \texttt{tex}t \\ \hline
\texttt{\detokenize{SRTEXT}} & \texttt{text} \\ \hline
    \end{tabular} \centering
    \caption[Tabulka \texttt{spatial\textunderscore ref\textunderscore
sys}~– sloupce]{Tabulka \texttt{spatial\textunderscore
ref\textunderscore sys}~– sloupce (zdroj: autor)}
    \label{tab:spatial_ref_sys}
\end{table}

%%% ML: vysvetlit, ze OGR je soucastni knihovny GDAL a je urcen pro cteni vektorovych dat
% OS: Pridal jsem to jako poznamku pod carou.
Během práce s~daty souboru \zk{VFK} bylo zjištěno, že OGR\footnote{Knihovna OGR
slouží pro práci s vektorovými daty. Je součástí knihovny GDAL.} poskytovatel
dat programu QGIS při~změně atributových hodnot nepoužíval
transakce. V~důsledku toho trvalo uložení změn extrémně dlouho. Proto
byla podána
žádost\footnote{\url{http://issues.qgis.org/issues/16216}}, chyba
byla opravena a~od verze 2.18.5 poskytovatel dat OGR transakce
využívá.

Algoritmus zásuvného modulu pro~načítání \zk{VFK} souboru zmíněný
problém zo\-hledňuje. Ve~verzi programu QGIS nižší než~2.18.5
naimportuje data do~databáze SpatiaLite\footnote{SpatiaLite je extenze
SQLite, která umožňuje ukládat geoprostorová data a~obsahuje mnoho
prostorových funkcí \citep{spatialite} \citep{wiki_spatialite}.}
a~dále s~ní pracuje. Poskytovatel dat SpatiaLite programu
QGIS totiž ukládá změny v~transakcích a~tudíž nemá problémy s~pomalým
zápisem. Pro~potřeby zásuvného modulu funkcionalita databáze SQLite
dostačuje, a~proto je převod do~SpatiaLite pouze dočasné řešení, které
zajišťuje použitelnost pluginu i~ve~starších verzích QGISu.

%%% ML: tuto ukazku klidne vyrad (rozhodnuti necham na Tebe), navic preteka stranku
% OS: Vymazano.

\subsubsection{Popis algoritmu}
\label{popis_algoritmu_nacteni_vfk}

Algoritmus načtení \zk{VFK} souboru funguje následovně.

Pokud v~adresáři, ve~kterém se nachází vstupní \zk{VFK} soubor,
neexistuje databáze SQLite se~stejným názvem, vytvoří se pomocí
\zk{VFK} Driveru knihovny GDAL (viz ukázka
kódu~\ref{vytvoreni_db_vfk_driver}).

{\scriptsize
\begin{lstlisting}[style=python, caption={Vytvoření SQLite databáze
pomocí VFK Driveru}, captionpos=b, label=vytvoreni_db_vfk_driver,
backgroundcolor = \color{light-gray}, numbers=left]
QgsApplication.registerOgrDrivers()

vfkDriver = ogr.GetDriverByName('VFK')
vfkDataSource = vfkDriver.Open(filePath)
\end{lstlisting}}

%%% ML: zde nebo jinde zminit, ze z tohoto duvodu plugin nepracuje s
%%% verejne dostupnymi VFK daty (http://services.cuzk.cz/vfk)
% OS: Veta pridana.
Poté algoritmus zkontroluje, zda je v~databázi tabulka
\texttt{\zk{PAR}}. Celý plugin pracuje téměř výhradně právě s~touto
tabulkou, proto když se v databázi nenachází, algoritmus se
ukončí a uživatel je na tento problém upozorněn. Veřejně dostupná \zk{VFK}
data\footnote{\url{http://services.cuzk.cz/vfk}} tabulku \texttt{\zk{PAR}}
neobsahují, proto je nelze použít jako vstup zásuvného modulu.

Následuje tvorba geometrie pro~tabulky \texttt{\zk{PAR}}, \texttt{\zk{SOBR}}
a~\texttt{\zk{SPOL}}.

V~dalším kroku se otevře databázové připojení. Databázovým dotazem se
zkontroluje přítomnost tabulek \texttt{\detokenize{geometry_columns}}
a~\texttt{\detokenize{spatial_ref_sys}}. Pokud to je nutné, pomocí SQL
dávky se obě tabulky vytvoří a~nahrají se do nich potřebné
údaje. Pro zapisování dat je potřeba přidat vlastní sloupce.
Dotazem se zjistí, zda sloupce v~databázi existují,
v~případě že ne, další SQL dávka zajistí jejich vytvoření. Seznam přidaných
sloupců a~jejich datové typy se nachází
v~tab.~\ref{tab:pridane_sloupce_par}.

\begin{table}[H]
    \begin{tabular}{|l|l|} \hline název sloupce & datový typ \\ \hline
\hline \texttt{\detokenize{PU_ID}} & \texttt{bigint} \\ \hline
\texttt{\detokenize{PU_KMENOVE_CISLO_PAR}} & \texttt{integer} \\
\hline \texttt{\detokenize{PU_PODDELENI_CISLA_PAR}} & \texttt{integer}
\\ \hline \texttt{\detokenize{PU_VYMERA_PARCELY}} & \texttt{integer}
\\ \hline \texttt{\detokenize{PU_VYMERA_PARCELY_ABS_ROZDIL}} &
\texttt{integer} \\ \hline
\texttt{\detokenize{PU_VYMERA_PARCELY_MEZNI_ODCHYLKA}} &
\texttt{integer} \\ \hline
\texttt{\detokenize{PU_VYMERA_PARCELY_MAX_KODCHB_KOD}} &
\texttt{integer} \\ \hline \texttt{\detokenize{PU_KATEGORIE}} &
\texttt{integer} \\ \hline \texttt{\detokenize{PU_VZDALENOST}} &
\texttt{integer} \\ \hline \texttt{\detokenize{PU_CENA}} &
\texttt{real} \\ \hline
\texttt{\detokenize{PU_BPEJ_BPEJCENA_VYMERA_CENA}} & \texttt{integer}
\\ \hline \texttt{\detokenize{PU_MERITKO_PODKLADU}} & \texttt{integer}
\\ \hline
    \end{tabular} \centering
    \caption[Tabulka \texttt{\zk{PAR}}~– přidané sloupce]{Tabulka
\texttt{\zk{PAR}}~– přidané sloupce (zdroj: autor)}
    \label{tab:pridane_sloupce_par}
\end{table}

Když je proces nahrávání spuštěn ve~verzi programu nižší než 2.18.5, tak
se všechna data z databáze SQLite naimportují do~databáze SpatiaLite.

Nakonec se tabulka parcel v závislosti na~verzi QGISu pomocí SQLite
nebo SpatiaLite Driveru nahraje do~programu QGIS jako platná vrstva.

%%% ML: zmin proc to tak je
% OS: Zmineno.
Celé načtení \zk{VFK} souboru je kvůli plynulosti spuštěno v~samostatném
vlákně. Pro~přehlednost celý popsaný proces ilustruje diagram
na~obr.~\ref{fig:diagram_nacitani_vfk}.

%%% ML: ta ukazka zminuje zakazani VFK driveru, nemas to v textu posano
% OS: Tento kod prinasi vic otazek nez odpovedi, proto jsem ho vymazal.
% Mas pravdu, ze ukazek kodu tam mam trochu moc, tak to trochu zredukuju.

	\begin{figure}[H] \centering
		\includegraphics[width=1.2\textwidth]{./pictures/nacitani_vfk_souboru.pdf}
		\caption[Načtení \zk{VFK} souboru~– diagram
algoritmu]{Načtení \zk{VFK} souboru~– diagram algoritmu (zdroj: autor)}
		\label{fig:diagram_nacitani_vfk}
 	\end{figure}

\subsection{Symbologie vrstvy \texttt{\zk{PAR}}}
\label{symbologie_par}

Symbologie nahrané vrstvy parcel je dána podle předem připraveného QML
souboru, ve~kterém jsou definovány barvy podle druhů
pozemků. V~tabulce~\texttt{\zk{PAR}} je informace o~druhu pozemku
uvedena ve~sloupci \texttt{\detokenize{DRUPOZ_KOD}} (viz
tab.~\ref{tab:par_sloupce}), kódy druhů pozemku s~názvy se nachází v
tab.~\ref{tab:druhy_pozemku}.

\begin{table}[H]
    \begin{tabular}{|l|l|} \hline kód & název \\ \hline \hline
\texttt{2} & orná půda \\ \hline \texttt{3} & chmelnice \\ \hline
\texttt{4} & vinice \\ \hline \texttt{5} & zahrada \\ \hline
\texttt{6} & ovocný sad \\ \hline \texttt{7} & trvalý travní porost \\
\hline \texttt{10} & lesní pozemek \\ \hline \texttt{11} & vodní
plocha \\ \hline \texttt{13} & zastavěná plocha a nádvoří \\ \hline
\texttt{14} & ostatní plocha \\ \hline
    \end{tabular} \centering
    \caption[Druhy pozemků]{Druhy pozemků (zdroj
\citep{vyhlaska_357})}
    \label{tab:druhy_pozemku}
\end{table}

\newpage

\subsection{Atributová tabulka vrstvy \texttt{\zk{PAR}}}
\label{tabulka_par}

Tabulka parcel sama o~sobě obsahuje mnoho sloupců. Společně
se~sloupci, které přidává zásuvný modul, se stává nepřehlednou, proto
plugin všechny nepotřebné sloupce v~atributové tabulce skrývá. Kvůli
větší srozumitelnosti pro~uživatele navíc zásuvný modul přidává sloupcům
aliasy, viz tab.~\ref{tab:viditelne_sloupce_aliasy_par}.

%%% ML: tabulka by mohla byt sirsi, potom by se text zbytecne nezalamoval
% OS: Upraveno.
\begin{table}[H]
    \begin{tabular}{|l|l|} \hline název sloupce & alias \\ \hline
\hline \texttt{\detokenize{KMENOVE_CISLO_PAR}} & \texttt{KMENOVE
C. (PUV.)} \\ \hline \texttt{\detokenize{PU_PODDELENI_CISLA_PAR}} &
\texttt{PODDELENI C. (EDI.)} \\ \hline
\texttt{\detokenize{PODDELENI_CISLA_PAR}} & \texttt{PODDELENI
C. (PUV.)} \\ \hline \texttt{\detokenize{PU_KMENOVE_CISLO_PAR}} &
\texttt{KMENOVE C. (EDI.)} \\ \hline
\texttt{\detokenize{PU_KATEGORIE}} & \texttt{KATEGORIE} \\ \hline
\texttt{\detokenize{VYMERA_PARCELY}} & \texttt{VYMERA (SPI)} \\ \hline
\texttt{\detokenize{PU_VYMERA_PARCELY}} & \texttt{VYMERA (SGI)} \\
\hline
          \begin{tabular}{@{}l@{}}
\texttt{\detokenize{PU_VYMERA_PARCELY}} \\
\texttt{\detokenize{_ABS_ROZDIL}} \end{tabular} & \texttt{ROZ. VYMER}
\\ \hline
          \begin{tabular}{@{}l@{}}
\texttt{\detokenize{PU_VYMERA_PARCELY}} \\
\texttt{\detokenize{_MEZNI_ODCHYLKA}} \end{tabular} &
\texttt{MEZ. ODCH. ROZ. VYMER} \\ \hline
\texttt{\detokenize{PU_VZDALENOST}} & \texttt{VZDALENOST} \\ \hline
\texttt{\detokenize{PU_CENA}} & \texttt{CELK. CENA} \\ \hline
          \begin{tabular}{@{}l@{}} \texttt{\detokenize{PU_BPEJ}} \\
\texttt{\detokenize{_BPEJCENA_VYMERA_CENA}} \end{tabular}
& \texttt{BPEJ KOD-CENA ZA M2-VYMERA-CENA} \\ \hline
\texttt{\detokenize{PU_MERITKO_PODKLADU}} & \texttt{MERITKO PODKL.} \\
\hline
    \end{tabular} \centering
    \caption[Vrstva \texttt{\zk{PAR}}~– viditelné sloupce
a~aliasy]{Vrstva \texttt{\zk{PAR}}~– viditelné sloupce a~aliasy (zdroj: autor)}
    \label{tab:viditelne_sloupce_aliasy_par}
\end{table}

\newpage

\section{Editace}
\label{editace}

Velmi důležitou činností během přípravné fáze pozemkových úprav je
určení obvodu pozemkové úpravy a~rozdělení parcel do~kategorií
(viz~\ref{obvod_a_predmet_pu}).

Algoritmus načtení \zk{VFK} souboru otvírá vrstvu parcel pomocí SQLite
Driveru\footnote{Ve~verzi programu QGIS nižší než~2.18.5 je vrstva
otevřená Spatialite Driverem, viz~\ref{nacteni_vfk_algoritmus}.},
takže ji lze editovat.

\subsection{Kategorie parcel}
\label{kategorie_parcel}

Kvůli zařazení parcel do~kategorií se během načítání přidává do~vrstvy
parcel sloupec \texttt{\detokenize{PU_KATEGORIE}} (alias
\texttt{KATEGORIE}), jehož datový typ je celé číslo
(\texttt{integer}). Zásuvný modul místo dlouhých názvů jednotlivých
kategorií používá číslice \texttt{0} až \texttt{2},
viz~tab~\ref{tab:kategorie_hodnoty}.

\begin{table}[H]
    \begin{tabular}{|l|l|} \hline hodnota & kategorie parcel \\ \hline
\hline \texttt{0} & mimo obvod \\ \hline \texttt{1} & v obvodu~–
neřešené \\ \hline \texttt{2} & v obvodu~– řešené \\ \hline
    \end{tabular} \centering
    \caption[Sloupec \texttt{PU\textunderscore KATEGORIE}~–
hodnoty]{Sloupec \texttt{PU\textunderscore KATEGORIE}~– hodnoty (zdroj: autor)}
    \label{tab:kategorie_hodnoty}
\end{table}

Plugin disponuje mechanismy pro~nastavení této hodnoty (viz ukázka
kódu~\ref{nastaveni_hodnoty_kategorie}) a~pro~výběr prvků v~kategorii
(viz ukázka kódu~\ref{vyber_v_kategorii}).

{\scriptsize
\begin{lstlisting}[style=python, caption={Kategorie parcel~– nastavení
hodnoty}, captionpos=b, label=nastaveni_hodnoty_kategorie,
backgroundcolor = \color{light-gray}, numbers=left]
fieldId = layer.fieldNameIndex('PU_KATEGORIE')

layer.startEditing() layer.updateFields()

for feature in features:
   if feature.attribute('PU_KATEGORIE') != value:
   id = feature.id()
layer.changeAttributeValue(id, fieldId, value)

layer.commitChanges()
\end{lstlisting}}

{\scriptsize
\begin{lstlisting}[style=python, caption={Kategorie parcel~– výběr
prvků v~kategorii}, captionpos=b, label=vyber_v_kategorii,
backgroundcolor = \color{light-gray}, numbers=left]
expression = QgsExpression("\"PU_KATEGORIE\" = {}".format(value))
features = layer.getFeatures(QgsFeatureRequest(expression))

ids = [feature.id() for feature in features]
layer.selectByIds(ids)
\end{lstlisting}}

\subsection{Vrstva obvodu}
\label{vrstva_obvodu}

Obvod pozemkové úpravy je území dotčené pozemkovými úpravami, patří
do~něj tedy parcely zařazené do~jednotlivých kategorií.

Bývá znázorňován tak, že všechny sousedící pozemky ve~stejné kategorii
tvoří pouze jeden prvek, u~kterého nejsou viditelné vnitřní
hranice. Tyto větší sloučené prvky je zvykem doplňovat o~popisky,
které značí, do~které kategorie náleží.

Zásuvný modul vytváří vrstvu obvodu na~základě sloupce
\texttt{\detokenize{PU_KATEGORIE}} ve~vrstvě parcel. Nejprve je
na~vrstvu \texttt{\zk{PAR}} zavolán nástroj \textit{Dissolve}
s~parametrem sloupce \texttt{\detokenize{PU_KATEGORIE}}. Díky tomu se
sousedící parcely v~jednotlivých kategoriích sloučí do~větších
celků. Program QGIS funguje tak, že prvky, které jsou tvořeny několika
body, liniemi, nebo polygony, mají pouze jeden popisek. Výstupem
nástroje \textit{Dissolve} mohou být i~multipolygony (prvky tvořeny
více polygony), proto je nutné zavolat funkci \textit{Multiparts
to~singleparts}, která problémové multipolygony rozdělí. Nakonec se
odstraní prvky, které mají nulovou hodnotu ve~sloupci
\texttt{\detokenize{PU_KATEGORIE}}, neboť takové prvky do vrstvy
obvodu nepatří. Ukázka kódu \ref{obvod_tvorba} obsahuje volání
nástrojů \textit{Dissolve} a~\textit{Multiparts to~singleparts}.

{\scriptsize
\begin{lstlisting}[style=python, caption={Vrstva obvodu~– tvorba},
captionpos=b, label=obvod_tvorba, backgroundcolor =
\color{light-gray}, numbers=left]
tempPerimeterLayerPath = processing.runalg('qgis:dissolve', layer,
                                           False, 'PU_KATEGORIE',
                                           None)['OUTPUT']
tempPerimeterLayer = QgsVectorLayer(tempPerimeterLayerPath,
                                    tempPerimeterLayerName, 'ogr')

processing.runalg('qgis:multiparttosingleparts', tempPerimeterLayer,
                  perimeterLayerFilePath)
perimeterLayer = QgsVectorLayer(perimeterLayerFilePath,
                                perimeterLayerName, 'ogr')
\end{lstlisting}}

\subsection{Symbologie vrstvy obvodu}
\label{symbologie_obvod}

Symbologie vrstvy obvodu se stejně jako u~vrstvy parcel řídí podle QML
souboru. V~popiscích vrstvy jsou hodnoty sloupce
\texttt{\detokenize{PU_KATEGORIE}}, jejich význam je popsán
v~tab.~\ref{tab:kategorie_hodnoty}. Popisky se zobrazují při jakémkoli
měřítku.

\subsection{Atributová tabulka vrstvy obvodu}
\label{tabulka_obvod}

Vrstva obvodu se vytváří z~vrstvy parcel, ovšem pouze informace
o~kategorii je pro~obvod relevantní. Z~toho důvodu je viditelný pouze
sloupec \texttt{\detokenize{PU_KATEGORIE}}, viz
tab.~\ref{tab:viditelne_sloupce_aliasy_obvod}.

\begin{table}[H]
    \begin{tabular}{|l|l|} \hline název sloupce & alias \\ \hline
\hline \texttt{\detokenize{PU_KATEGORIE}} & \texttt{KATEGORIE} \\
\hline
    \end{tabular} \centering
    \caption[Vrstva obvodu~– viditelné sloupce a~aliasy]{Vrstva
obvodu~– viditelné sloupce a~aliasy (zdroj: autor)}
    \label{tab:viditelne_sloupce_aliasy_obvod}
\end{table}

%%% ML: je nutne aby podkapitola zacinala vzdy na nove strance?, pokud
%%% to tak mas jinde v textu tak to nech
% OS: Nutne to neni, mazu to.

\section{Kontroly a analýzy}
\label{kontroly_analyzy}

Během přípravné fáze je nutné zkontrolovat soulad \zk{SPI} a~\zk{SGI}
dat katastru nemovitostí, ověřit správnost rozdělení parcel~do
kategorií a~také provést analýzy pro~sestavení vstupních soupisů
nároků vlastníků.

\subsection{Kontroly}
\label{kontroly}

\subsubsection{Kontrola~– obvodem}
\label{kontrola_obvodem}

Kontrola \textit{obvodem} slouží k~výběru parcel, které se nenachází
kompletně uvnitř vrstvy obvodu.

%%% ML: z textu neni uplne zrejme, ze jde o nastroje QGISu, alespon
%%% zde jsem to explicitne uvedl
Do~algoritmu vstupuje vrstva parcel a~vrstva obvodu. Použit je nástroj
systému QGIS \textit{Select by~location} s~geometrickým predikátem
\textit{within} a~poté je zavolána funkce pro~převrácení výběru prvků,
viz ukázka kódu~\ref{kontrola_obvodem_kod}.

{\scriptsize
\begin{lstlisting}[style=python, caption={Kontrola \textit{obvodem}~–
výběr prvků}, captionpos=b, label=kontrola_obvodem_kod,
backgroundcolor = \color{light-gray}, numbers=left]
processing.runalg('qgis:selectbylocation', layer, perimeterLayer,
                  u'within', 0, 0)

layer.invertSelection()
\end{lstlisting}}

\subsubsection{Kontrola~– není v SPI}
\label{kontrola_neni_v_spi}

Jak vyplývá z~názvu, kontrola \textit{není v~SPI} provádí výběr
parcel, které nejsou uvedeny v~souboru popisných informací. Pro výběr
používá sloupec \texttt{\detokenize{KMENOVE_CISLO_PAR}}, neboť patří
mezi povinně vyplněné \citep{struktura_vfk}. Pokud má parcela tento
sloupec prázdný, znamená to, že se jedná o~chybu nebo nově vytvořenou
parcelu.

{\scriptsize
\begin{lstlisting}[style=python, caption={Kontrola \textit{není
v~SPI}~– vzorec pro~výběr prvků}, captionpos=b,
label=kontrola_spi_kod, backgroundcolor = \color{light-gray},
numbers=left]
expression = QgsExpression("\"KMENOVE_CISLO_PAR\" is null")
\end{lstlisting}}

\subsubsection{Kontrola~– není v mapě}
\label{kontrola_neni_v_mape}

Výsledkem kontroly \textit{není v~mapě} je výběr parcel, které mají
prázdnou geometrii a~tudíž se nezobrazují v~mapovém okně.

{\scriptsize
\begin{lstlisting}[style=python, caption={Kontrola \textit{není
v~mapě}~– vzorec pro~výběr prvků}, captionpos=b,
label=kontrola_mapa_kod, backgroundcolor = \color{light-gray},
numbers=left]
expression = QgsExpression("$geometry is null")
\end{lstlisting}}

\subsubsection{Kontrola~– výměra nad mezní odchylkou}
\label{kontrola_vymera}

Kontrola \textit{výměra nad~mezní odchylkou} zjišťuje, jestli~rozdíl
mezi výměrou dle~souboru popisných informací a~výměrou danou souborem
geodetických informací překračuje mezní odchylku. Hodnota mezní
odchylky závisí na~kódu kvality nejméně přesně určeného lomového bodu
na~hranici parcely \citep{vyhlaska_357}, viz
tab.~\ref{tab:odchylky_vymer}. Pro~digitalizované parcely se kód
kvality podrobných bodů určí podle~měřítka podkladové mapy, viz
tab.~\ref{tab:kody_kvality_digit}.

Algoritmus z~vrstvy parcel nejdříve vyfiltruje prvky, které mají
validní geometrii a~zadanou výměru podle \zk{SPI}. Poté v~cyklu všemi
takovými prvky prochází. Pro~identifikaci parcel, které byly
digitalizované, slouží sloupec
\texttt{\detokenize{PU_MERITKO_PODKLADU}}. Hodnota \texttt{1} značí,
že parcela nemá validní geometrii, jiné číslo udává měřítko podkladové
mapy. Pokud je tedy v~tomto sloupci uvedeno číslo různé od~\texttt{1},
znamená to, že~se jedná o~digitalizovanou parcelu. V~takovém případě
algoritmus zjistí kód kvality podrobných bodů podle
%%% ML: co presne znaci nejvetsi kod kvality?
% OS: Opraveno na "nejvyšší".
tab.~\ref{tab:kody_kvality_digit}. Pomocí lomového bodu s~nejvyšším
kódem kva\-lity se vypočte mezní odchylka výměr a~porovná se
s~absolutním rozdílem výměr dle~\zk{SPI} a~\zk{SGI}. Když je mezní
odchylka překročena, přidá se parcela do~výběru. V~momentě, kdy už není
k~dispozici žádný další prvek, se kontrola ukončí. Celý postup
znázorňuje diagram na obr.~\ref{fig:diagram_vymera}.

\subsubsection{Kontrola~– bez vlastníka}
\label{kontrola_bez_vlastnika}

Kontrola \textit{bez~vlastníka} používá sloupec
\texttt{\detokenize{TEL_ID}} pro~výběr parcel, které jsou
bez~vlastníka, tzn. že~nemají přiřazený list vlastnictví, viz ukázka
kódu~\ref{kontrola_vlastnik_kod}. Takové parcely se označují jako
\textit{LV~0}.

{\scriptsize
\begin{lstlisting}[style=python, caption={Kontrola
\textit{bez~vlastníka}~– vzorec pro~výběr prvků}, captionpos=b,
label=kontrola_vlastnik_kod, backgroundcolor = \color{light-gray},
numbers=left]
expression = QgsExpression("\"TEL_ID\" is null")
\end{lstlisting}}

	\begin{figure}[H] \centering
		\includegraphics[width=1.2\textwidth]{./pictures/vymera.pdf}
		\caption[Kontrola \textit{výměra nad mezní
odchylkou}~– diagram algoritmu]{Kontrola \textit{výměra nad mezní
odchylkou}~– diagram algoritmu (zdroj: autor)}
		\label{fig:diagram_vymera}
 	\end{figure}

\subsection{Analýzy}
\label{analyzy}

\subsubsection{Analýza~– měření vzdálenosti}
\label{analyza_vzdalenosti}

Analýza \textit{měření vzdálenosti} určuje pro~všechny řešené parcely
vzdálenost jejich těžiště od~referenčního bodu, viz ukázka
kódu~\ref{analyza_vzdalenost_vypocet_vzdalenosti_teziste_od_ref_bodu}. Výsledné
zaokrouhlené hodnoty v~metrech ukládá do~sloupce
\texttt{\detokenize{PU_VZDALENOST}}

Do~kontroly kromě vrstvy parcel vstupuje i~vrstva referenčního bodu,
která musí obsahovat právě jeden prvek a~kvůli zamezení neočekávaných
výsledků musí mít stejný souřadnicový systém jako vrstva parcel.

{\scriptsize
\begin{lstlisting}[style=python, caption={Analýza \textit{měření
vzdálenosti}~– výpočet vzdálenosti těžiště\newline od~referenčního
bodu}, captionpos=b,
label=analyza_vzdalenost_vypocet_vzdalenosti_teziste_od_ref_bodu,
backgroundcolor = \color{light-gray}, numbers=left]
centroid = geometry.centroid().asPoint()
distanceDouble = sqrt(refPoint.sqrDist(centroid))
distance = int(round(distanceDouble))
\end{lstlisting}}

\subsubsection{Analýza~– oceňování podle BPEJ}
\label{analyza_bpej}

Analýza \textit{oceňování podle BPEJ} vypočítá cenu pozemku na~základě
vrstvy hranic \zk{BPEJ}.

Pro~určení ceny za~metr čtvereční jednotlivých kódů \zk{BPEJ} analýza
používá číselník \zk{BPEJ} z~Českého úřadu zeměměřičského a
katastrálního\footnote{Informace o~číselníku jsou dostupné
na~\url{http://goo.gl/uXf8FC}. Samotný číselník lze stáhnout
z~\url{http://www.cuzk.cz/CUZK/media/CiselnikyISKN/SC_BPEJ/SC_BPEJ.zip?ext=.zip}.}. Tento
číselník je aktualizován každý den kolem třetí hodiny ranní.

Do~algoritmu vstupují vrstvy \texttt{\zk{PAR}} a~hranice \zk{BPEJ},
na~které je volán nástroj systému QGIS vektorového překryvu \textit{Union}. Poté se
zkontroluje aktuálnost číselníku \zk{BPEJ}. Jestliže číselník není
aktuální a~lze se připojit k~internetu\footnote{Pro testování
internetového připojení byla zvolena adresa
%%% ML: je potreba testovat pripojeni zvlast? Nestacilo by se pokusit
%%% stahnout ciselnik a pripadne o tom informovat uzivatele? Ted to
%%% tak ale uz nech.
% OS: Testovani pripojeni k internetu jsem pridal kvuli tomu, ze by se mohlo
% zmenit URL pro stazeni ciselniku. Kdyby tam to testovani nebylo, tak by
% neslo rozlisit mezi tim, kdy se nelze pripojit k internetu a kdy se zmenilo
% URL ciselniku.
\url{http://www.google.com}.}, stáhne zásuvný modul nový
číselník. V~dalším kroku se z~nejnovějšího dostupného číselníku
přečtou data a~vypočítá se cena. Do~atributové tabulky se zapíše nejen
cena celková (sloupec \texttt{\detokenize{PU_CENA}}), ale~také cena
za~metr čtvereční, výměra a~cena dle jednotlivých bonit v~příslušné
parcele (sloupec
\texttt{\detokenize{PU_BPEJ_BPEJCENA_VYMERA_CENA}}). Může se stát, že
uživatel zvolí špatný sloupec, nebo že kód \zk{BPEJ} nebude uveden
v~číselníku. V~takovém případě plugin vybere ve~vrstvě obvodu prvky,
pro~které nenalezl ceny, a~informuje uživatele o~problému.

Algoritmus počítá i~s~možností změny adresy pro~stažení číselníku,
když tato situace nastane, oznámí to uživateli.

Princip algoritmu je znázorněn na obr.~\ref{fig:diagram_bpej}.

	\begin{figure}[H] %%\centering
		\includegraphics[width=1.2\textwidth]{./pictures/bpej.pdf}
		\caption[Analýza \textit{oceňování podle BPEJ}~–
diagram algoritmu]{Analýza \textit{oceňování podle BPEJ}~– diagram
algoritmu (zdroj: autor)}
		\label{fig:diagram_bpej}
 	\end{figure}

\chapter{Závěr}
\label{zaver}


%% zminka o podpore int64 v GDALu 2.2


% vysázení seznamu zkratek

\begin{seznamzkratek}{ABCDE}

	\novazkratka{GIS}
	      {GIS}
	      {Geografický informační systém}
	\novazkratka{PU}
	      {PÚ}
	      {pozemkové úpravy}
	\novazkratka{JPU}
	      {JPÚ}
	      {jednoduché pozemkové úpravy}
	\novazkratka{KoPU}
	      {KoPÚ}
	      {komplexní pozemkové úpravy}
	\novazkratka{ObPU}
	      {ObPÚ}
	      {obvod pozemkových úprav}
	\novazkratka{CAD}
	      {CAD}
	      {Computer Aided Design}
	\novazkratka{VFK}
	      {VFK}
	      {výměnný formát katastru}
	\novazkratka{VFP}
	      {VFP}
	      {výměnný formát pozemkových úprav}
	\novazkratka{DKM}
	      {DKM}
	      {digitální katastrální mapa}
	\novazkratka{SPI}
	      {SPI}
	      {soubor popisných informací}
	\novazkratka{SGI}
	      {SGI}
	      {soubor geodetických informací}
	\novazkratka{BPEJ}
	      {BPEJ}
	      {bonitovaná půdně ekologická jednotka}
	\novazkratka{VUMOP}
	      {VÚMOP}
	      {Výzkumný ústav meliorací a ochrany půdy}
	\novazkratka{DOSS}
	      {DOSS}
	      {dotčené orgány státní správy}
	\novazkratka{ISKN}
	      {ISKN}
	      {informační systém katastru nemovitostí}
	\novazkratka{KN}
	      {KN}
	      {katastr nemovitostí}
	\novazkratka{S-JTSK}
	      {S-JTSK}
	      {souřadnicový systém Jednotné trigonometrické sítě katastrální}
	\novazkratka{SVF}
	      {SVF}
	      {starý výměnný formát}
	\novazkratka{PAR}
	      {PAR}
	      {datový blok parcel souboru \zk{VFK}}
	\novazkratka{SOBR}
	      {SOBR}
	      {datový blok souřadnic obrazu bodů polohopisu v mapě souboru \zk{VFK}}
	\novazkratka{SPOL}
	      {SPOL}
	      {datový blok souřadnic polohy bodů polohopisu (měřených) souboru \zk{VFK}}
	\novazkratka{OK}
	      {OK}
	      {opravný koeficient výměr}
	\novazkratka{VUMOP}
	      {VÚMOP}
	      {Výzmuný ústav meliorací a půdy}
	      
\end{seznamzkratek}

% literatura
\nocite{*}
\def\refname{Literatura}
\bibliographystyle{mystyle}
\bibliography{literatura}


% začátek příloh
\def\figurename{Figure}%
\prilohy

% vysázení seznamu příloh
% \seznampriloh

% Vložení souboru s přílohami
\chapter{Struktura zásuvného modulu}
\label{struktura_pluginu}

\begin{minipage}{0.9\textwidth}
  \dirtree{%
  .1 /.
  .2 data/.
  .3 bpej/.
  .4 \detokenize{SC_BPEJ.csv}.
  .3 icons/.
  .4 checkanalysis.png.
  .4 edit.png.
  .4 loadvfk.png.
  .3 qml/.
  .4 PAR.qml.
  .4 perimeter.qml.
  .3 sql/.
  .4 \detokenize{add_pu_columns_PAR.sql}.
  .4 \detokenize{check_gc_srs.sql}.
  .4 \detokenize{check_pu_columns_PAR.sql}.
  .4 \detokenize{create_fill_gc_srs.sql}.
  .4 \detokenize{create_sobr_spol.sql}.
  .2 pubin/.
  .2 \detokenize{__init__.py}.
  .2 metadata.txt.
  .2 puplugin.cfg.
  .2 puplugin.png.
  .2 puplugin.py.
  .2 puplugin.svg.
  }
\end{minipage}

\begin{description}
	\item[\texttt{data}:] Složka obsahující všechna data.
	\begin{description}[leftmargin=1cm]
		\item[\texttt{bpej}:] Složka obsahující data pro~analýzu \textit{oceňování podle BPEJ}.
		\begin{description}[leftmargin=1cm]
			\item[\texttt{\detokenize{SC_BPEJ.csv}}:] Číselník \zk{BPEJ}.
		\end{description}
		\item[\texttt{icons}:] Složka obsahující ikony záložek.
		\begin{description}[leftmargin=1cm]
			\item[\texttt{checkanalysis.png}:] Ikona záložky \textit{Kontroly a analýzy}.
			\item[\texttt{edit.png}:] Ikona záložky \textit{Editace}.
			\item[\texttt{loadvfk.png}:] Ikona záložky \textit{Načtení VFK souboru}.
		\end{description}
		\item[\texttt{qml}:] Složka obsahující QML soubory.
		\begin{description}[leftmargin=1cm]
			\item[\texttt{PAR.qml}:] QML soubor pro~vrstvu \texttt{\zk{PAR}}.
			\item[\texttt{perimeter.qml}:] QML soubor pro~vrstvu obvodu.
		\end{description}
		\item[\texttt{sql}:] Složka obsahující SQL dávky.
		\begin{description}[leftmargin=1cm]
			\item[\texttt{\detokenize{add_pu_columns_PAR.sql}}:] SQL dávka pro přidání vlastních sloupců.
			\item[\texttt{\detokenize{check_gc_srs.sql}}:] SQL dávka pro kontrolu přítomnosti tabulek \texttt{\detokenize{geo-}}\newline\texttt{\detokenize{metry_columns}} a~\texttt{\detokenize{spatial_ref_sys}}.
			\item[\texttt{\detokenize{check_pu_columns_PAR.sql}}:] SQL dávka pro~kontrolu přítomnosti vlastních sloupců.
			\item[\texttt{\detokenize{create_fill_gc_srs.sql}}:] SQL dávka pro~vytvoření a~naplnění tabulek \texttt{\detokenize{geometry_columns}} a~\texttt{\detokenize{spatial_ref_sys}}.
			\item[\texttt{\detokenize{create_sobr_spol.sql}}:] SQL dávka pro~vytvoření tabulek \zk{SOBR} a~\zk{SPOL}.
		\end{description}
	\end{description}
	\item[\texttt{pubin}:] Složka vytvořeného Python balíčku, více viz příloha \ref{popis_python_balicku}.
	\item[\texttt{\detokenize{__init__.py}}:] Modul pro~inicializaci zásuvného modulu.
	\item[\texttt{metadata.txt}:] Soubor obsahující metadata o~zásuvném modulu.
	\item[\texttt{puplugin.cfg}:] Konfigurační soubor zásuvného modulu.
	\item[\texttt{puplugin.png}:] Ikona zásuvného modulu ve~formátu PNG.
	\item[\texttt{puplugin.py}:] Hlavní Python modul zásuvného modulu.
\end{description}

\chapter{Popis vytvořeného Python balíčku}
\label{popis_python_balicku}

Všechny třídy a~metody balíčku mají svůj vlastní \textit{docstring}, tedy komentář, ve~kterém je stručně napsáno, k~čemu třída či~metoda slouží, jaké má vstupní hodnoty, jaké vyvolává výjimky a~jaké~hodnoty vrací. Při~vytváření těchto komentářů bylo vycházeno z~\textit{Google Python Style Guide}\footnote{\url{https://google.github.io/styleguide/pyguide.html}}.

Plugin se bude dále vyvíjet, proto jsou zde popsány pouze základní informace, díky kterým je možné se v balíčku a modulech orientovat.

\bigskip

\begin{minipage}{0.9\textwidth}
  \dirtree{%
  .1 pubin/.
  .2 pustack/.
  .3 puca/.
  .4 \detokenize{__init__.py}.
  .4 \detokenize{area_pucawidget.py}.
  .4 \detokenize{bpej_pucawidget.py}.
  .4 \detokenize{distance_pucawidget.py}.
  .4 \detokenize{notinmap_pucawidget.py}.
  .4 \detokenize{notinspi_pucawidget.py}.
  .4 \detokenize{perimeter_pucawidget.py}.
  .4 pucawidget.py.
  .4 \detokenize{unowned_pucawidget.py}.
  .3 \detokenize{__init__.py}.
  .3 \detokenize{checkanalysis_puwidget.py}.
  .3 \detokenize{edit_puwidget.py}.
  .3 \detokenize{execute_thread.py}.
  .3 \detokenize{load_thread.py}.
  .3 \detokenize{loadvfk_puwidget.py}.
  .3 puwidget.py.
  .2 \detokenize{__init__.py}.
  .2 dockwidget.py.	
  .2 stackedwidget.py.
  .2 statusbar.py.
  .2 toolbar.py.
  }
\end{minipage}

\begin{description}
	\item[\texttt{pubin}:] Hlavní Python balíček, který obsahuje všechny vytvořené moduly.
	\begin{description}[leftmargin=1cm]
		\item[\texttt{pustack}:] Balíček obsahující moduly všech záložek a~jimi používaných tříd. Třídy záložek dědí z~abstraktní bázové třídy \texttt{PuWidget} nacházející se v~mo\-dulu \texttt{puwidget.py}.
		\begin{description}[leftmargin=1cm]
			\item[\texttt{puca}:] Balíček obsahující moduly záložky \textit{Kontroly a~analýzy}. Písmena \texttt{ca} jsou zkratkou pro~anglický název záložky~– \texttt{CheckAnalysis}. Všechny třídy kontrol a~analýz dědí z abstraktní bázové třídy \texttt{PuCaWidget} nacházející se v~modulu \texttt{pucawidget.py}. Pro spuštění kontroly nebo~analýzy slouží metoda \texttt{execute}.
			\begin{description}[leftmargin=1cm]
				\item[\texttt{\detokenize{__init__.py}}:] Modul pro~inicializaci balíčku.
				\item[\texttt{\detokenize{area_pucawidget.py}}:] Modul pro~kontrolu \textit{výměra nad~mezní odchylkou}.
				\item[\texttt{\detokenize{bpej_pucawidget.py}}:] Modul pro~analýzu \textit{oceňování podle BPEJ}.
				\item[\texttt{\detokenize{distance_pucawidget.py}}:] Modul pro~analýzu \textit{měření vzdálenosti}.
				\item[\texttt{\detokenize{notinmap_pucawidget.py}}:] Modul pro~kontrolu \textit{není v~mapě}.
				\item[\texttt{\detokenize{notinspi_pucawidget.py}}:] Modul pro~kontrolu \textit{není v~SPI}.
				\item[\texttt{\detokenize{perimeter_pucawidget.py}}:] Modul pro~kontrolu \textit{obvodem}.
				\item[\texttt{pucawidget.py}:] Abstraktní bázová třída, ze~které dědí všechny třídy kontrol a~analýz.
				\item[\texttt{\detokenize{unowned_pucawidget.py}}:] Modul pro~kontrolu \textit{bez~vlastníka}.
			\end{description}
			\item[\texttt{\detokenize{__init__.py}}:] Modul pro~inicializaci balíčku.
			\item[\texttt{\detokenize{checkanalysis_puwidget.py}}:] Modul pro~záložku \textit{Kontroly a~analýzy}.
			\item[\texttt{\detokenize{edit_puwidget.py}}:] Modul pro~záložku \textit{Editace}.
			\item[\texttt{\detokenize{execute_thread.py}}:] Modul pro~spouštění procesů editace, kontrol a~ana\-lýz v~samostatném vlákně.
			\item[\texttt{\detokenize{load_thread.py}}:] Modul pro~spouštění procesu načítání \zk{VFK} souboru v~samostatném vlákně.
			\item[\texttt{\detokenize{loadvfk_puwidget.py}}:] Modul pro~záložku \textit{Načtení VFK souboru}.
			\item[\texttt{puwidget.py}:] Abstraktní bázová třída, ze~které dědí všechny třídy zálo\-žek.
		\end{description}
		\item[\texttt{\detokenize{__init__.py}}:] Modul pro~inicializaci balíčku.
		\item[\texttt{dockwidget.py}:] Modul pro~hlavní grafickou komponentu pluginu.
		\item[\texttt{stackedwidget.py}:] Modul pro~grafickou komponentu, která obsahuje všechny záložky.
		\item[\texttt{statusbar.py}:] Modul pro~stavový řádek.
		\item[\texttt{toolbar.py}:] Modul pro~ikony na~přepínání mezi záložkami a~sadu standardních nástrojů programu QGIS.
	\end{description}
\end{description}

\chapter{Uživatelský manuál}
\label{uzivatelsky_manual}

\section{Instalace}
\label{manual_instalace}

Zásuvný modul není součástí oficiálního repositáře QGIS, přesto ho lze nainstalovat stejným způsobem jako~jiné pluginy. Stačí do~programu QGIS přidat repositář organizace GeoForAll Lab\footnote{\url{http://geomatics.fsv.cvut.cz/research/geoforall/}}.

Nejprve tedy otevřete okno \textit{Zásuvné moduly $\rightarrow$ Spravovat a~instalovat zásuvné moduly}.

	\begin{figure}[H]
		\centering
		\includegraphics[width=.6\textwidth]{./pictures/instalace-otevreni_okna_zasuvne_moduly.png}
		\caption[Otevření okna \textit{Zásuvné moduly}]{Otevření okna \textit{Zásuvné moduly} (zdroj: autor)}
		\label{fig:manual_otevreni_okna_zasuvne_moduly}
 	\end{figure}

V~záložce \textit{Nastavení} aktivujte volbu \textit{Zobrazit také experimentální zásuvné moduly}.

Pomocí tlačítka \textit{Přidat...} doplňte repositář organizace GeoForAll Lab:

\begin{lstlisting}[basicstyle=\footnotesize\ttfamily, backgroundcolor = \color{light-gray},  numbers=left, columns=fullflexible, keepspaces=true]
Název:   CVUT GeoForAll Lab
URL:     http://geo.fsv.cvut.cz/geoforall/qgis-plugins.xml
\end{lstlisting}

	\begin{figure}[H]
		\centering
		\includegraphics[width=.85\textwidth]{./pictures/instalace-pridani_repositare.png}
		\caption[Přidání repositáře]{Přidání repositáře (zdroj: autor)}
		\label{fig:manual_pridani_repozitare}
 	\end{figure}
 	
	\begin{figure}[H]
		\centering
		\includegraphics[width=.6\textwidth]{./pictures/instalace-pridani_repositare_geoforall.png}
		\caption[Přidání repositáře GeoForAll Lab]{Přidání repositáře GeoForAll Lab (zdroj: autor)}
		\label{fig:manual_pridani_repozitare_geoforall_lab}
 	\end{figure}

V záložce \textit{Vše} nebo \textit{Nenainstalované} vyhledejte \textit{PU Plugin}. Vyberte zásuvný modul a klikněte na \textit{Instalovat zásuvný modul}.

	\begin{figure}[H]
		\centering
		\includegraphics[width=.8\textwidth]{./pictures/instalace-instalace_zasuvneho_modulu.png}
		\caption[Zásuvný modul~– instalace]{Zásuvný modul~– instalace (zdroj: autor)}
		\label{fig:manual_instalace_puplugin}
 	\end{figure}

Po úspěšném nainstalování se v~\textit{Panelu nástrojů zásuvného modulu} objeví jeho ikona. Okno zásuvného modulu je možné vyvolat poklepáním na~jeho ikonu nebo volbou \textit{Zásuvné moduly $\rightarrow$ PU Plugin $\rightarrow$ PU Plugin}.

	\begin{figure}[H]
		\centering
		\includegraphics[width=.4\textwidth]{./pictures/instalace-toolbar.png}
		\caption[Ikona zásuvného modulu v panelu nástrojů]{Ikona zásuvného modulu v panelu nástrojů (zdroj: autor)}
		\label{fig:manual_ikona_v_panelu_nastroju}
 	\end{figure}

\section{Grafické uživatelské rozhraní}
\label{manual_gui}

	\begin{figure}[H]
		\centering
		\includegraphics[width=.55\textwidth]{./pictures/main_gui.png}
		\caption[Zásuvný modul~– grafické uživatelské rozhraní]{Zásuvný modul~– grafické uživatelské rozhraní (zdroj: autor)}
		\label{fig:manual_main_gui}
 	\end{figure}

\begin{description}
	\item[Prvek 1:] Skupina tří ikon pro~přepínání mezi záložkami:
	\begin{itemize}[leftmargin=1.5cm, noitemsep]
		\item \img{./pictures/loadvfk.png} \textit{Načtení VFK souboru}
		\item \img{./pictures/edit.png} \textit{Editace}
		\item \img{./pictures/checkanalysis.png} \textit{Kontroly a analýzy}
 	\end{itemize} 	
	\item[Prvek 2:] Skupina nástrojů, které jsou propojené se~standardními nástroji programu QGIS.
	\item[Prvek 3:] Okna záložek zobrazující se v~závislosti na~tom, která ze~tří ikon záložek (prvek~1) je aktivní.
	\item[Prvek 4:] Stavový řádek, ve~kterém se ukazují zprávy.
\end{description}

\newpage

\section{Komunikace s uživatelem}
\label{manual_komunikace}

Zásuvný modul komunikuje s~uživatelem třemi způsoby:

\begin{enumerate}[leftmargin=1.5cm, noitemsep]
	\item \underline{Stavový řádek} (viz prvek~4 obr.~\ref{fig:manual_main_gui}) představuje nejčastější způsob zobrazování zpráv zásuvného modulu. Když nevíte jak postupovat, zde s~největší pravděpodobností najdete potřebné informace. Běžné zprávy mají černou barvu písma, důležité zprávy se~zobrazují červeně (viz obr.~\ref{fig:manual_dulezita_zprava}).
	
	\begin{figure}[H]
		\centering
		\includegraphics[width=.23\textwidth]{./pictures/statusbar-red_message.png}
		\caption[Stavový řádek~– důležitá zpráva]{Stavový řádek~– důležitá zpráva (zdroj: autor)}
		\label{fig:manual_dulezita_zprava}
 	\end{figure}	

	\item \underline{Pole zpráv} je standardní způsob komunikace mezi programem QGIS a~uživatelem. Zobrazuje pole v~horní části mapového okna, které může být nastaveno tak, že po~určité době samo zmizí, nebo vyžaduje manuální zavření. Zásuvný modul využívá této komunikace pouze pro~zobrazení významných zpráv, které by neměly být uživatelem opomenuty (viz obr. \ref{fig:manual_zprava_pole_zprav}).

	\begin{figure}[H]
		\centering
		\includegraphics[width=.7\textwidth]{./pictures/message_bar-message.png}
		\caption[Pole zpráv~– zpráva upozornění]{Pole zpráv~– zpráva upozornění (zdroj: autor)}
		\label{fig:manual_zprava_pole_zprav}
 	\end{figure}

	\item \underline{Logování} je posledním prostředkem pro~předávání informací, který zásuvný modul používá. Informace v~anglickém jazyce, zejména chybové hlášky, zapisuje do~vlastní záložky s~názvem \textit{PU Plugin} (viz obr.~\ref{fig:manual_logovaci_panel}). Panel logovacích zpráv lze zobrazit kliknutím na~ikonu \img{./pictures/log.png} v~pravém dolním rohu QGISu.

	\begin{figure}[H]
		\centering
		\includegraphics[width=1.0\textwidth]{./pictures/log_panel.png}
		\caption[Panel logovacích zpráv]{Panel logovacích zpráv (zdroj: autor)}
		\label{fig:manual_logovaci_panel}
 	\end{figure}

\end{enumerate}

\newpage

\section{Načtení VFK souboru}
\label{manual_nacteni_vfk}

Záložka \textit{Načtení VFK souboru} slouží k~načtení vrstvy parcel ze~souboru~\zk{VFK}.

	\begin{figure}[H]
		\centering
		\includegraphics[width=.55\textwidth]{./pictures/nacteni_vfk_gui.png}
		\caption[Záložka \textit{Načtení VFK souboru}~– grafické uživatelské rozhraní]{Záložka \textit{Načtení VFK souboru}~– grafické uživatelské rozhraní (zdroj: autor)}
		\label{fig:manual_nacteni_vfk_gui}
 	\end{figure}

\begin{description}
	\item[Prvek 1:] Textové pole pro~cestu k~\zk{VFK} souboru.
	\item[Prvek 2:] Tlačítko pro~zobrazení dialogového okna pro~procházení adresářů. Filtruje soubory s~příponou \textit{*.vfk}, pamatuje si poslední použitou cestu.
	\item[Prvek 3:] Indikátor průběhu načítání \zk{VFK} souboru.
	\item[Prvek 4:] Tlačítko pro~načítání \zk{VFK} souboru. Aktivuje se pouze v~případě, že textové pole (prvek~1) obsahuje cestu k~existujícímu \zk{VFK} souboru.
\end{description}

\subsection{Postup}
\label{manual_nacteni_postup}

Nejprve je zapotřebí zvolit \zk{VFK} soubor, který chcete načíst. To lze udělat dvěma způsoby. Buď kliknete na~tlačítko \textit{Procházet} (prvek~2), vyberete požadovaný soubor a~cesta k~souboru se automaticky zapíše do~textového pole (prvek~1), nebo zkopírujete cestu k~\zk{VFK} souboru přímo do~zmíněného textové pole.

Když se v~textovém poli nachází cesta k~validnímu \zk{VFK} souboru, aktivuje se tlačítko \textit{Načíst} (prvek~4) a~můžete zahájit import.

O~průběhu načítání vás informuje indikátor průběhu (prvek~3) a~zprávy ve~stavovém řádku.

\subsection{Symbologie vrstvy parcel}
\label{manual_nacteni_symbologie}

Symbologie načtené vrstvy parcel se řídí podle druhu pozemku. Při měřítku 1:4000 a~větším přiblížení se zobrazí parcelní čísla.

	\begin{figure}[H]
		\centering
		\includegraphics[width=.9\textwidth]{./pictures/symbologie_par.png}
		\caption[Vrstva parcel~– symbologie]{Vrstva parcel~– symbologie (zdroj: autor)}
		\label{fig:manual_symbologie_par}
 	\end{figure}

\subsection{Atributová tabulka vrstvy parcel}
\label{manual_nacteni_tabulka}

Zásuvný modul v~atributové tabulce kvůli přehlednosti skrývá všechny nepotřebné sloupce. Pro~větší srozumitelnost mají viditelné sloupce aliasy.

	\begin{figure}[H]
		\centering
		\includegraphics[width=.95\textwidth]{./pictures/nacteni-tabulka.png}
		\caption[Vrstva parcel~– atributová tabulka]{Vrstva parcel~– atributová tabulka (zdroj: autor)}
		\label{fig:manual_tabulka_par}
 	\end{figure}

\newpage

\section{Editace}
\label{manual_editace}

Po úspěšném nahrání vrstvy parcel lze začít s~editací. Záložka \textit{Editace} poskytuje nástroje k~úpravě geometrie a~zařazení parcel do~kategorií.

	\begin{figure}[H]
		\centering
		\includegraphics[width=.55\textwidth]{./pictures/editace_gui.png}
		\caption[Záložka \textit{Editace}~– grafické uživatelské rozhraní]{Záložka \textit{Editace}~– grafické uživatelské rozhraní (zdroj: autor)}
		\label{fig:manual_editace_gui}
 	\end{figure}

\begin{description}
	\item[Prvek 1:] Skupina nástrojů pro~editaci, které jsou propojené se~standardními nástro\-ji programu QGIS.
	\item[Prvek 2:] Rozbalovací menu s~aktuálně načtenými polygonovými vrstvami.
	\item[Prvek 3:] Tlačítko pro~zobrazení dialogového, ve~kterém lze zvolit adresář a~název vrstvy obvodu. Filtruje soubory s~příponou \textit{*.shp}, pamatuje si poslední použitou cestu.
	\item[Prvek 4:] Rozbalovací menu s~kategoriemi parcel. Na~výběr jsou tyto kategorie, číslo v~závorce udává hodnotu, kterou zásuvný modul pro~kategorii používá:
	\begin{itemize}[leftmargin=1.5cm, noitemsep]
		\item \textit{mimo obvod (0)}
		\item \textit{v~obvodu~– neřešené (1)}
		\item \textit{v~obvodu~– řešené (2)}
		\item \textit{bez kategorie}
	\end{itemize}
	\item[Prvek 5:] Tlačítko pro~zobrazení (výběr) parcel v~aktuálně zvolené kategorii.
	\item[Prvek 6:] Rozbalovací menu s~variantami zařazení parcel. K~dispozici jsou dvě možnosti:
	\begin{itemize}[leftmargin=1.5cm, noitemsep]
		\item \textit{vybrané parcely}~– zařadí vybrané parcely do~aktuálně zvolené kategorie.
		\item \textit{obvodem}~– zařadí všechny parcely do~kategorií na~základě obvodu.
	\end{itemize}
	\item[Prvek 7:] Tlačítko pro~provedení zařazení.
\end{description}

\subsection{Postup}
\label{manual_editace_postup}

Zásuvný modul pracuje s~aktivní vrstvou, tj. vrstva vybraná v~panelu vrstev, který se ve~výchozím nastavení nachází na~levé straně okna.

Vrstvu parcel můžete editovat pomocí sady standardních nástrojů v~horní části pluginu (prvek~1).

Nejdůležitější funkcionalitou této záložky je ovšem zařazení parcel do~kategorií. Aby bylo na~první pohled zřejmé, ve~které kategorii jsou jednotlivé parcely zařazeny, používá zásuvný modul tzv.~vrstvu obvodu. Jedná se o~samostatnou vrstvu ve~formátu \textit{shapefile}. Adresář a~název této vrstvy můžete specifikovat pomocí tlačítka \textit{Vytvořit} (prvek~3). Po~poklepání na~zmíněné tlačítko se otevře dialogové okno, kde lze zvolit umístění vrstvy obvodu. Z~aktivní vrstvy, která musí být \zk{VFK}, se vytvoří vrstva obvodu, zásuvný modul ji načte a~vybere v~rozbalovacím menu (prvek~2).

Pokud cesta k~vrstvě obvodu není definována (rozbalovací menu je prázdné), nebo je v~rozbalovací menu vybrána vrstva, která nebyla vytvořena zásuvným modulem a~tudíž neobsahuje potřebné sloupce, plugin automaticky vytvoří vrstvu obvodu ve~stejném adresáři, ve~kterém se nachází aktivní vrstva parcel.

Funkce pro~vytvoření obvodu je volána v~momentě, kdy je pro~vrstvu parcel uložena změna geometrie, uložena změna, při~které došlo k~vymazání prvku, nebo je pomocí tlačítka \textit{Zařadit} (prvek~7) provedeno zařazení parcel.

Zásuvný modul nabízí dvě varianty zařazení parcel (prvek~6). První možností je volba \textit{vybrané parcely}, která provede zařazení vybraných parcel do~zvolené kategorie (prvek~4).

Druhý způsob nazvaný \textit{obvodem} rozřadí všechny parcely ve~\zk{VFK} vrstvě do~ka\-tegorií. Jako podklad použije aktuálně vybranou vrstvu obvodu (viz prvek~2). Tato varianta pracuje pouze s~obvody, které vytvořil zásuvný modul pro~pozemkové úpravy. Pro~zařazení do~kategorie musí být parcela kompletně uvnitř geometrie příslušného prvku obvodu.

Pro kontrolu nabízí zásuvný modul tlačítko \textit{Zobrazit} (prvek~5), které vybere, a~tím pádem zvýrazní, prvky v~kategorii.

Pokud vytvoříte novou parcelu, nebo~pomocí nástroje \textit{Přidat část} doplníte popis\-né údaje o~geometrii, vyplňte měřítko podkladů do~sloupce \textit{MERITKO PODKL.}. Tento údaj používá kontrola \textit{výměra nad mezní odchylkou}.

\subsection{Symbologie vrstvy obvodu}
\label{manual_editace_symbologie}

Pro symbologii vrstvy obvodu byla zvolena červená barva, popisky obsahují pouze číslo kategorie.

	\begin{figure}[H]
		\centering
		\includegraphics[width=1.0\textwidth]{./pictures/symbologie_obvod.png}
		\caption[Vrstva obvodu~– symbologie]{Vrstva obvodu~– symbologie (zdroj: autor)}
		\label{fig:manual_symbologie_obvod}
 	\end{figure}

\subsection{Atributová tabulka vrstvy obvodu}
\label{manual_editace_tabulka}

Vrstva obvodu se vytváří z~vrstvy parcel, ovšem pouze informace o kategorii je pro obvod relevantní. Z~toho důvodu je viditelný pouze sloupec \texttt{KATEGORIE}.

	\begin{figure}[H]
		\centering
		\includegraphics[width=.7\textwidth]{./pictures/editace-tabulka.png}
		\caption[Vrstva obvodu~– atributová tabulka]{Vrstva obvodu~– atributová tabulka (zdroj: autor)}
		\label{fig:manual_tabulka_obvod}
 	\end{figure}

\newpage

\section{Kontroly a analýzy}
\label{manual_kontroly_analyzy}

Poslední záložka zásuvného modulu nabízí možnost zkontrolovat data, zejména soulad mezi~\zk{SPI} a~\zk{SGI}, a~provést analýzy nezbytné pro~sestavení nárokových listů.

	\begin{figure}[H]
		\centering
		\includegraphics[width=.55\textwidth]{./pictures/ca_gui.png}
		\caption[Záložka \textit{Kontroly a analýzy}~– grafické uživatelské rozhraní]{Záložka \textit{Kontroly a analýzy}~– grafické uživatelské rozhraní (zdroj: autor)}
		\label{fig:manual_ca_gui}
 	\end{figure}

\begin{description}
	\item[Prvek 1:] Rozbalovací menu pro~přepínání mezi kontrolami a~analýzami.
	\item[Prvek 2:] Okna kontrol a~analýz zobrazující se v~závislosti na~tom, která položka rozbalovacího menu (prvek~1) je vybrána.
	\item[Prvek 3:] Tlačítko pro~provedení kontroly či~analýzy.
\end{description}

V~rozbalovacím menu (prvek~1) zvolte kontrolu či~analýzu, důsledkem čehož se změní dolní okno (prvek~2). Když je vše potřebné zadané, lze kontrolu či~analýzu spustit. Zprávy ve~stavovém řádku poskytují informace o~výsledku.

\subsubsection{Kontrola - obvodem}
\label{manual_kontrola_obvodem}

Kontrola obvodem provádí výběr parcel, které nejsou kompletně uvnitř vrstvy obvodu.

Jestliže od~začátku pracujete pouze s~jednou vrstvou obvodu, měl by být výsledek této kontroly stejný jako při~zvolení kategorie \textit{bez kategorie} (prvek~4 na~obr.~\ref{fig:manual_editace_gui}) a~provedení výběru prvků v~kategorii pomocí tlačítka \textit{Zobrazit} (prvek~5 na~obr.~\ref{fig:manual_editace_gui}). Lišit se tyto dvě metody budou v~momentě, kdy si do~QGISu nahrajete vrstvu obvodu, kterou jste vytvořili s~jinou vrstvou parcel. Jinými slovy tato kontrola používá geometrii vrstvy obvodu a~tlačítko \textit{Zobrazit} v~záložce \textit{Editace} vybírá parcely na~základě údajů uložených v~atributové tabulce.

Jediným potřebným vstupem je zmiňovaná vrstva obvodu v~rozbalovacím menu (viz obr.~\ref{fig:manual_kontrola_obvodem_gui}), které je propojené s~menu vrstvy obvodu v~záložce \textit{Editace}.

	\begin{figure}[H]
		\centering
		\includegraphics[width=.55\textwidth]{./pictures/kontrola-obvodem.png}
		\caption[Kontrola \textit{obvodem}~– grafické uživatelské rozhraní]{Kontrola \textit{obvodem}~– grafické uživatelské rozhraní (zdroj: autor)}
		\label{fig:manual_kontrola_obvodem_gui}
 	\end{figure}

\subsubsection{Kontrola - není v SPI}
\label{manual_kontrola_neni_v_spi}

Kontrola \textit{není v SPI} slouží k~zobrazení parcel, které nejsou v~souboru popisných informací.

\subsubsection{Kontrola - není v mapě}
\label{manual_kontrola_neni_v_mape}

Kontrola \textit{není v~mapě} vybírá parcely, které mají nulovou geometrii a~tudíž se nezobrazují v~mapovém okně.

\subsubsection{Kontrola - výměra nad mezní odchylkou}
\label{manual_kontrola_vymera}

Kontrola \textit{výměra nad~mezní odchylkou} ověřuje, zda~rozdíl mezi výměrou dle~\zk{SPI} a~výměrou vypočtenou z~\zk{SGI} nepřekračuje mezní odchylku. Ta je stanovena katastrální vyhláškou a~závisí na~kódu kvality nejméně přesně určeného lomového bodu na~hranici parcely. Jestliže je parcela digitalizovaná, kód kvality podrobných bodů se určí podle~měřítka podkladové mapy (viz sloupec \texttt{MERITKO PODKL.}).

\subsubsection{Kontrola - bez vlastníka}
\label{manual_kontrola_bez_vlastnika}

Kontrola \textit{bez~vlastníka} vybírá parcely, které jsou bez~vlastníka, tzn. že~nemají přiřazený list vlastnictví.

\subsubsection{Analýza - měření vzdálenosti}
\label{manual_analyza_vzdalenosti}

Analýza \textit{měření vzdálenosti} počítá pro~všechny řešené parcely vzdálenost jejich těžiště od~referenčního bodu. Výsledné zaokrouhlené hodnoty v~metrech ukládá do sloupce \texttt{VZDALENOST}.

Pro~spuštění této kontroly je zapotřebí v~rozbalovacím menu, které filtruje bodové vrstvy, zvolit vrstvu referenčního bodu, viz obr.~\ref{fig:manual_analyza_vzdalenosti_gui}. Vybraná vrstva referenčního bodu musí obsahovat právě jeden prvek a~musí mít stejný souřadnicový systém jako vrstva parcel.

	\begin{figure}[H]
		\centering
		\includegraphics[width=0.55\textwidth]{./pictures/analyza_vzdalenost.png}
		\caption[Analýza \textit{měření vzdálenosti}~– grafické uživatelské rozhraní]{Analýza \textit{měření vzdálenosti}~– grafické uživatelské rozhraní (zdroj: autor)}
		\label{fig:manual_analyza_vzdalenosti_gui}
 	\end{figure}

\subsubsection{Analýza - oceňování podle BPEJ}
\label{manual_analyza_bpej}

Analýza \textit{oceňování podle BPEJ} počítá cenu pozemku na~základě vrstvy hranic \zk{BPEJ}.

	\begin{figure}[H]
		\centering
		\includegraphics[width=0.55\textwidth]{./pictures/analyza_bpej.png}
		\caption[Analýza \textit{oceňování podle BPEJ}~– grafické uživatelské rozhraní]{Analýza \textit{oceňování podle BPEJ}~– grafické uživatelské rozhraní (zdroj: autor)}
		\label{fig:manual_analyza_bpej_gui}
 	\end{figure}

\begin{description}
	\item[Prvek 1:] Rozbalovací menu s~aktuálně načtenými polygonovými vrstvami.
	\item[Prvek 2:] Rozbalovací menu se~sloupci vybrané vrstvy \zk{BPEJ}.
\end{description}

Vyberte vrstvu hranic \zk{BPEJ} (prvek~1) a~poté zvolte sloupec, ve~kterém jsou uloženy kódy \zk{BPEJ}. Vrstva hranic \zk{BPEJ} musí mít stejný souřadnicový systém jako vrstva parcel.

Pro určení ceny za~metr čtvereční jednotlivých kódů \zk{BPEJ} analýza používá číselník \zk{BPEJ} z~Českého úřadu zeměměřičského a~katastrálního.

Do~atributové tabulky se zapíše nejen cena celková (sloupec \texttt{CELK. CENA}), ale~také cena za~metr čtvereční, výměra a~cena dle jednotlivých bonit v~příslušné parcele (sloupec \texttt{BPEJ KOD-CENA ZA M2-VYMERA-CENA}).

Pokud omylem zvolíte špatný slou\-pec, nebo když kód \zk{BPEJ} není nalezen v~číselníku, zásuvný modul vybere ve~vrstvě obvodu prvky, pro~které nenalezl ceny, a~informuje vás o~problému.


% konec dokumentu
\end{document}
