\chapter{Zásuvný modul}
\label{plugin}

V~této kapitole je popsán samotný zásuvný modul (plugin). Pro~názornost a~srozumitelnost jsou zde uvedeny důležité části kódu a~diagramy znázorňující složitější algoritmy.

Kapitola se věnuje zejména technickému řešení a~jeho důvodům. Popis toho, k~čemu jednotlivé prvky grafického uživatelského rozhraní slouží, je obsahem uživatelského manuálu, viz příloha~\ref{uzivatelsky_manual}.

	\begin{figure}[H]
		\centering
		\includegraphics[width=.1\textwidth]{./pictures/puplugin.png}
		\caption[Zásuvný modul - ikona]{Zásuvný modul - ikona}
		\label{fig:ikona_pluginu}
 	\end{figure}

\section{Vývoj}
\label{vyvoj}

Vývoj projektu probíhal pomocí verzovacího systému Git\footnote{\url{https://git-scm.com/}}, zdrojový kód je dostupný na~serveru GitHub\footnote{\url{https://github.com/ctu-geoforall-lab-projects/dp-svoboda-2017}}.

Pro~vytvoření základní kostry pluginu byl použit zásuvný modul \textit{Plugin Builder}\footnote{\url{https://github.com/g-sherman/Qgis-Plugin-Builder}}, který je součástí oficiálního repozitáře programu QGIS. Postupem času byly ovšem názvy tříd, modulů a celá struktura zásuvného modulu změněny.

K~testování a~ladění byly použity další zásuvné moduly \textit{Remote Debug}\footnote{\url{https://github.com/sourcepole/qgis-remote-debug}}, \textit{Plugin Reloader}\footnote{\url{https://github.com/borysiasty/plugin_reloader}} a~\textit{Script Runner}\footnote{\url{https://github.com/g-sherman/Script-Runner}}.

Během~vývoje zásuvného modulu bylo čerpáno z~literatury zabývající se programem QGIS \citep{qgis_book}~\citep{pyqgis_book}, programovacím jazykem Python \citep{python3_oop_book}~\citep{dive_into_python} a~modulem PyQt \citep{pyqt_book}.

\section{Grafické uživatelské rozhraní}
\label{gui}

Grafické uživatelské rozhraní zásuvného modulu je reprezentováno jedním oknem třídy \textit{QDockWidget}\footnote{\url{http://pyqt.sourceforge.net/Docs/PyQt4/qdockwidget.html}}, jehož hlavní výhodou je možnost ukotvení do~samotného programu QGIS. Díky tomu není nutné přepínat mezi~okny a~práce s~pluginem se stává uživatelsky přívětivou. 

Protože zásuvný modul se řídí podle~legislativy České republiky a~používá výměnný formát katastru nemovitostí, je grafické uživatelské rozhraní v~českém jazyce. 

\section{Načtení VFK souboru}
\label{nacteni_vfk}

Soubor \zk{VFK} obsahuje mnoho datových bloků, pro~pozemkové úpravy je tím nejdůležitějším vrstva parcel (\texttt{\zk{PAR}}).

\subsection{Algoritmus}
\label{nacteni_vfk_algoritmus}

Algoritmus pro načtení vrstvy parcel ze souboru \zk{VFK} patří mezi komplikovanější části zásuvného modulu a~je klíčový pro~správný chod navazujících procesů.

První verze tohoto algoritmu byla inspirována \zk{VFK} Pluginem\footnote{\url{\detokenize{https://github.com/ctu-geoforall-lab/qgis-vfk-plugin}}}, ale během vývoje bylo nezbytné algoritmus mnohokrát upravovat a~ve~výsledku se dosti liší.

Pro čtení \zk{VFK} souborů používá QGIS knihovnu GDAL (viz kapitola č.~\ref{podklady}), konkrétně se jedná o \zk{VFK} Driver\footnote{\url{\detokenize{http://www.gdal.org/drv_vfk.html}}}. Ten funguje tak, že při prvním čtení souboru vytvoří ve stejném adresáři, ve kterém se nachází čtený \zk{VFK} soubor, SQLite databázi a~do~ní naimportuje všechna data. Při~dalším čtení se již databáze nevytváří, proto je čtení mnohonásobně rychlejší, viz tab.~\ref{tab:nacteni_vfk_driver}. 

\begin{table}[H]
    \begin{tabular}{|l|l|}
        \hline
         načtení & čas [s] \\
        \hline
        \hline
         první & 6.516 \\ \hline
         opakované & 0.160 \\
         \hline
    \end{tabular}
    \centering
    \caption[Porovnání rychlosti načtení VFK Driverem]{Porovnání rychlosti načtení VFK Driverem}
    \label{tab:nacteni_vfk_driver}
\end{table}

\zk{VFK} Driver ovšem umožňuje otevřít \zk{VFK} soubor pouze v~režimu čtení a~to je pro~potřeby zpracování pozemkových úprav nedostatečné.

Pro~takové případy je knihovna GDAL vybavena SQLite Driverem\footnote{\url{\detokenize{http://gdal.org/drv_sqlite.html}}}, který nabízí možnost zápisu do SQLite databáze. Aby byl SQLite driver schopen rozpoznat a~přečíst geometrii, musí databáze obsahovat tabulky \texttt{\detokenize{geometry_columns}} a~\texttt{\detokenize{spatial_ref_sys}}. V~tabulce \texttt{\detokenize{geometry_columns}} je uveden seznam tabulek, které mají geometrii, společně s~údaji jako název sloupce s~geometrií, typ geometrie, souřadnicový systém a~další, viz tab. \ref{tab:geometry_columns}. Údaje o~souřadnicovém systému odkazují na~tabulku \texttt{\detokenize{spatial_ref_sys}}, ve~které jsou souřadnicové systémy definovány. Seznam sloupců tabulky \texttt{\detokenize{spatial_ref_sys}} a~jejich datové typy popisuje tab. \ref{tab:spatial_ref_sys}. 

\begin{table}[H]
    \begin{tabular}{|l|l|}
        \hline
         název sloupce & datový typ \\
        \hline
        \hline
         \texttt{\detokenize{F_TABLE_NAME}} & varchar unique \\ \hline
         \texttt{\detokenize{F_GEOMETRY_COLUMN}} & varchar \\ \hline
         \texttt{\detokenize{GEOMETRY_TYPE}} & integer \\ \hline
         \texttt{\detokenize{COORD_DIMENSION}} & integer \\ \hline
         \texttt{\detokenize{SRID}} & integer \\ \hline
         \texttt{\detokenize{GEOMETRY_FORMAT}} & varchar \\
         \hline
    \end{tabular}
    \centering
    \caption[Tabulka \texttt{geometry\textunderscore columns} - sloupce]{Tabulka \texttt{geometry\textunderscore columns} - sloupce}
    \label{tab:geometry_columns}
\end{table}

\begin{table}[H]
    \begin{tabular}{|l|l|}
        \hline
         název sloupce & datový typ \\
        \hline
        \hline
         \texttt{\detokenize{SRID}} & integer unique \\ \hline
         \texttt{\detokenize{AUTH_NAME}} & text \\ \hline
         \texttt{\detokenize{AUTH_SRID}} & text \\ \hline
         \texttt{\detokenize{SRTEXT}} & text \\
         \hline
    \end{tabular}
    \centering
    \caption[Tabulka \texttt{spatial\textunderscore ref\textunderscore sys} - sloupce]{Tabulka \texttt{spatial\textunderscore ref\textunderscore sys} - sloupce}
    \label{tab:spatial_ref_sys}
\end{table}

Během práce s~daty souboru \zk{VFK} bylo zjištěno, že OGR poskytovatel dat programu QGIS při~změně atributových hodnot nepoužíval transakce. V~důsledku toho trvalo uložení změn extrémně dlouho. Proto byla podána žádost\footnote{\url{https://issues.qgis.org/issues/16216}}, chyba byla opravena a~od verze 2.18.5 poskytovatel dat OGR transakce využívá.

Algoritmus zásuvného modulu pro~načítání \zk{VFK} souboru zmíněný problém zohledňuje. Ve~verzi programu QGIS nižší než~2.18.5 naimportuje data do~databáze SpatiaLite\footnote{SpatiaLite je extenze SQLite, která umožňuje ukládat geoprostorová data a~obsahuje mnoho prostorových funkcí \citep{spatialite} \citep{wiki_spatialite}.} a~dále s~ní pracuje (viz~\ref{kontrola_verze_qgis}). Poskytovatel dat SpatiaLite programu QGIS totiž ukládá změny v~transakcích a~tudíž nemá problémy s~pomalým zápisem. Pro~potřeby zásuvného modulu funkcionalita databáze SQLite dostačuje, a~proto je převod do~SpatiaLite pouze dočasné řešení, které zajišťuje použitelnost pluginu i~ve~starších verzích QGISu.

{\scriptsize
\begin{lstlisting}[style=python, caption={Kontrola verze programu QGIS}, captionpos=b, label=kontrola_verze_qgis, backgroundcolor = \color{light-gray},  numbers=left]
if QGis.QGIS_VERSION < '2.18.5':
    self.fixedSqliteDriver = False
else:
    self.fixedSqliteDriver = True
\end{lstlisting}}

\subsubsection{Popis algoritmu}
\label{popis_algoritmu_nacteni_vfk}

Algoritmus načtení \zk{VFK} souboru funguje následovně.

Pokud v~adresáři, ve~kterém se nachází vstupní \zk{VFK} soubor, neexistuje databáze SQLite se~stejným názvem, vytvoří se pomocí \zk{VFK} Driveru knihovny GDAL (viz~\ref{vytvoreni_db_vfk_driver}).

{\scriptsize
\begin{lstlisting}[style=python, caption={Vytvoření SQLite databáze pomocí VFK Driveru}, captionpos=b, label=vytvoreni_db_vfk_driver, backgroundcolor = \color{light-gray},  numbers=left]
QgsApplication.registerOgrDrivers()

vfkDriver = ogr.GetDriverByName('VFK')
vfkDataSource = vfkDriver.Open(filePath)
\end{lstlisting}}

Poté algoritmus zkontroluje, zda je v~databázi tabulka \texttt{\zk{PAR}}. Celý plugin pracuje téměř výhradně právě s~touto tabulkou, proto když ji databáze neobsahuje, algoritmus se ukončí. Uživatele je na tento problém upozorněn. Následuje tvorba geometrie pro~tabulky \texttt{\zk{PAR}}, \texttt{\zk{SOBR}} a \texttt{\zk{SPOL}}.

V~dalším kroku se otevře databázové připojení. Databázovým dotazem se zkontroluje přítomnost tabulek \texttt{\detokenize{geometry_columns}} a~\texttt{\detokenize{spatial_ref_sys}}. Pokud to je nutné, pomocí SQL dávky se obě tabulky vytvoří a~nahrají se do nich potřebné údaje. Při~editaci a~navazujících kontrolách a~analýzách zásuvný modul do~tabulky parcel zapisuje vlastní data, proto je zapotřebí přidat vlastní sloupce. Dotazem se zjistí, jestli již v~databázi existují, když ne, další SQL dávka zajistí jejich vytvoření. Seznam přidaných sloupců a~jejich datové typy se nachází v~tab.~\ref{tab:pridane_sloupce_par}.

\begin{table}[H]
    \begin{tabular}{|l|l|}
        \hline
         název sloupce & datový typ \\
        \hline
        \hline
         \texttt{\detokenize{PU_KMENOVE_CISLO_PAR}} & \texttt{integer} \\ \hline
         \texttt{\detokenize{PU_PODDELENI_CISLA_PAR}} & \texttt{integer} \\ \hline
         \texttt{\detokenize{PU_VYMERA_PARCELY}} & \texttt{integer} \\ \hline
         \texttt{\detokenize{PU_VYMERA_PARCELY_ABS_ROZDIL}} & \texttt{integer} \\ \hline
         \texttt{\detokenize{PU_VYMERA_PARCELY_MEZNI_ODCHYLKA}} & \texttt{integer} \\ \hline
         \texttt{\detokenize{PU_VYMERA_PARCELY_MAX_KODCHB_KOD}} & \texttt{integer} \\ \hline
         \texttt{\detokenize{PU_KATEGORIE}} & \texttt{integer} \\ \hline
         \texttt{\detokenize{PU_VZDALENOST}} & \texttt{integer} \\ \hline
         \texttt{\detokenize{PU_CENA}} & \texttt{real} \\ \hline
         \texttt{\detokenize{PU_BPEJ_BPEJCENA_VYMERA_CENA}} & \texttt{integer} \\ \hline
         \texttt{\detokenize{PU_MERITKO_PODKLADU}} & \texttt{integer} \\
         \hline
    \end{tabular}
    \centering
    \caption[Tabulka \texttt{\zk{PAR}} - přidané sloupce]{Tabulka \texttt{\zk{PAR}} - přidané sloupce}
    \label{tab:pridane_sloupce_par}
\end{table}

Když je proces nahrávání spuštěn ve verzi programu nižší než 2.18.5, všechna data databáze SQLite se naimportují do databáze SpatiaLite.

Nakonec se tabulka parcel v závislosti na verzi QGISu pomocí SQLite (viz \ref{vytvoreni_vrstvy_parcel_pomoci_sqlite}) nebo SpatiaLite Driveru nahraje do~programu QGIS jako platná vrstva.

Celé načtení \zk{VFK} souboru je spuštěno v~samostatném vlákně. Pro~přehlednost celý popsaný proces ilustruje diagram na~obr.~\ref{fig:diagram_nacitani_vfk}.

{\scriptsize
\begin{lstlisting}[style=python, caption={Vytvoření QGIS vrstvy parcel pomocí SQLite Driveru}, captionpos=b, label=vytvoreni_vrstvy_parcel_pomoci_sqlite, backgroundcolor = \color{light-gray},  numbers=left]
blacklistedDriver = ogr.GetDriverByName('VFK')
blacklistedDriver.Deregister()

composedURI = dbPath + '|layername=PAR'
layer = QgsVectorLayer(composedURI, layerName, 'ogr')
	
blacklistedDriver.Register()
\end{lstlisting}}

%% TODO - zmenit pismo tabulek
	\begin{figure}[H]
		\centering
		\includegraphics[width=1.2\textwidth]{./pictures/nacitani_vfk_souboru.pdf}
		\caption[Načítání VFK souboru - diagram algoritmu]{Načítání VFK souboru - diagram algoritmu}
		\label{fig:diagram_nacitani_vfk}
 	\end{figure}

\subsection{Symbologie vrstvy \texttt{\zk{PAR}}}
\label{symbologie_par}

Symbologie nahrané vrstvy parcel je dána podle předem připraveného QML souboru, ve~kterém jsou definovány barvy podle druhů pozemků. V~tabulce \texttt{\zk{PAR}} je informace o~druhu pozemku uvedena ve~sloupci \texttt{\detokenize{DRUPOZ_KOD}} (viz tab.~\ref{tab:par_sloupce}), kódy druhů pozemku s~názvy se nachází v tab.~\ref{tab:druhy_pozemku}.

\begin{table}[H]
    \begin{tabular}{|l|l|}
        \hline
         kód & název \\
        \hline
        \hline
          \texttt{2} & orná půda \\ \hline
          \texttt{3} & chmelnice \\ \hline          
          \texttt{4} & vinice \\ \hline
          \texttt{5} & zahrada \\ \hline
          \texttt{6} & ovocný sad \\ \hline
          \texttt{7} & trvalý travní porost \\ \hline
          \texttt{10} & lesní pozemek \\ \hline
          \texttt{11} & vodní plocha \\ \hline
          \texttt{13} & zastavěná plocha a nádvoří \\ \hline
          \texttt{14} & ostatní plocha \\
         \hline
    \end{tabular}
    \centering
    \caption[Druhy pozemků]{Druhy pozemků (zdroj \citep{vyhlaska_357})}
    \label{tab:druhy_pozemku}
\end{table}

\newpage

\subsection{Atributová tabulka vrstvy \texttt{\zk{PAR}}}
\label{tabulka_par}

Tabulka parcel sama o~sobě obsahuje mnoho sloupců. Společně se~sloupci, které přidává zásuvný modul, se atributová tabulka stává nepřehlednou, proto plugin všechny nepotřebné sloupce v~atributové tabulce skrývá. Kvůli větší srozumitelnost pro~uživatele navíc přidává sloupcům aliasy, viz tab.~\ref{tab:viditelne_sloupce_aliasy_par}.

\begin{table}[H]
    \begin{tabular}{|l|l|}
        \hline
         název sloupce & alias \\
        \hline
        \hline
          \texttt{\detokenize{KMENOVE_CISLO_PAR}} & KMENOVE C. (PUV.) \\ \hline
          \texttt{\detokenize{PU_PODDELENI_CISLA_PAR}} & PODDELENI C. (EDI.) \\ \hline
          \texttt{\detokenize{PODDELENI_CISLA_PAR}} & PODDELENI C. (PUV.) \\ \hline
          \texttt{\detokenize{PU_KMENOVE_CISLO_PAR}} & KMENOVE C. (EDI.) \\ \hline
          \texttt{\detokenize{PU_KATEGORIE}} & KATEGORIE \\ \hline
          \texttt{\detokenize{VYMERA_PARCELY}} & VYMERA (SPI) \\ \hline
          \texttt{\detokenize{PU_VYMERA_PARCELY}} & VYMERA (SGI) \\ \hline
          \begin{tabular}{@{}l@{}} \texttt{\detokenize{PU_VYMERA_PARCELY}} \\ \texttt{\detokenize{_ABS_ROZDIL}} \end{tabular} & ROZ. VYMER \\ \hline
          \begin{tabular}{@{}l@{}} \texttt{\detokenize{PU_VYMERA_PARCELY}} \\ \texttt{\detokenize{_MEZNI_ODCHYLKA}} \end{tabular} & MEZ. ODCH. ROZ. VYMER \\ \hline
          \texttt{\detokenize{PU_VZDALENOST}} & VZDALENOST \\ \hline
          \texttt{\detokenize{PU_CENA}} & CELK. CENA \\ \hline
          \begin{tabular}{@{}l@{}} \texttt{\detokenize{PU_BPEJ}} \\ \texttt{\detokenize{_BPEJCENA_VYMERA_CENA}} \end{tabular} & \begin{tabular}{@{}l@{}} BPEJ KOD-CENA ZA M2 \\ -VYMERA-CENA \end{tabular} \\ \hline
          \texttt{\detokenize{PU_MERITKO_PODKLADU}} & MERITKO PODKL. \\
         \hline
    \end{tabular}
    \centering
    \caption[Vrstva \texttt{\zk{PAR}} - viditelné sloupce a~aliasy]{Vrstva \texttt{\zk{PAR}} - viditelné sloupce a~aliasy}
    \label{tab:viditelne_sloupce_aliasy_par}
\end{table}

\newpage

\section{Editace}
\label{editace}

Velmi důležitou činností běhěm přípravné fáze pozemkových úprav je určení obvodu pozemkové úpravy a~rozdělení parcel do~kategorií (viz~\ref{obvod_a_predmet_pu}).

Algoritmus načtení \zk{VFK} souboru otvírá vrstvu parcel pomocí SQLite Driveru\footnote{Ve~verzi programu QGIS nižší než~2.18.5 je vrstva otevřená Spatialite Driverem, viz~\ref{nacteni_vfk_algoritmus}.}, takže ji lze editovat.

\subsection{Kategorie parcel}
\label{kategorie_parcel}

Kvůli zařazení parcel do~kategorií se běhěm načítání přidává do~vrstvy parcel sloupec \texttt{\detokenize{PU_KATEGORIE}} (alias KATEGORIE), jehož datový typ je celé číslo (\texttt{integer}). Zásuvný modul místo dlouhých názvů jednotlivých kategorií používá číslice \texttt{0} - \texttt{2}, viz~tab~\ref{tab:kategorie_hodnoty}. 

\begin{table}[H]
    \begin{tabular}{|l|l|}
        \hline
         hodnota & kategorie parcel \\
        \hline
        \hline
          \texttt{0} & mimo obvod \\ \hline
          \texttt{1} & v obvodu - neřešené \\ \hline          
          \texttt{2} & v obvodu - řešené \\
         \hline
    \end{tabular}
    \centering
    \caption[Sloupec \texttt{PU\textunderscore KATEGORIE} - hodnoty]{Sloupec \texttt{PU\textunderscore KATEGORIE} - hodnoty}
    \label{tab:kategorie_hodnoty}
\end{table}

Plugin disponuje mechanismy pro~nastavení této hodnoty (viz \ref{nastaveni_hodnoty_kategorie}) a~pro~výběr prvků v~kategorii (viz \ref{vyber_v_kategorii}).

{\scriptsize
\begin{lstlisting}[style=python, caption={Kategorie parcel - nastavení hodnoty}, captionpos=b, label=nastaveni_hodnoty_kategorie, backgroundcolor = \color{light-gray},  numbers=left]
fieldId = layer.fieldNameIndex('PU_KATEGORIE')

layer.startEditing()
layer.updateFields()

for feature in features:
    if feature.attribute('PU_KATEGORIE') != value:
        id = feature.id()
        layer.changeAttributeValue(id, fieldId, value)

layer.commitChanges()
\end{lstlisting}}

{\scriptsize
\begin{lstlisting}[style=python, caption={Kategorie parcel - výběr parcel v~kategorii}, captionpos=b, label=vyber_v_kategorii, backgroundcolor = \color{light-gray},  numbers=left]
expression = QgsExpression("\"PU_KATEGORIE\" = {}".format(value))
features = layer.getFeatures(QgsFeatureRequest(expression))

ids = [feature.id() for feature in features]
layer.selectByIds(ids)
\end{lstlisting}}

\subsection{Vrstva obvodu}
\label{vrstva_obvodu}

Obvod pozemkové úpravy je území dotčené pozemkovými úpravami, patří do~něj tedy parcely zařazené do~jednotlivých kategorií.

Bývá znázorňován tak, že všechny sousedící pozemky ve~stejné kategorii tvoří pouze jeden prvek, u~kterého nejsou viditelné vnitřní hranice. Tyto větší sloučené prvky je zvykem doplňovat o~popisky, které značí, do které kategorie náleží.

Zásuvný modul vytváří vrstvu obvodu na~základě sloupce \texttt{\detokenize{PU_KATEGORIE}} ve~vrstvě parcel. Nejprve je na~vrstvu \texttt{\zk{PAR}} zavolán nástroj \textit{Dissolve} s~parametrem sloupce \texttt{\detokenize{PU_KATEGORIE}}. Díky tomu se sousedící parcely v~jednotlivých kategoriích sloučí do~větších celků. Programu QGIS funguje tak, že prvky, které jsou tvořeny několika body, liniemi, nebo polygony, mají pouze jeden popisek. Výstupem nástroje \textit{Dissolve} mohou být i~multipolygony (prvky tvořeny více polygony), proto je nutné zavolat funkci \textit{Multiparts to~singleparts}, která problémové multipolygony rozdělí. Nakonec se odstraní prvky, které mají nulovou hodnotu ve~sloupci \texttt{\detokenize{PU_KATEGORIE}}, neboť takové prvky do vrstvy obvodu nepatří. Ukázka kódu \ref{obvod_tvorba} obsahuje volání nástrojů \textit{Dissolve} a~\textit{Multiparts to~singleparts}.

{\scriptsize
\begin{lstlisting}[style=python, caption={Vrstva obvodu - tvorba}, captionpos=b, label=obvod_tvorba, backgroundcolor = \color{light-gray},  numbers=left]
tempPerimeterLayerPath = processing.runalg(
    'qgis:dissolve',
    layer, False, 'PU_KATEGORIE', None)['OUTPUT']
tempPerimeterLayer = QgsVectorLayer(
    tempPerimeterLayerPath, tempPerimeterLayerName, 'ogr')

processing.runalg(
    'qgis:multiparttosingleparts',
    tempPerimeterLayer, perimeterLayerFilePath)
perimeterLayer = QgsVectorLayer(
    perimeterLayerFilePath, perimeterLayerName, 'ogr')
\end{lstlisting}}

\subsection{Symbologie vrstvy obvodu}
\label{symbologie_obvod}

Symbologie vrstvy obvodu se stejně jako u~vrstvy parcel řídí podle QML souboru. V~popiscích vrstvy jsou hodnoty sloupce \texttt{\detokenize{PU_KATEGORIE}}, jejich význam je popsán v~tab.~\ref{tab:kategorie_hodnoty}. Popisky se zobrazují při jakémkoli měřítku.

\subsection{Atributová tabulka vrstvy obvodu}
\label{tabulka_obvod}

Vrstva obvodu se vytváří z vrstvy parcel, ovšem pouze informace o~kategorii je pro~obvod relevantní. Z~toho důvodu je viditelný pouze sloupec \texttt{\detokenize{PU_KATEGORIE}}, viz tab.~\ref{tab:viditelne_sloupce_aliasy_obvod}.

\begin{table}[H]
    \begin{tabular}{|l|l|}
        \hline
         název sloupce & alias \\
        \hline
        \hline
          \texttt{\detokenize{PU_KATEGORIE}} & KATEGORIE \\
         \hline
    \end{tabular}
    \centering
    \caption[Vrstva obvodu - viditelné sloupce a~aliasy]{Vrstva obvodu - viditelné sloupce a~aliasy}
    \label{tab:viditelne_sloupce_aliasy_obvod}
\end{table}

\newpage

\section{Kontroly a analýzy}
\label{kontroly_analyzy}

Během přípravné fáze je nutné zkontrolovat soulad \zk{SPI} a~\zk{SGI} dat katastru nemovitostí, ověřit správnost rozdělení parcel~do kategorií a~také provést analýzy pro~sestavení vstupních soupisů nároků vlastníků.

\subsection{Kontroly}
\label{kontroly}

\subsubsection{Kontrola - obvodem}
\label{kontrola_obvodem}

Kontrola obvodem slouží k~výběru parcel, které se nenachází kompletně uvnitř vrstvy obvodu.

Do algoritmu vstupuje vrstva parcel a~vrstva obvodu. Použit je nástroj \textit{Select by~location} s~geometrickým predikátem \textit{within} a~poté je zavolána funkce pro~převrácení výběru prvků, viz ukázka kódu~\ref{kontrola_obvodem_kod}.

{\scriptsize
\begin{lstlisting}[style=python, caption={Kontrola \textit{obvodem} - výběr prvků}, captionpos=b, label=kontrola_obvodem_kod, backgroundcolor = \color{light-gray},  numbers=left]
processing.runalg(
    'qgis:selectbylocation',
    layer, perimeterLayer, u'within', 0, 0)

layer.invertSelection()
\end{lstlisting}}

\subsubsection{Kontrola - není v SPI}
\label{kontrola_neni_v_spi}

Jak vyplývá z~názvu, kontrola \textit{není v~SPI} provádí výběr parcel, které nejsou uvedeny v~souboru popisných informací. Pro výběr používá sloupec \texttt{\detokenize{KMENOVE_CISLO_PAR}}, neboť patří mezi povinně vyplněné \citep{struktura_vfk}. Pokud má parcela tento sloupec prázdný, znamená to, že se jedná o~chybu nebo nově vytvořenu parcelu.

{\scriptsize
\begin{lstlisting}[style=python, caption={Kontrola \textit{není v~SPI} - vzorec pro~výběr prvků}, captionpos=b, label=kontrola_spi_kod, backgroundcolor = \color{light-gray},  numbers=left]
expression = QgsExpression("\"KMENOVE_CISLO_PAR\" is null")
\end{lstlisting}}

\subsubsection{Kontrola - není v mapě}
\label{kontrola_neni_v_mape}

Výsledkem kontroly \textit{není v~mapě} je výběr parcel, které mají nulovou geometrii a~tudíž se nezobrazují v~mapovém okně. Vybrány jsou též parcely s~nevalidní geometrií.

{\scriptsize
\begin{lstlisting}[style=python, caption={Kontrola \textit{není v~mapě} - vzorec pro~výběr prvků}, captionpos=b, label=kontrola_mapa_kod, backgroundcolor = \color{light-gray},  numbers=left]
expression = QgsExpression("$geometry is null")
\end{lstlisting}}

\subsubsection{Kontrola - výměra nad mezní odchylkou}
\label{kontrola_vymera}

Kontrola \textit{výměra nad~mezní odchylkou} zjišťuje, jestli~rozdíl mezi výměrou dle~souboru popisných informací a~výměrou danou souborem geodetickým informací překračuje mezní odchylku. Hodnota mezní odchylky závisí na~kódu kvality nejméně přesně určeného lomového bodu na~hranici parcely \citep{vyhlaska_357}, viz tab.~\ref{tab:odchylky_vymer}. Pro~digitalizované parcely se kód kvality podrobných bodů určí podle~měřítka podkladové mapy, viz tab.~\ref{tab:kody_kvality_digit}.

Algoritmus z~vrstvy parcel nejdříve vyfiltruje prvky, které mají validní geometrii a~zadanou výměru podle \zk{SPI}. Poté v~cyklu všemi takovými prvky prochází. Pro~identifikaci parcel, které byly digitalizované, slouží sloupec \texttt{\detokenize{PU_MERITKO_PODKLADU}}. Hodnota 1 značí, že parcela nemá validní geometrii, jiné číslo udává měřítko podkladové mapy. Pokud je tedy v~tomto sloupci uvedeno číslo různé od~1, znamená to, že~se jedná o~digitalizovanou parcelu. V~takovém případě algoritmus zjistí kód kvality podrobných bodů podle tab.~\ref{tab:kody_kvality_digit}. Pomocí lomového bodu s~největším kódem kvality se vypočte mezní odchylka výměr a~porovná se s~absolutním rozdílem výměr dle~\zk{SPI} a~\zk{SGI}. Když je mezní odchylka překročena, přidá se do~výběru. V~momentě, kdy už není k~dispozici žádný další prvek, se kontrola ukončí. Celý postup znázorňuje diagram na obr.~\ref{fig:diagram_vymera}.

\subsubsection{Kontrola - bez vlastníka}
\label{kontrola_bez_vlastnika}

Kontrola \textit{bez~vlastníka} používá sloupec \texttt{\detokenize{TEL_ID}} pro~výběr parcel, které jsou bez~vlastníka, tzn. že~nemají přiřazený list vlastnictví, viz ukázka kódu~\ref{kontrola_vlastnik_kod}. Takové parcely se označují jako \textit{LV~0}.

{\scriptsize
\begin{lstlisting}[style=python, caption={Kontrola \textit{bez~vlastníka} - vzorec pro~výběr prvků}, captionpos=b, label=kontrola_vlastnik_kod, backgroundcolor = \color{light-gray},  numbers=left]
expression = QgsExpression("\"TEL_ID\" is null")
\end{lstlisting}}

	\begin{figure}[H]
		\centering
		\includegraphics[width=1.2\textwidth]{./pictures/vymera.pdf}
		\caption[Kontrola \textit{výměra nad mezní odchylkou} - diagram algoritmu]{Kontrola \textit{výměra nad mezní odchylkou} - diagram algoritmu}
		\label{fig:diagram_vymera}
 	\end{figure}

\subsection{Analýzy}
\label{analyzy}

\subsubsection{Analýza - měření vzdálenosti}
\label{analyza_vzdalenosti}

Analýza \textit{měření vzdálenosti} určuje pro~všechny parcely v~kategorii \textit{v~obvodu~-~řešené~(2)} vzdálenost jejich těžiště od~referenčního bodu, viz ukázka kódu~\ref{analyza_vzdalenost_vypocet_vzdalenosti_teziste_od_ref_bodu}. Výsledné zaokrouhlené hodnoty v~metrech ukládá do~sloupce \texttt{\detokenize{PU_VZDALENOST}}

Do kontroly kromě vrstvy parcel vstupuje i~vrstva referečního bodu, která musí obsahovat právě jeden prvek a~kvůli zamezení neočekávaných výsledků musí mít stejný souřadnicový systém jako vrstva parcel.

{\scriptsize
\begin{lstlisting}[style=python, caption={Analýza \textit{měření vzdálenosti} - výpočet vzdálenosti težiště od~refenčního bodu}, captionpos=b, label=analyza_vzdalenost_vypocet_vzdalenosti_teziste_od_ref_bodu, backgroundcolor = \color{light-gray},  numbers=left]
centroid = geometry.centroid().asPoint()
distanceDouble = sqrt(refPoint.sqrDist(centroid))
distance = int(round(distanceDouble))
\end{lstlisting}}

\subsubsection{Analýza - oceňování podle BPEJ}
\label{analyza_bpej}

Analýza \textit{oceňování podle BPEJ} vypočítá cenu pozemku na~základě vrstvy hranic \zk{BPEJ}.

Pro určení ceny za~metr čtvereční jednotlivých kódů \zk{BPEJ} analýza používá číselník \zk{BPEJ} z Českého úřadu zeměměřičského a katastrálního\footnote{Informace o~číselníku jsou dostupné na~\url{https://goo.gl/uXf8FC}. Samotný číselník lze stáhnout z~\url{http://www.cuzk.cz/CUZK/media/CiselnikyISKN/SC_BPEJ/SC_BPEJ.zip?ext=.zip}.}. Tento číselník je aktualizován každý den kolem třetí hodiny ranní.

Do algoritmu vstupují vrstvy \texttt{\zk{PAR}} a~hranice \zk{BPEJ}, na~které je volán nástroj vektorového překryvu \textit{Union}. Poté se zkontroluje aktuálnost číselníku \zk{BPEJ}. Jesliže číselník není aktuální a~lze se připojit k~internetu\footnote{Pro testování internetového připojení byla zvolena adresa \url{https://www.google.com}.}, stáhne zásuvný modul nový číselník. V~dalším kroku se z~nejnovějšího dostupného číselníku přečtou data a~vypočítá se cena. Do~atributové tabulky se zápíše nejen cena celková (sloupec \texttt{\detokenize{PU_CENA}}), ale~také cena za~metr čtvereční, výměra a~cena dle jednotlivých bonit v~příslušné parcele (sloupec \texttt{\detokenize{PU_BPEJ_BPEJCENA_VYMERA_CENA}}). Může se stát, že uživatel zvolí špatný sloupec, nebo že kód \zk{BPEJ} nebude uveden v~číselníku. V~takovém případě plugin vybere ve~vrstvě obvodu prvky, pro~které nenalezl ceny, a~informuje uživatele o~problému.

Algoritmus počítá i~s~možností změny adresy pro~stažení číselníku, když tato situace nastane, oznámí to uživateli.

Princip algoritmu je znázorněn na obr. \ref{fig:diagram_bpej}.

	\begin{figure}[H]
		%%\centering
		\includegraphics[width=1.2\textwidth]{./pictures/bpej.pdf}
		\caption[Analýza \textit{oceňování podle BPEJ} - diagram algoritmu]{Analýza \textit{oceňování podle BPEJ} - diagram algoritmu}
		\label{fig:diagram_bpej}
 	\end{figure}

%%BPEJ je zjednodusene - viz 9.8 http://www.spucr.cz/frontend/webroot/uploads/files/2015/12/metodickynavodkprovadenipozemkovychuprav1327.pdf